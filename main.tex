\documentclass{book}
% import packages
\usepackage[margin=2.5cm]{geometry} 
\usepackage[utf8]{inputenc}
\usepackage[ngerman]{babel}
\usepackage[backend=biber]{biblatex}
\usepackage{physics}
\usepackage{subcaption}
%\usepackage{fancyref}
\usepackage{booktabs}
\usepackage{bbold}
\usepackage{bm}
\usepackage{graphicx}
\usepackage{mathtools}
\usepackage{cancel}
\usepackage{wrapfig}
\usepackage{float}
\usepackage{chngcntr}
\usepackage{multicol}
\usepackage{makecell}

% import commands
\newcommand{\mrm}[1]{\mathrm{#1}}
\newcommand{\unit}[1]{\,\mathrm{#1}}
\newcommand{\unitp}[1]{\,(\mathrm{#1})}
%\renewcommand{\thefootnote}{\fnsymbol{footnote}}
\newcommand{\ds}{\displaystyle} 
%\newcommand{\diff}[3]{\qty(\pdv{#1}{#2})_{#3}} 
%\newcommand{\EXP}[1]{\langle#1\rangle} 
%\newcommand{\x}{\cdot} 

\newcommand{\green}[1][]{G_{#1}(\vb*r,\vb*r')}
\newcommand{\dA}{d\vb A}
\newcommand{\ddA}{\cdot d\vb A}
\newcommand{\vr}{\vb*r}
\newcommand{\rr}{\abs{\vb*r-\vb*r'}}
\newcommand{\frr}[1][1]{\frac{#1}{\abs{\vb*r-\vb*r'}}}

% import variables
\newcommand{\thetitle}{Zusammenfassung Theoretische Physik II:\@ Elektrodynamik}

\newcommand{\theAuthor}{Luc Jean Bert Kusters}
\newcommand{\BAenddatemonth}{September 2020}

\newcommand{\subtexdir}{subtex}

\setcounter{tocdepth}{2}
\counterwithin{equation}{subsection}

% make sure to count figures within sections
\counterwithin{figure}{section}
\counterwithin{table}{section}
\counterwithin{equation}{section}

\font\myfont=cmr12 at 30pt
% use mathrm as standard math font
% \DeclareSymbolFont{letters}{OT1}{cmr}{m}{n}
\begin{document}
  \pagenumbering{roman}
  \thispagestyle{empty}
  \begin{titlepage}
     \begin{center}
        \vspace*{1cm}
        
        \textbf{\myfont\thetitle}\\
        \vspace{1.5cm}
        \large
        durch
        \vspace{1.5cm}
        
        \Large
          \textbf{\theAuthor}
        
        \vspace{1.5cm}
    \end{center}
  \end{titlepage}
  
  % back titlepage
  \vfill
  \textbf{\author}\\
  \normalsize
  \thispagestyle{empty}

  \newpage
  \section{Einleitung}\label{cha:Einleitung}
  Diese Zusammenfassung wurde geschrieben um meine persönliches Verständniss 
zum Thema Elektrodynamik zu verbessern, denn man lernt am Meisten indem man
versucht, sich in eine lehrende weise mit die Themen auseinanderzusetzen.
Daneben finde ich persönlich das viele Skripte die ich gesehen habe einen 
unübersichtlichen überblick von der Elektrodynamik gaben, teilweise durch
eine Informationsoverload und teilweise wegen oft fehlende motivierung der
verschiedene Themenbereiche. Diese Zusammenfassung versucht ein bottom-up
konstruierte Übersicht zu geben des Themas. 

Natürlich ist eine Zusammenfassung nie so vollständig wie Skripte zur Vorlesungen oder Literatur. Auch Übungsaufgaben und Beispiele werden hier großenteils für kompaktheit aus der Zusammenfassung gelassen. Theorie verstehen und 
rechnen können sind immer zwei unterschiedliche Sachen, also zum 
vollständigen Verständniss des Themas muss man auch viele Zusammenhänge
selbst Herleiten, sodass man eine intuition aufbaut wie man die Theorie in
der Praxis anwendet. Dazu kann man am Besten Übungsaufgaben von der Üniversität widerholen oder aus Literatur Rechnenaufgaben machen.

Weil Deutsch nicht meine Muttersprache ist gibt es leider bestimmt welche Schreibfehler. Ich versuche sie so viel wie Möglich rauszunehmen wo ich sie sehe.

  \dominitoc[n]
  \tableofcontents 
  
  \newpage
  \pagenumbering{arabic}
  \setcounter{page}{1}
\setcounter{minitocdepth}{3}
\nomtcrule%
  \chapter{Elektrostatik}\label{cha:Elektrostatik}
    \minitoc%
    \clearpage
    \section{Ladungen und Ladungsverteilungen}%
\label{sub:Ladungen-und-Ladungsverteilungen}
Die Ursprung der EM-Kraft und EM-Felder ist die \textbf{Ladung}. Mathematisch wird die
Ladungsverteilung eines Systems in Raum durch die \textbf{Ladungsdichte} 
gegeben. Ganz algemein
kann eine Ladungsverteilung von N Punktladungen am Zeit t an den Orten $\vb*x_i$ wie folgt beschrieben werden

\begin{equation}
  \rho(\vb*r, t)= \sum_{i=1}^N q_i \delta(\vb*r - \vb*x_i(t))
\end{equation}
Wobei $q_i$ sowohl positiv oder negativ sein kann.

Prinzipiell ist die Ladung in der Natur eine Eigenschaft von Punktteilchen,
sodass es eigentlich keine kontinuierliche Ladungsverteilungen gibt. Auf
makroskopische Skala ist diese Beschreibung jedoch sehr unhandlich, und ist
eine Mittelung der Ladungsdichte eine wesentlich einfachere Beschreibung.
Eine mögliche Definition für eine kontinuierliche Ladungsverteilung aus einer
diskreten Ladungsverteilung sieht wie folgt aus
\begin{equation}
  \rho_{\textrm{knt.}}(\vb*r, t) 
  = 
  \frac{1}{\Delta V(\vb*r)} 
  \int_{\Delta V(\vr)}d^3r'\rho_{\text{dskr}}(\vr')
  = 
  \frac{1}{\Delta V(\vb*r)} 
  \sum_{q_i\in\Delta V(\vr)}q_i
\end{equation}
wobei man das Volumenkästchen $\Delta V$ um $\vr$ zentriert kontinuierlich verschieben kann, und annimt, daß es viele Ladungen gibt, sodaß die Kontinuierliche Ladungsverteilung praktisch glatt wird.

Für Physikalische Systeme betrachtet man in der Regel immer nur beschränkte 
Ladungsverteilungen, das will sagen, die Ladungsdichte soll nur auf 
ein beschränktes Raumbereich von 0 verschieden sein. 
Dazu dürfen Ladungsverteilungen auch keine Pollstellen haben, 
d.h.\ nie divergieren. 
In manche theoretische Fälle, gibt es auch Ladungsverteilungen die 
nicht beschränkt sind, sondern hinreichend schnell im Unendlichen 
verschwinden (z.B.\ exponentiell unterdrückt), sodass zumindest
\begin{equation*}
  \int d^3r \rho(\vb*r, t) = Q_\textrm{ges} < \infty
\end{equation*}
gilt.

In der Elektrostatik betrachtet man erstmal ruhende Ladungsverteilungen, 
d.h.
\begin{equation}
  \rho(\vb*r, t) = \rho(\vb*r)
\end{equation}
Es gibt also keine Ströme. Manchmal werden in der Elektrostatik auch 
quasi statische Prozeße betrachtet, wo die Effekte von bewegte Ladungen 
erstmal vernachlässicht werden. Die Elektrostatik ist somit ein Spezialfall
der Elektrodynamik.


\newpage
\section{Elektrische Kraft, Elektrisches Feld und Elektrisches Potential}%
\label{sub:Ladungen-und-Ladungsverteilungen}
\subsection{Coulombkraft}%
\label{ssub:Coulombkraft}
Schon in 1785 gelang Charles Augustin de Coulomb eine Beschreibung der
Elektrische Kraft, welche man heutzutage dann auch die 
\textbf{Coulombkraft} 
nennt. Sie ist sehr ähnlich zum Newtons Schwerkraftsgesetz und lautet für 
zwei Punktladungen $q_1$ und $q_2$ an den Orten $\vb*r_1$ und $\vb*r_2$ wie 
folgt (im Vakuum)

\begin{equation}
  \vb F_{1\to2}(\vb*r_1, \vb*r_2, q_1, q_2)= k q_1 q_2 \frac{\vb*r_2-\vb*r_1}{\abs{\vb*r_2-\vb*r_1}^3} 
\end{equation}

\noindent
\begin{center}
Oder in bekanntere Form (nicht vektoriell, Punktladungen auf Abstand r)
\end{center}
\begin{equation*}
  F_c = k \frac{q_1 q_2}{r^2} 
\end{equation*}
wobei $\vb F_{1\to2}$ die Kraft ist, die Teilchen 1 auf Teilchen 2 ausübt. 
Sei $q_1q_2>0$ so wirkt die Kraft abstoßend, 
und sei $q_1q_2<0$, so wirkt sie anziehend. Diese Gleichung wird auch das
\textbf{Coulomb-Gesetz} genannt. 
Dabei ist $k$ eine Konstante die vom Einheitensystem abhängt. In SI gilt z.B. 
$k=\frac{1}{4\pi\epsilon_0}$ mit $\epsilon_0$ die \textit{elekrische Permitivität} des Vakuums. In Gauß-Einheiten gilt $k=1$.

Die Elektrische Kraft ist eine Zentralkraft und deswegen 
\textbf{konservativ}, d.h. 
\begin{equation}
  \oint \vb F(\vb* r) \cdot d\vb*s = 0 \stackrel{*}{\Leftrightarrow} \curl \vb F =0
\end{equation}
\begin{center}
(*Satz von Stokes)
\end{center}
Dies heißt auch, dass die Coulombkraft ein skalares Potential $V(\vb*r)$ besitzt, mit
$\vb F(\vb*r)=-\grad V(\vb*r)$

\subsection{Das Elekrtische Feld}%
\label{ssub:E-feld}
Neben die Kraft, die nur über die Wechselwirkung zweier Teilchen definiert
ist, ist es Sinnvoll das \textbf{Elektrische Feld} zu definieren. Man 
definiert das Elektrische Feld aus ersten Prinzipien aus der Coulombkraft.

\begin{equation*}
  \vb F(q_1, q_2, \vb*r_1, \vb*r_2) 
  = 
  q_1 
  \qty(k q_2 \frac{\vb*r_2 - \vb*r_1}{\abs{\vb*r_2 - \vb*r_1}^3})
  = q_1 \vb E_2(\vb*r_1)
\end{equation*}
Man verstehe dies so, dass Teilchen 1 mit Ladung $q_1$ 
sich am Ort $\vb*r_1$ im äußeren elektrisches Feld $\vb E_2$, 
erzeugt von Teilchen 2 mit Ladung $q_2$ welches am Ort $r_2$ liegt, befindet und daher eine Kraft spürt.

Eine Probeladung $q$ im externen elektrischen Feld $\vb E$ spürt also die Kraft

\begin{equation}
  \vb F(\vb*r) = q \vb E(\vb*r)
\end{equation}
Das Elektrische Feld einer Ladungsverteilung $\rho_1$ ist dann ein 
maß für die Kraft, die eine andere, äußere Ladungsverteilung $\rho_2$ 
wegen der Anwesendheid von $\rho_1$ spüren würde. 

\subsection{Das Elekrtische Potential}%
\label{ssub:E-pot}
Weil die Coulombkraft ein
Skalares Potential $V(\vb*r)$ besitzt, 
besitzt das Elektrische Feld auch ein skalares \textbf{Elektrisches Potential}
$\phi(\vb*r)$ mit
\begin{equation}
  \vb E(\vb*r)=-\grad \phi(\vb*r)
\end{equation}

Nun wollen wir wissen, wie man aus eine beliebige 
Ladungsverteilung $\rho(\vb*r)$ das zugehörige
elekrische Potential $\phi(\vb*r)$ bzw.\ elektrische Feld $\vb E(\vb*r)$ 
findet. Aus der Coulombkraft kann man herleiten, dass das Potential
einer Punktladung $Q$ welches im Ursprung liegt gegeben ist durch
\begin{equation*}
  \phi(\vb*r) = k  \frac{Q}{r} \quad r=\abs{\vb*r}
\end{equation*}
Denn $\vb E(\vb*r)=-\grad\phi(\vb*r)=k \frac{Q}{r^2}\vu e_r $ 
woraus die Coulombkraft $F_c(\vb*r)=k \frac{qQ}{r^2}$ wieder folgt.
Liegt das Teilchen nicht im Ursprung, sondern am beliebigen Ort, $\vb*r_0$
so findet man den Zusammenhang (durch eine einfache Translation)
\begin{equation*}
  \phi(\vb*r) = k \frac{Q}{\abs{\vb*r - \vb*r_0}}
\end{equation*}
Das Potential mehrere Punktteilchen ergibt sich aus der addition für die 
Potenziale der einzelnen Teilchen.
\begin{equation*}
  \phi(\vb*r) = \sum_i \phi_i(\vb*r) = k\sum_i \frac{q_i}{\abs{\vb*r-\vb*r_i}}
\end{equation*}
Daraus lässt sich dann über das Limesprozeß einer Riemannsche Summe das 
Potential einer kontinuierliche Ladungsverteilung $\rho(\vb*r)$ definieren.
\begin{equation}
  \label{eq:potential}
  \phi(\vb*r) = k \int d^3 r' \frac{\rho(\vb*r')}{\abs{\vb*r-\vb*r'}} 
\end{equation}

Man kann auch direkt das E-Feld aus die Ladungsverteilung berechnen, falls 
man den Gradient (nach $\vb*r$, nicht $\vb*r'$) von der obere Formel nimmt, 
und findet
\begin{equation}%
  \label{eq:E-feld}
  \vb E(\vb*r) = k \int d^3r' 
  \rho(\vb*r')\frac{(\vb*r-\vb*r')}{\abs{\vb*r-\vb*r'}^3} 
\end{equation}

Das elektrische Potential ist aber nicht eindeutig fetgelegt, denn die
\textbf{Eichtransformation}
\begin{equation}
  \phi'(\vr) = \phi(\vr) + \phi_0 \quad \phi_0=\const
\end{equation}
läßt das Elektrische Feld invariant (denn $\grad\phi_0=0$). Hierdurch wird im
Allgeimeinen das Potential nur als hilfreiches mathematisches Hilfsmittel
gesehen, obwohl quantenmechanische Effekte wie das Aharanov-Bohm-Effekt auch
darauf weisen können, dass das Potential auch ein physikalisches Feld ist. Dies ist aber erstmal nicht wichtig für die Einführung in der Elektrodynamik.
Meistens wird $\phi_0$ so gewählt, sodass $\phi(\vr\to\infty)\to0$,
dies ist aber nicht notwendig.

\subsection{Die Feldgleichungen der Elektrostatik}%
\label{ssub:Die-Feldgleichungen}
Die Divergenz des elektrischen Feldes gibt nun die Quellendichte des Feldes.
Wir wissen schon aus Erfahrung, dass die elektrische Ladung die Quelle der 
Coulombkraft ist und deswegen auch des elektrischen Feldes. Die genaue Zusammenfang zwischen die Divergenz und die Ladungsdichte folgt aus dem Zerlegungssatz. Zusammen mit die
Rotationsfreiheit der Coulombkraft, und somit auch die Rotationsfreiheit des
E-Feldes folgen die beiden \textbf{Feldgleichungen der Elektrostatik}.
\begin{equation}
  \begin{aligned}
    \div \vb E &= 4\pi k \rho(\vb*r) & \text{(inhomogen)}\\
    \curl \vb E &= 0 & \text{(homogen)}
  \end{aligned}
\end{equation}
Diese Feldgleichungen sind allgemeingültig für alle elektrostatische Felder.

\subsection{Poisson Gleichung}%
Weiter folgt aus den Zusammenhänge $\vb E = -\grad \phi$ und 
$\div E = 4\pi k \rho(\vb*r)$ die \textbf{Poisson Gleichung} 
\begin{equation}
  \Delta \phi(\vb*r) = - 4\pi k \rho(\vb*r)
\end{equation}
Die Poisson-Gleichung ist eine Differentialgleichung 2.\ Ordnung. Unter 
vorgabe von \textbf{Randbedingungen} (das geometrische Analogon zu Anfangsbedinungen von DGLS in der Mechanik) sind die Gleichungen~\ref{eq:potential} und~\ref{eq:E-feld} meistens nicht mehr ausreichend, um korrekte Lösungen zu finden. 
Wir werden Später sehen, das sie nur in dem Spezialfall von verschwindenden
Randbedingungen gültig sind, und die allgemeine Zusammenhänge formaler aus der Poisson-Gleichung herleiten.
Das Lösen der Poisson-Gleichung ist aber ein Thema für ein eigenes Kapitel. 

\subsection{Die Hauptaufgabe der Elektrostatik} 
Die Hauptaufgabe der Elektrostatik ist also das Berechnen von 
elektrostatische Potentiale und Felder unter vorgabe von 
Ladungsverteilungen und Randbedingungen. 
Für Hochsymmetrische Probleme sind im Allgemeinen analytische 
Lösungen möglich, für schwierigere Probleme können meistens nur numerische 
Lösungen gefunden werden. Für uns ist erstmal das Finden von analytische
Lösungen interessant um ein Grundverständnis aufzubauen. Dazu werden auch
Methoden besprochen die uns analytische Näherungen geben können, wie die
Multipolentwicklung.


\newpage
\section{Multipolentwicklung}%
\label{sub:Multipolentwicklung}
Zunächst besprechen wir wie man analytische Näherungen finden kann
unter angabe von komplexere Ladungsverteilungen. 
Das Problem ist meistens das Integrieren der
\begin{equation*}
  \frac{1}{\abs{\vb*r-\vb*r'}}
\end{equation*}
Term (in Kombination mit die Ladungsdichte $\rho(\vb*r')$). 
Deswegen möchten wir in der \textbf{karthesische Multipolentwicklung} 
diesen  Term Taylor-entwickln, damit wir es in eine 
Reihe von Polinomiale Terme umwandeln können, 
weil diese Einfach(er) zu integrieren ist. 
Daneben gibt es noch die \textbf{Kugelfächen-Entwicklung} die sich 
insbesondere für Radial- bzw. Rotationssymmetrische Probleme eignet, 
die wir Später besprechen werden. 

\subsection{Karthesische Multipolentwicklung}%
\label{ssub:Karthesische-Multipolentwicklung}
Die karthesische Multipolentwicklung wird mittels eine Taylor-Entwicklung
hergeleitet. Man muss dabei eine Multidimensionale Taylor-Entwicklung 
durchführen.

Eine allgemeine mehrdimensionale Taylorentwicklung wird gegeben durch:
\begin{equation*}
  f(\vb*\alpha, \vb*\beta)
  =\qty(\exp(\vb*\beta\cdot\nabla_{\vb*\beta'})
  f(\vb*\alpha, \vb*\beta'))
  \bigg|_{\vb*\beta'=\vb*\beta_0}
  =\qty(\sum_{n=0}^\infty
  \frac{(\vb*\beta\cdot\nabla_{\vb*\beta'})^n}{n!}
  f(\vb*\alpha, \vb*\beta'))
  \bigg|_{\vb*\beta'=\vb*\beta_0}
\end{equation*}
Man bemerke dass $\nabla$ einen Operator ist! Die exponential Funktion
dient hier nur zur vereinfachung der Darstellung! Man Taylore nun um $\vb*r'=0$ (dies heißt, daß $\vb*r\gg\vb*r'$ sodass 
$\vb*r-\vb*r'\approx\vb*r$, dafür muss $\vb*r$ natürlich weit
von der Quelle entfernt sein). Die Multipolentwicklung ist also ein 
\textbf{Fernfeldnäherung}.

Man definiere nun $f(\vb*r,\vb*r')=f(r_1,r_2,r_3,r_1',r_2',r_3')
\equiv\frac{1}{\abs{\vb*r-\vb*r'}}$ sodass bis zur 2. Ordnung die Taylorentwicklung für unsere Funktion
wie folgt aussieht:

\begin{equation*}
  \frac{1}{\abs{\vb*r-\vb*r'}}=f(\vb*r, 0)
  + (\vb*r'\cdot \nabla_{\vb*{\bar{r}'}}) f(\vb*r, \vb*{\bar{r}}')\bigg|_{\vb*{\bar{r}}'=0}
  + \frac{1}{2}(\vb*r'\cdot\nabla_{\vb*{\bar r'}})^2
  f(\vb*r,\vb*{\bar r}')
  \bigg|_{\vb*{\bar r'}=0}
  + \ldots
\end{equation*}

\underline{0. Ordnung:}
\begin{equation*}
  \frac{1}{\abs{\vb*r-\vb*{\bar r'}}}\bigg|_{\vb*{\bar r'}=0} 
  = \frac{1}{\abs{\vb*r}}= \frac{1}{r} 
\end{equation*}

\underline{1. Ordnung:}
\begin{equation*}
  \vb*{r'}\cdot
  \nabla_{\vb*{\bar{r}'}}\frac{1}{\abs{\vb*r-\vb*{\bar{r}'}}}
  \bigg|_{\vb*{\bar r'}=0}
  = \frac{\vb*{ r'}\cdot\vb*r}{r^3} 
\end{equation*}

\underline{2. Ordnung:}
\begin{equation*}
  \begin{split}
  \frac{1}{2}(\vb*{\bar r'}\cdot\nabla_{\vb*{\bar{r}'}})^2
  \frac{1}{\abs{\vb*r-\vb*{\bar r'}}}\bigg|_{\vb*{\bar r'}=0}
  &=
  \frac{1}{2}\qty[r_i'\pdv {\bar r_i'}]\qty[r_j'\pdv{\bar r_j'}]
  \frac{1}{\abs{\vb*r-\vb*{\bar r'}}}
  \bigg|_{\vb*{\bar r'}=0}\\
  &=
  \frac{r'_ir'_j}{2} 
  \frac{\partial^2}{\partial_{\bar r_i'}\partial_{\bar r_j'}} 
  \frac{1}{\abs{\vb*r-\vb*{\bar r'}}}\bigg|_{\vb*{\bar r'}=0}\\
  &=
  \frac{r'_ir'_j}{2} \frac{\partial}{\partial_{\bar r_i'}} 
  \frac{x_j}{\abs{\vb*r-\vb*{\bar r'}}^3}\bigg|_{\vb*{\bar r'}=0}
  \qquad x_j\equiv (r_j-\bar r_j')\\
  &=
  \frac{r'_ir'_j}{2} \frac{\partial}{\partial_{\bar r_i'}} 
  \frac{x_j}{\abs{\vb*r-\vb*{\bar r'}}^3}\bigg|_{\vb*{\bar r'}=0}\\
  &=
  \frac{r'_ir'_j}{2} 
  \qty(
    \text{-}\frac{3}{2}
    \frac{\text{-}2x_i x_j}{\abs{\vb*r-\vb*{\bar r'}}^5}
    -\frac{\delta_{ij}}{\abs{\vb*r-\vb*{\bar r'}}^3} 
  )\bigg|_{\vb*{\bar r'}=0}\quad\text{(Produktregel)}\\
  &=
  \frac{1}{2}\sum_{i,j=1}^3 \frac{3r_ir_j-\delta_{ij}\vb*r^2}{r^5}
  r_i'r_j'
  \end{split}
\end{equation*}
\textit{Eine wichtige Bemerkung:}
Aus Symmetrie Grunden gilt
\begin{equation*}
  \frac{1}{2}\sum_{i,j=1}^3 \frac{3r_ir_j-\delta_{ij}\vb*r^2}{r^5}
  r_i'r_j'
  =
  \frac{1}{2}\sum_{i,j=1}^3 \frac{3r'_ir'_j-\delta_{ij}\vb*r'^2}{r^5}
  r_ir_j
\end{equation*}
Wir benutzen im allgemeinen die letzte Definition wenn wir die
elektrische und magnetische Multipole berechnen. Man findet also bis zum
2. Ordnung
\begin{equation}
  \frac{1}{\abs{\vb*r-\vb*r'}}
  \approx 
  \frac{1}{r} 
  + \frac{\vb*r'\cdot\vb*r}{r^3} 
  + \frac{1}{2}\sum_{i,j=1}^3 
    \frac{3r'_ir'_j-\delta_{ij}\vb*r'^2}{r^5}r_ir_j
\end{equation}
Setzt man dies in die Definition für das Elektrische Potential ein, so 
findet man
\begin{equation*}
  \begin{split}
    \phi(\vb*r) 
    &= k \int d^3 r' \frac{\rho(\vb*r')}{\abs{\vb*r - \vb*r'}} \\
    & \approx k\int d^3 r' \rho(\vb*r')
    \qty(
    \frac{1}{r} 
    + \frac{\vb*r'\cdot\vb*r}{r^3} 
    + \frac{1}{2}\sum_{i,j=1}^3 
    \frac{3r'_ir'_j-\delta_{ij}\vb*r'^2}{r^5}r_ir_j)\\
    &\equiv 
    k \frac{Q}{r} + k \frac{\vb*p\cdot\vb*r}{r^3} 
    + \frac{k}{2} \sum_{i,j=1}^{3} \frac{r_ir_j}{r^5}Q_{ij}
  \end{split}
\end{equation*}

\begin{center}
\begin{tabular}{ll}
  Monopol:    & $\ds Q=\int d^3r\rho(\vb* r)$ \quad\text(Gesammtladung)\\
  Dipol:      & $\ds p_i=\int d^3r\rho(\vb* r)r_i
                \quad\vb* p=p_i\vu{e}_i $\\
  Quadrupol:  & $\ds Q_{ij}=\int d^3r\rho(\vb* r)
  \qty(3r_ir_j - \delta_{ij}\abs{\vb* r}^2)$
\end{tabular}
\end{center}
Dabei hat die Quadrupoltensor $Q_{ij}$ nur 5 Freiheitsgraden. 
Mit nur 5 Rechnungen alle (9) Quadrupol Elemente berechnen. 
Es gilt zwar $Q_{ij}=Q_{ji}$ (symmertrisch) und $\text{sp}(\bm Q)=\sum_i Q_{ii}=0$ (spurfrei).

Im allgemeinen berechnet 
man keine weitere Ordnungen analytisch im Bachelorstudium, 
und zum Verständniss bringt dies auch nicht mehr (außer ärger), 
sodass höhere Ordnungen berechnen 
ein Problem ist das man lieber an Computer überlässt.

\subsection{Sphärische Multipolentwicklung}%
\label{ssub:sphaerische-Multipolentwicklung}
Für Ladungsverteilungen die Radial- oder Rotationssymetrisch sind ist die 
Sphärische Multipolentwicklung besonders geeignet, vor allem falls man die 
Ladungsdichte als Linearkombination von Kugelfächen-Funktionen schreiben 
kann. Um die Sphärische Multipolentwicklung zu motivieren machen wir 
zunächst die Annahme (mit dem Seperationsansatz), dass das 
Winkelanteil des Potentials unabhähngig vom Radialanteil ist, also
\begin{equation*}
  \phi(\vb*r) = \phi_r(r)\phi_\Omega(\theta,\varphi)
\end{equation*}
Weil man (wie wir später besprechen werden) die Poisson-Gleichung mit dem
gleichen Ansatz lösen wird, und die Kugelflächen-Funktionen 
$Y_{lm}(\theta,\varphi)$ eine Eigenfunktion der Laplace Operator ist, 
ist es eine gute Idee um dies als Basis für eine Sphärische Entwicklung 
zu benutzen. Dazu formen die Kugelflächen-Funktionen unter Integration über
$d\Omega=\sin\theta d\theta d\varphi$ eine orthonormale Basis. Es gilt die untere Zusammenhang.
\begin{equation*}
  \int d\Omega Y_{lm}(\Omega)Y^*_{l'm'}(\Omega)=\delta_{ll'}\delta_{mm'}
\end{equation*}

Die Entwicklung in Kugelkoordinaten ist rechnerisch etwas aufwändig, und
wird hier erstmal übersprungen, aber sie wird in gute Literatur über
die Elektrodynamik oft gegeben, wie z.B. in Kapitel 2.3.8 von Noltings
``Grundkurs Theoretische Physik 3, Elektrodynamik''. 
Man findet 
\begin{equation}
  \frac{1}{\abs{\vb*r-\vb*r'}}
  =
  \sum_{lm} \frac{4\pi}{2l+1} \frac{r_<^l}{r_>^{l+1}}
  Y^*_{lm}(\Omega')Y_{lm}(\Omega)
\end{equation}
\begin{center}
mit $l=0,1,2,\ldots$ und $m=-l,\ldots,l$\\
$r_<=\min(r,r')$, $r_>=\max(r,r')$.
\end{center}
Setzt man dies in der Definition für das Elektrische Potential ein, 
so findet man
\begin{equation}
  \begin{split}
    \phi(\vb*r) 
    &= k\int d^3r' \rho(\vb*r') \frac{1}{\abs{\vb*r-\vb*r'}} \\
    &= k\sum_{lm} \frac{4\pi}{2l+1} 
    \int_0^\infty dr'{r'}^2 \frac{r_<^l}{r_>^{l+1}} 
    \int_\Omega d\Omega' Y^*_{lm}(\Omega')Y_{lm}(\Omega) \rho(\vb*r')\\
    &= k\sum_{lm} \frac{4\pi}{2l+1} Y_{lm}(\Omega) 
    \qty(
    \int_0^{r} dr'{r'}^2 \frac{{r'}^l}{r^{l+1}} +
    \int_{r}^{\infty} dr'{r'}^2 \frac{r^l}{{r'}^{l+1}} 
    )
    \int_\Omega d\Omega' Y^*_{lm}(\Omega')\rho(\vb*r')\\
  \end{split}
\end{equation}
Sei nun die Ladungsdichte 
$\rho(\vb*r')=\sum_{l'm'}f_{f'm'}(r')Y_{l'm'}(\Omega')$
für beliebige $l'$ und $m'$ (also einfach irgendeine Linearkombination von
beliebige Kugelflächenfunktionen) so vereinfacht sich das Problem weiter,
und kann man für eine Endliche Summe sogar exakte Lösungen finden.
(Es ist nicht notwendig dass man die Ladungsdichte als Linearkombination
von Kugelflächenfunktionen schreibt, nur einfacher).
\begin{equation}
  \begin{split}
    \phi(\vb*r) 
    &= k\sum_{lm} \frac{4\pi}{2l+1} Y_{lm}(\Omega) \sum_{l'm'}
    \qty(
    \int_0^{r} dr'\frac{{r'}^(l+2)}{r^{l+1}}f_{l'm'}(r) +
    \int_{r}^{\infty} dr'\frac{r^l}{{r'}^{l-1}}f_{l'm'}(r) 
    )
    \int_\Omega d\Omega' Y^*_{lm}(\Omega')Y_{l'm'}(\Omega')\\
    &= k\sum_{lm} \frac{4\pi}{2l+1} Y_{lm}(\Omega) \sum_{l'm'}
    \qty(
    \int_0^{r} dr'\frac{{r'}^{l+2}}{r^{l+1}}f_{l'm'}(r) +
    \int_{r}^{\infty} dr'\frac{r^l}{{r'}^{l-1}}f_{l'm'}(r) 
    )
    \delta_{ll'}\delta_{mm'}\\
    &= k\sum_{l'm'} \frac{4\pi}{2l+1} Y_{l'm'}(\Omega)
    \qty(
    \int_0^{r} dr'\frac{{r'}^{l+2}}{r^{l+1}}f_{l'm'}(r) +
    \int_{r}^{\infty} dr'\frac{r^l}{{r'}^{l-1}}f_{l'm'}(r) 
    )
  \end{split}
\end{equation}
d.h.\ alle Terme wo $l\neq l'$ oder $m\neq m'$ fallen wegen dem Kronecker-delta weg. Wir werden aber weiter wieder von eine allgemeine Ladungsdichte 
ausgehen.

Ist die Ladungsdichte nun nach Innen oder Nach außen beschränkt, dann kann
man $r_>$ und $r_<$ eindeutig als $r$ oder $r'$ festlegen in bestimmte 
Raumbereiche. Hat man z.B. eine geladene Kugelschale wobei die 
Ladungsdichte, noch von $\theta$ und $\varphi$ abhängen darf, so kann man
das Potential im Inneren mit nur dem $\int_r^\infty dr$ Integral beschreiben, und
dem Außenraum mit durch das $\int_0^r dr$ Integral. Es macht manchmal 
sogar kein Sinn die Lösungen der beide Raumbereiche gleichzeitig zu 
berechnen weil die Lösung für das Innere Raumbereich im äußeren Raumbereich
divergiren kann oder umgekehrt. Wir Teilen hier also zunächst die 
Raumgebiete auf und finden Multipolmomente für dem innen und außen Räume 
$q_<^{lm}$ und $q_>^{lm}$\\

\noindent
Im Innenraum ($r\approx 0$):
\begin{equation}
  \begin{split}
    \phi(\vb*r) 
    &= k\sum_{lm} \frac{4\pi}{2l+1} Y_{lm}(\Omega)
    \int_0^{r} dr'\frac{{r}^{l}}{{r'}^{l-1}}
    \int_\Omega d\Omega'Y^*_{lm}(\Omega')\rho(\vb*r')\\
    &= k\sum_{lm} \frac{4\pi}{2l+1}r^l q_<^{lm} Y_{lm}(\Omega) 
  \end{split}
\end{equation}
\begin{center}
  mit $\ds q_<^{lm}=\int_0^\infty dr' {r'}^{1-l}
  \int_\Omega d\Omega' Y^*_{lm}(\Omega')\rho(\vb*r)$
\end{center}
\noindent
Im Außenraum ($r\gg0$):
\begin{equation}
  \begin{split}
    \phi(\vb*r) 
    &= k\sum_{lm} \frac{4\pi}{2l+1} Y_{lm}(\Omega)
    \int_0^{r} dr'\frac{{r'}^{l+2}}{r^{l+1}}
    \int_\Omega d\Omega'Y^*_{lm}(\Omega')\rho(\vb*r')\\
    &= k\sum_{lm} \frac{4\pi}{2l+1} 
    \frac{q_>^{lm}}{r^{l+1}} Y_{lm}(\Omega) 
  \end{split}
\end{equation}
\begin{center}
  mit $\ds q_>^{lm}=\int_0^\infty dr' {r'}^{l+2}
  \int_\Omega d\Omega' Y^*_{lm}(\Omega')\rho(\vb*r)$
\end{center}


\newpage
\section{Verhalten von Elektrostatische Felder an Randflächen}%
\label{sub:randflaechen}
In diesem Kapitel überlegen wir uns, wie Elektrische Felder sich an 
\textbf{Randflächen} verhalten. Dabei ist es wichtig dass wir die aus den
Vektorkalkulus folgenden \textbf{Satz von Gauß} und \textbf{Satz von Stokes}
näher betrachten.

\subsection{Randflächen und Randkurven}%

Eine \textbf{Randfläche} ist ganz allgemein eine Fläche im Raum die Zwei 
Raumgebiete trennt, z.B. trennt die Randfläche einer Kugel das innere der
Kugel von dem Außenraum. 
Man kann das Vakuum unendlich viele triviale Randflächen
zuordnen. Was interessanter ist, sind Randflächen zwischen z.B. 
unterschiedliche Medien wie Vakuum, metallische Leiter, (un)geladene 
Dielektrika oder Ränder von Ladungsverteilungen. 

Eine \textbf{Randkurve} ist eine Kurve die eine Oberfläche abschließt. 

\subsection{Bedeutung der Sätze von Gauß und Stokes für Elektrostatische 
Felder}%
Zunächst wiederholen wir wie die Sätze mathematisch ganz 
allgemein aussehen.\\

\noindent
Sei $V\subset \mathbb{R}^n$ eine Kompakte Menge mit glattem Rand 
$\partial V$ mit ein nach äußeren orientierten Normaleneinheitsvektor 
$\vb*n$ bzw. Flächen Element $d\vb A =  \vb*n dA$. Sei ferner das Vektorfeld
$\vb F$ stetig differenzierbar auf einer offenen Menge $U$ 
mit $V\subseteq U$ so gilt
\begin{equation}
  \int_V \div \vb FdV  = \oint_{\partial V} \vb F \cdot d\vb A 
  \qquad \textrm{\textbf{Satz von Gauß}}
\end{equation}
Der Satz von Gauß besagt daß die Quellendichte (Divergenz) eines 
Vektorfeldes $\div \vb F$
integriert über ein Volumen $V$ proportional zur Flußintegrals 
des Feldes durch der Randfläche des Volumen $\partial V$.\\

\noindent
Sei $A\subset \mathbb{R}^n$ eine einfach zusammenhängende Fläche mit
glattem Randkurve $\partial A$ (die gegen dem Urzeigersinn durchlaufen 
wird beim Integrieren). Sei ferner das Vektorfeld
$\vb F$ stetig differenzierbar auf einer offenen Menge $U$ 
mit $A\subseteq U$ so gilt
\begin{equation}
  \int_A (\curl \vb F) \cdot d\vb A= \oint_{\partial A} \vb F \cdot d\vb*l 
  \qquad \textrm{\textbf{Satz von Stokes}}
\end{equation}
Der Satz von Stokes besagt daß das Flußintegral über die Wirbeldichte 
(Rotation) eines
Vektorfeldes $\curl \vb F$ über eine Fläche A proportional zur 
Kurvenintegral entlang die Randkurve der Fläche $\partial A$

In der Regel verhalten physikalische Felder sich immer sehr schön, sodass
wir uns über die Stetige Differentationsbedingung erstmal keine gedanken
machen müssen. Die Sätze von Gauß und Stokes schränken unsere zu betrachten
Volumina und Flächen zwar ein, aber nur sehr exotische Objekte erfüllen
die Bedingungen nicht, sodaß wir uns in der Regel auch keine Gedanken
machen müßen ob die mathematische Bedingungen erfüllt sind.

Zusammen mit die Feldgleichungen der Elektrostatik und die obere Sätze
Folgen direkt die Folgende Aussagen
\begin{equation}
  \begin{split}
    \oint_{\partial V} \vb E \cdot d\vb A 
    &
    =\int_V \underbrace{(\div \vb E)}_{4\pi k \rho(\vb*r)} dV
    =4\pi k Q_{V,\text{ges}}\\
    \oint_{\partial A} \vb E \cdot d\vb*l 
    &
    =\int_A \underbrace{(\curl \vb E)}_{0} \cdot d \vb A
    = 0
  \end{split}
\end{equation}

\subsection{Anwendung auf Randflächen}%
\label{ssub:anwendungen-auf-randflaechen}
Man kann nun eine allgemeine Randfläche im Raum betrachten. Legt man ein
kleines Kästchen (Volumen) auf der Rand, zentriert um dem Rand --- also ein
teil des Kästchens liegt an eine Seite der Randfläche, und ein Teil an 
der andere Seite --- und läßt man dieses Kästchen nun immer kleiner werden,
so betrachtet man annäherend zu die Randfläche selbst. Im limes von $V\to0$
findet man sogar lokale Punkte auf der Oberfläche und gilt
\begin{equation*}
  \lim_{V \to 0} \oint_{\partial V} \vb E \cdot 
  d\vb A 
  = \vb E \cdot \vb*n= 4\pi k \sigma(\vb*r)
\end{equation*}
Wobei $\sigma(\vb*r)$ die lokale Flächenladungsdichte auf der Randfläche 
ist.

Analog kann man eine infinitisimale Flächenstückchen 
--- die orthogonal auf der 
Randfläche steht, mit ein Teil an einer Seite der Randfläche und ein Teil
auf der andere Seite --- um die Randfläche zentrieren. Im limes von 
$A\to0$ findet man
\begin{equation*}
    \lim_{A \to 0} \oint_{\partial A} \vb E \cdot d\vb*l 
    = \vb E \times \vb*n
    = 0
\end{equation*}

Es folgen die \textbf{Randflächenbedingungen}
\begin{equation}
  \begin{split}
    \vb E \cdot \vb*n &= 4\pi k \sigma(\vb*r) \\
    \vb E \times \vb*n &= 0
  \end{split}
\end{equation}
Dies bedeutet weiter, daß die orthogonalkomponente 
$\vb E_{\perp}= \vb E\cdot \vb*n$ (in der Anwesendheit einer 
Flächenladungsdichte) einen Sprung um $4\pi k \sigma(\vb*r)$ macht, während
die tangentialkomponente $\vb E_{\parallel}=\vb E \times \vb*n$ immer
stetig ist.


\newpage
\section{Lösung der Poisson Gleichung und Greensche Funktion}%
\label{sub:poisson-green}
In diesem Kapitel besprechen wir wie man die Poisson gleichung lösen kann.

\subsection{die Poissongleichung}%
\label{ssub:poissongleichung}

Wie wir schon in Kapitel~\ref{ssub:Die-Feldgleichungen} hergeleitet haben
gilt in der Elektrostatik die \textbf{Poissongleichung}
\begin{equation*}
  \Delta \varphi (\vb*r) = -4\pi k \rho(\vb*r)
\end{equation*}
Die aufgabe besteht nun darein, $\varphi$ zu finden. Die Poissongleichung 
stellt eine partielle Differentialgleichung 2. Ordnung dar, die gelößt
werden kann unter vorgabe von Randbedingungen (analog zu Anfangsbedingungen
in der Mechanik). 

Praktische Lösungsmethoden sind meistens abhänging von die Geometrie
des zu lösenden Problems. Zunächst besprechen wir aber die Allgemeinste 
Lösung (jedoch meistens nicht schnellste oder einfachste Lösungsweg) der
Poissongleichung, die \textbf{Greensche Funktion}.

\subsection{Die Greensche Identitäte und die Greensche Funktion}%
\label{sub:green}

Für das Poissonproblem gibt es 3 wichtige Identitäten, die 3 Green'sche 
woraus man die allgemeine Lösung der Poissongleichung finden kann 
mittels die Greensche funktion. Zunächst betrachten wir die allgemeine
mathematische Identitäten und wenden sie danach Spezifisch für das Poisson
Problem an.

\subsubsection{1. Green'sche Identität}%
\label{ssub:green-id-1}
Gäbe es ein das Vektorfeld $\vb F = \psi \grad \varphi \equiv \psi \vb \Gamma$ 
mit $\psi$ und $\varphi$ beliebige Skalarfelder mit $\psi\in C^1$ und
$\varphi\in C^2$ (d.h. 1 bzw.\ 2 mal stetig diffbar) so folgt aus dem 
Satz von Gauß
\begin{equation}
  \int_V \underbrace{\qty(\psi\Delta\varphi + 
  \grad\psi\grad\varphi)}_{\div\vb F}dV =
  \oint_{\partial V} \qty(\psi\grad\varphi) \cdot d\vb A
\end{equation}
wobei man beachtet daß
\begin{equation*}
  \div \vb F =\div (\psi \vb \Gamma) = 
  \grad \psi \vb \Gamma + \psi\div\vb \Gamma = \grad \psi \grad \varphi + 
  \psi \Delta \varphi
\end{equation*}
Die erste Greensche Identität ist also einen Spezialfall des Gauß Gesetzes

\subsubsection{2. Green'sche Identität}%
\label{ssub:green-id-2}
Sei nun auch noch $\psi\in C^2$, und gäbe es ein weiteres $\epsilon\in C^1$
sodass man das Vektorfeld 
$\vb F=\psi(\epsilon\grad\psi) - \varphi(\epsilon\grad\psi)$ 
definiert folgt wieder
unter Anwendung der Satz von Gauß.

\begin{equation}
  \int_V \underbrace{\psi\div(\epsilon\grad\varphi) - \varphi\div(\epsilon\grad\psi)}_{\div \vb F} dV
  = \oint_{\partial V} \epsilon (\psi \grad \varphi - \varphi \grad \psi) 
  \cdot d\vb A
\end{equation}

wobei man beachtet daß
\begin{equation*}
  \div \vb F = \div (\psi (\epsilon\grad\varphi) - \varphi(\epsilon\grad\psi))
  = \cancel{(\grad\psi)(\epsilon\grad\varphi)} 
  + \psi\div(\epsilon\grad\varphi)
  - \cancel{(\grad\varphi)(\epsilon\grad\psi)}
  - \varphi\div(\epsilon\grad\psi)
\end{equation*}

Wir sind erstmal interessiert an dem Spezialfall wo $\epsilon=1$ ist, sodass
folgt
\begin{equation}
  \int_V \psi\Delta\varphi - \varphi\Delta\psi dV= 
  \oint_{\partial V} \psi\grad\varphi - \varphi\grad\psi d\vb A
\end{equation}

\subsubsection{3. Green'sche Identität}%
\label{ssub:green-id-3}

Man definiere nun die \textbf{Greensche Funktion} sodass gilt
\begin{equation}
  \Delta \green= \delta(\vb*r - \vb*r')
\end{equation}
Für den Laplace Operator $\Delta$ erfüllt die Funktion 
$\green=-\frac{1}{4\pi}\frac{1}{\abs{\vb*r-\vb*r'}}$ 
diese Bedingung. Die 
Green'sche Funktion ist aber nicht eindeutich definiert, denn die Funktion 
$\green
=-\frac{1}{4\pi}\frac{1}{\abs{\vb*r-\vb*r'}} + F(\vb*r,\vb*r')$ erfüllt die
Bedingung auch, falls $\Delta F(\vb*r, \vb*r')=0$

Setzt man nun $\psi=\green$ und $\varphi=\phi(\vr')$ in die die 2.
Greensche Identität ein, so folgt die 3. Greensche Identität (hier schon
als Spezialfal für das elektrische Potenzial)
\begin{equation}
  \begin{split}
    \int_V G(\vb*r,\vb*r')\underbrace{\Delta\phi(\vb*r')}_{%
    -4\pi k\rho(\vb*r')} 
    - \phi(\vb*r')\underbrace{\Delta \green}_{\delta(\vb*r - \vb*r')}
    dV'
    &= -4\pi k\int_V \rho(\vb*r') \green dV' -\phi(\vb*r)\\
    &= \oint_{\partial V} (\green\grad\phi(\vr')-\phi(\vr')\grad\green)\ddA'\\
  \end{split}
\end{equation}
Es folgt die Allgemeine Lösung für das Potenzial
\begin{equation*}
  \Leftrightarrow \phi(\vr) = -4\pi k\int_V \rho(\vr')\green dV' 
  - \oint_{\partial V} (\green\grad\phi(\vr')-\phi(\vr')\grad\green)\ddA
\end{equation*}

Bemerke, dass falls $\phi$ und $\green$ im Unendlichen nach null abfallen,
mit $\green=-\frac{1}{4\pi} \frr $ einfach Funktion~\ref{eq:potential} folgt.
Dies ist also ein Spezialfall der allgemeine Lösung von $\phi(\vr)$, wo
keine Randbedingungen vorgegeben sind, bzw.\ wo die Randbedingungen im
Unendlichen liegen und deswegen verschwinden.

Ist nun auf die Randfläche $\phi$ vorgegeben (Dirichlet Randbedingungen), 
so kann man $G$ so wählen, 
sodass $\green[D]\equiv\green\big|_{\vr'\in\partial V}= 0$ gilt, und vereinfacht sich 
die 3. Greensche Identität sich zu
\begin{equation}
  \phi(\vb*r') = -4\pi k\int_V \rho(\vr')\green[D] dV' 
  + \oint_{\partial V} \phi(\vr')\grad\green[D] \ddA
\end{equation}
vereinfacht


\newpage
\section{Spiegelladungsmethode}%

\newpage
\section{Elektrostatik in Materie}%

\newpage
\section{Energie von Elektrostatische Felder}%

  
  \newpage
  \chapter{Magnetostatik}\label{cha:Magnetostatic}
    \minitoc%
    \clearpage
    In der Magnetostatik betrachtet man nur \textbf{Zeitunabhängige} Stromdichten (daher Magneto\textit{statik}). Dies führt dazu, sowie in der
Elektrostatik, das man Stationäre Felder betrachtet, und das Zeitunabhängige
Effekte vernachlässigt werden können. Zwischen der Elektrostatik und
der Magnetostatik gibt es viele Ähnlichkeiten. Deswegen werden wir nur
kurz alle wichtige Themen besprechen ohne Herleitungen zu wiederholen.

\section{Stromdichten und Kontinuitätsgleichung}%
\label{sec:stromdichten}
In der Magnetostatik dürfen sich Ladungen endlich bewegen. Dies führt zu
\textbf{Ladungsströme}, die mathematisch durch die \textbf{Stromdichte}
\(\vb*j(\vr, t)\) beschrieben werden. In der Magnetostatik betrachtet man
zunächst Zeitunabhängige Stromdichten $\vb*j(\vr, t)=\vb*j(\vr)$. Weiter
ist auch noch immer die Ladungsverteilung im Raum zeitunabhängig. Weil die
Ladung eine physikalische Erhaltungsgröße ist (\textbf{Ladungserhaltung}),
und Ladung sowohl die Ursprung der Ladungsdichte als die Stromdichte ist,
sind die Größen $\rho$ und $\vb*j$ natürlich verknüpft. Die Ladungserhaltung
wird Mathemathisch beschrieben durch die \textbf{Kontinuitätsgleichung}
\begin{equation}
  -\div\vb*j(\vr, t) = \partial_t \rho(\vr, t)
\end{equation}
Man kann dies So auffassen, dass die Fluß von Ladung in durch die Randfläche
einer Raumbereich rein (\(-\div\vb*j(\vr, t)\)) gleich die Zeitliche Änderung
(\(\partial_t\rho(\vr, t)\)) der Ladung im Raumbereich ist. Bemerke, dass das Minus da ist, weil für eine positive Ladungszuname die man für einen strom
von positiven Ladungen, $\div\vb*j(\vr, t)$ eine Senke
darstellen soll, und nicht eine Quelle!

Weil die Ladungsdichte zeitunabhängig sein muss in der Magnetostatik, folgt
direkt dass $\div\vb*j(\vr)=0$. Dies bedeutet also, dass die Stromdichte
in einem Raumbereich rein also immer gleich sein muss als die Stromdichte aus
dem Raumbereich raus, sowie zum Beispiel in einem Leiterdraht oder Leiterschleife.

Die Ladungsdichte die durch eine Punktladung erzeugt wird ist einfach die
Ladungsdichte der Punktladung (Gleichung~\ref{eq:}) mal seine Momentane geschwindigkeit.
\begin{equation*}
  \vb*j(\vr) = q \dot{\vr}(t) \delta(\vr -\vr(t))
\end{equation*}
Für mehrere Ladungsverteilungen folgt natürlich
\begin{equation}
  \vb*j(\vr) = \sum_i q_i \dot{\vr}_i(t) \delta(\vr-\vr_i(t))
\end{equation}
Man kann dies dann wieder Räumlich mitteln um eine kontinuierliche Ladungsverteilung zu erhalten.
\begin{equation}
  \vb*j_{\text{kont}}(\vr) = \frac{1}{V(\vr)}\sum_{\vb*j_i\in V(\vr)}\vb*j_i 
\end{equation}
Der gesammte Strom durch eine Oberfläche wird auch als $I$ angedeutet. Es
gilt
\begin{equation}
  I_A = \int_A \vb*j(\vr) \ddA
\end{equation}


\newpage
\section{Magnetfeld und Biot-Savart-Gesetz}%
\label{sec:magnetfeld}
\subsection{Ampèresche Kraft Gesetz}%
\label{sub:amperesche-gesetz}
Analog zu wie die Ladungsdichte die Quelle für das Elektrostatische-Feld, ist
die die Stromdichte die Quelle für das Magnetfeld. Im frühen 19. Jahrhundert wurde beobachtet dass Zwei stromdurchflossenen Leiterschleifen Kräfte auf einander ausübten. Die Kraft wirkt Anziehend falls die Ströme gleich ausgerichtet sind und vise versa. Diese magnetische Kraft wurde in 1823 von André Marie Ampère zuerst mathematisch beschrieben, und ist gegeben durch das \textbf{Ampèresche Kraft Gesetz} (nicht zu verwechseln
mit dem Ampèresche Durchfluttungsgesetz).

Gäbe es zwei Leiterschleifen die durch die einfache, geschlossene und disjunkte Raumkurven $C_1$ bzw. $C_2$ beschrieben werden, die durch die ströme $I_1$ bzw. $I_2$ durchflossen werden.
Sind $\vr_1$ und $\vr_2$ weiter zwei unabhängige Ortsvektoren die vom Ursprung
aus auf Punkte der Kurven $C_1$ bzw. $C_2$ weisen, so gilt für die Kraft die
Leiterschleife 1 auf Leiterschleife 2 ausübt. 
\begin{equation} 
  \vb F_{1\to2}=k'I_1I_2\oint_{C1}\oint_{C2} \frac{d\vr_1\times(d\vr_2
  \times \vr_{12})}{r_{12}^3}=k'I_1I_2\oint_{C1}\oint_{C2} d\vr_1 \cdot d\vr_2 \frac{\vr_{12}}{r_{12}^3} \qquad \vr_{12}=\vr_1-\vr_2
\end{equation}
$k'$ Ist wieder eine
Konstante die vom gewählten Einheitensystem abhängt. In SI gilt $k'=\frac{\mu_0}{4\pi}$. In Gausseinheiten gilt $k'=\frac{1}{c}$ mit $c$ die
Lichtgeschwindigkeit. Wie wir in der Elektrodynamik sehen werden ist auch $\mu_0$ zusammen mit $\epsilon_0$ durch mit $c$ verknüpft ($c^2=\frac{1}{\mu_0\epsilon_0}$ in SI). Eine relativistische beschreibung der Elektrodynamik weißt eine Äquivalenz zwischen dem Coulomb Gesetz und
das Ampèresche Gesetz auf. Ob eine Elektrische oder eine Magnetische Kraft
wirkt, wird dann abhängig vom gewählten Bezugssystem sein.

\subsection{Das Magnetfeld}%
\label{sub:magnetfeld}
Es ist nun wieder sinnvol, eine Beschreibung zu finden, analog zur Definition des Elektrischen Feldes, wobei die magnetische Kraft durch beschrieben wird durch ein äußeres (Magnet-)Feld das lokal auf eine Stromverteilung wirkt. Man definiere das Magnetfeld aus ersten Prinzipien
aus die Ampèresche Kraft sodass
\begin{equation}
  \vb F_{12} = I_1\oint_{C_1} d\vr_1\times \vb B_2(\vr_1)
  \quad\text{mit}\quad
  \vb B_1(\vr_1) = k'I_2 \oint_{C_2} d\vr_2 \times \frac{\vr_{12}}{r^3_{12}} 
\end{equation}
Das Magnetfeld läßt sich dann zu beliebige Stromverteilungen veralgemeinern, und wird das \textbf{Biot-Savart-Gesetz} genannt
\begin{equation}
  \vb B(\vr) = k'\int d^3 r' \vb*j(\vr')\times\frac{(\vr-\vr')}{\rr^3}
\end{equation}
Dies entspricht, sowie in der Elektrostatik, wieder eine lösung für
verschwindende Randbedingungen.
Die gesammte magnetische Kraft auf eine beliebige Ladungsverteilung im außeren Magnetfeld wird zu
\begin{equation}
  \vb F = \int d^3r \vb*j(\vr) \times \vb B(\vr)
\end{equation}
Das Magnetfeld ist nun also ein Maß, für die Kraft die ein infinitisimales Leiterstück bzw.\ eine bewegte Ladung lokal spüren Würde, durch die
Anwesendheit eines Äußeren Stroms.

\subsection{Das Vektorpotential}%
\label{sub:Vektorpotential}
Es ist nun offensichtlich dass das Magnetfeld ein reines Rotationsfeld ist,
indem man sieht dass
\begin{equation}
  \vb \nabla_r \times \frac{\vb*j(\vr')}{\rr} 
  = \vb*j(\vr')\times\frac{(\vr-\vr')}{\rr^3} 
\end{equation}
Aus dem Biot-Savart-Gesetz folgt also nun
\begin{equation}
    \label{eq:vektorpotential}
    \vb B(\vr) = \curl \vb A(\vr)\quad\text{mit}
    \quad\vb A(\vr) = k'\int d^3r' \frac{\vb*j(\vr')}{\rr}
\end{equation} 

\subsection{Feldgleichungen der Magnetostatik}%
\label{sub:feldgleichungen-magnetostatik}
Sowie schon im letzten Abschnitt besprochen ist das Magnetfeld ein reines
Rotationsfeld. Daraus folgt die homogene Feldgleichung der Magnetostatik.
Aus dem Zerlegungssatz folgt zusammen mit die homogene Feldgleichung auch
den Zusammenhang zwischen die Rotation des Magnetfeldes $\vb B(\vr)$ und die Stromdichte $\vb*j(\vr)$
\begin{equation}
  \begin{aligned}
    \div \vb B(\vr) &= 0 & \text{(homogen)}\\
    \curl \vb B(\vr) &= 4\pi k'\vb*j(\vr) & \text{(inhomogen)}\\
  \end{aligned}
\end{equation}

\subsection{Eichtransformation und Inhomogene Poisson-Gleichung}%
\label{sub:eichtransformation}
Das Vektorpotential ist nicht eindeutig Festgelegt, denn
\begin{equation}
  \vb B(\vr) = \curl \vb A'(\vr)= \curl (\vb A(\vr) + \vb F(\vr)) 
  \quad\text{falls}\quad\curl \vb F(\vr)=0
  \Leftrightarrow \vb F(\vr) \equiv \grad \chi(\vr)
\end{equation}
Für ein beliebiges Skalarfeld $\chi$.

Das heißt das für die Magnetostatik die \textbf{Eichtransformation}
\begin{equation}
  \vb A'(\vr) = \vb A(\vr) + \grad \chi(\vr)
\end{equation}
das Magnetfeld invariant läßt. In der Magnetostatik ist die \textbf{Coulomb-Eichung} üblich, wobei $\div \vb A=0$ gewählt wird, sodaß $\Delta \chi(\vr)=0$. In der Coulomb-Eichung folgt somit zusammen mit die magnetostatische Maxwellgleichungen die inhomogene Poisson-Gleichung
\begin{equation}
  \Delta \vb A(\vr) = -4\pi k' \vb*j(\vr)
\end{equation}

\subsection{Die Hauptaufgabe der Magnetostatik} 
Die Hauptaufgabe der Magnetostatik ist also das Berechnen von 
magnetostatische Potentiale und Felder unter vorgabe von 
Stromverteilungen und Randbedingungen. 
Sowie bei der Elektrostatik sind für Hochsymmetrische Probleme analytische 
Lösungen möglich, für schwierigere Probleme können wieder meistens nur numerische 
Lösungen gefunden werden. Für uns ist erstmal wieder das Finden von analytische
Lösungen interessant um ein Grundverständnis aufzubauen. Dazu werden weiter auch wieder Methoden besprochen die uns analytische Näherungen geben können, wie die Multipolentwicklung.


\newpage
\section{Multipolentwicklung: Magnetischer Dipol}%
Zunächst besprechen wir wie man analytische Näherungen finden kann
unter angabe von komplexere Ladungsverteilungen. 
Das Problem ist meistens das Integrieren der
\begin{equation*}
  \frac{1}{\abs{\vb*r-\vb*r'}}
\end{equation*}
Term (in Kombination mit die Ladungsdichte $\rho(\vb*r')$). 
Deswegen möchten wir in der \textbf{karthesische Multipolentwicklung} 
diesen  Term Taylor-entwickln, damit wir es in eine 
Reihe von Polinomiale Terme umwandeln können, 
weil diese Einfach(er) zu integrieren ist. 
Daneben gibt es noch die \textbf{Kugelfächen-Entwicklung} die sich 
insbesondere für Radial- bzw. Rotationssymmetrische Probleme eignet, 
die wir Später besprechen werden. 

\subsection{Karthesische Multipolentwicklung}%
\label{ssub:Karthesische-Multipolentwicklung}
Die karthesische Multipolentwicklung wird mittels eine Taylor-Entwicklung
hergeleitet. Man muss dabei eine Multidimensionale Taylor-Entwicklung 
durchführen.

Eine allgemeine mehrdimensionale Taylorentwicklung wird gegeben durch:
\begin{equation*}
  f(\vb*\alpha, \vb*\beta)
  =\qty(\exp(\vb*\beta\cdot\nabla_{\vb*\beta'})
  f(\vb*\alpha, \vb*\beta'))
  \bigg|_{\vb*\beta'=\vb*\beta_0}
  =\qty(\sum_{n=0}^\infty
  \frac{(\vb*\beta\cdot\nabla_{\vb*\beta'})^n}{n!}
  f(\vb*\alpha, \vb*\beta'))
  \bigg|_{\vb*\beta'=\vb*\beta_0}
\end{equation*}
Man bemerke dass $\nabla$ einen Operator ist! Die exponential Funktion
dient hier nur zur vereinfachung der Darstellung! Man Taylore nun um $\vb*r'=0$ (dies heißt, daß $\vb*r\gg\vb*r'$ sodass 
$\vb*r-\vb*r'\approx\vb*r$, dafür muss $\vb*r$ natürlich weit
von der Quelle entfernt sein). Die Multipolentwicklung ist also ein 
\textbf{Fernfeldnäherung}.

Man definiere nun $f(\vb*r,\vb*r')=f(r_1,r_2,r_3,r_1',r_2',r_3')
\equiv\frac{1}{\abs{\vb*r-\vb*r'}}$ sodass bis zur 2. Ordnung die Taylorentwicklung für unsere Funktion
wie folgt aussieht:

\begin{equation*}
  \frac{1}{\abs{\vb*r-\vb*r'}}=f(\vb*r, 0)
  + (\vb*r'\cdot \nabla_{\vb*{\bar{r}'}}) f(\vb*r, \vb*{\bar{r}}')\bigg|_{\vb*{\bar{r}}'=0}
  + \frac{1}{2}(\vb*r'\cdot\nabla_{\vb*{\bar r'}})^2
  f(\vb*r,\vb*{\bar r}')
  \bigg|_{\vb*{\bar r'}=0}
  + \ldots
\end{equation*}

\underline{0. Ordnung:}
\begin{equation*}
  \frac{1}{\abs{\vb*r-\vb*{\bar r'}}}\bigg|_{\vb*{\bar r'}=0} 
  = \frac{1}{\abs{\vb*r}}= \frac{1}{r} 
\end{equation*}

\underline{1. Ordnung:}
\begin{equation*}
  \vb*{r'}\cdot
  \nabla_{\vb*{\bar{r}'}}\frac{1}{\abs{\vb*r-\vb*{\bar{r}'}}}
  \bigg|_{\vb*{\bar r'}=0}
  = \frac{\vb*{ r'}\cdot\vb*r}{r^3} 
\end{equation*}

\underline{2. Ordnung:}
\begin{equation*}
  \begin{split}
  \frac{1}{2}(\vb*{\bar r'}\cdot\nabla_{\vb*{\bar{r}'}})^2
  \frac{1}{\abs{\vb*r-\vb*{\bar r'}}}\bigg|_{\vb*{\bar r'}=0}
  &=
  \frac{1}{2}\qty[r_i'\pdv {\bar r_i'}]\qty[r_j'\pdv{\bar r_j'}]
  \frac{1}{\abs{\vb*r-\vb*{\bar r'}}}
  \bigg|_{\vb*{\bar r'}=0}\\
  &=
  \frac{r'_ir'_j}{2} 
  \frac{\partial^2}{\partial_{\bar r_i'}\partial_{\bar r_j'}} 
  \frac{1}{\abs{\vb*r-\vb*{\bar r'}}}\bigg|_{\vb*{\bar r'}=0}\\
  &=
  \frac{r'_ir'_j}{2} \frac{\partial}{\partial_{\bar r_i'}} 
  \frac{x_j}{\abs{\vb*r-\vb*{\bar r'}}^3}\bigg|_{\vb*{\bar r'}=0}
  \qquad x_j\equiv (r_j-\bar r_j')\\
  &=
  \frac{r'_ir'_j}{2} \frac{\partial}{\partial_{\bar r_i'}} 
  \frac{x_j}{\abs{\vb*r-\vb*{\bar r'}}^3}\bigg|_{\vb*{\bar r'}=0}\\
  &=
  \frac{r'_ir'_j}{2} 
  \qty(
    \text{-}\frac{3}{2}
    \frac{\text{-}2x_i x_j}{\abs{\vb*r-\vb*{\bar r'}}^5}
    -\frac{\delta_{ij}}{\abs{\vb*r-\vb*{\bar r'}}^3} 
  )\bigg|_{\vb*{\bar r'}=0}\quad\text{(Produktregel)}\\
  &=
  \frac{1}{2}\sum_{i,j=1}^3 \frac{3r_ir_j-\delta_{ij}\vb*r^2}{r^5}
  r_i'r_j'
  \end{split}
\end{equation*}
\textit{Eine wichtige Bemerkung:}
Aus Symmetrie Grunden gilt
\begin{equation*}
  \frac{1}{2}\sum_{i,j=1}^3 \frac{3r_ir_j-\delta_{ij}\vb*r^2}{r^5}
  r_i'r_j'
  =
  \frac{1}{2}\sum_{i,j=1}^3 \frac{3r'_ir'_j-\delta_{ij}\vb*r'^2}{r^5}
  r_ir_j
\end{equation*}
Wir benutzen im allgemeinen die letzte Definition wenn wir die
elektrische und magnetische Multipole berechnen. Man findet also bis zum
2. Ordnung
\begin{equation}
  \frac{1}{\abs{\vb*r-\vb*r'}}
  \approx 
  \frac{1}{r} 
  + \frac{\vb*r'\cdot\vb*r}{r^3} 
  + \frac{1}{2}\sum_{i,j=1}^3 
    \frac{3r'_ir'_j-\delta_{ij}\vb*r'^2}{r^5}r_ir_j
\end{equation}
Setzt man dies in die Definition für das Elektrische Potential ein, so 
findet man
\begin{equation*}
  \begin{split}
    \phi(\vb*r) 
    &= k \int d^3 r' \frac{\rho(\vb*r')}{\abs{\vb*r - \vb*r'}} \\
    & \approx k\int d^3 r' \rho(\vb*r')
    \qty(
    \frac{1}{r} 
    + \frac{\vb*r'\cdot\vb*r}{r^3} 
    + \frac{1}{2}\sum_{i,j=1}^3 
    \frac{3r'_ir'_j-\delta_{ij}\vb*r'^2}{r^5}r_ir_j)\\
    &\equiv 
    k \frac{Q}{r} + k \frac{\vb*p\cdot\vb*r}{r^3} 
    + \frac{k}{2} \sum_{i,j=1}^{3} \frac{r_ir_j}{r^5}Q_{ij}
  \end{split}
\end{equation*}

\begin{center}
\begin{tabular}{ll}
  Monopol:    & $\ds Q=\int d^3r\rho(\vb* r)$ \quad\text(Gesammtladung)\\
  Dipol:      & $\ds p_i=\int d^3r\rho(\vb* r)r_i
                \quad\vb* p=p_i\vu{e}_i $\\
  Quadrupol:  & $\ds Q_{ij}=\int d^3r\rho(\vb* r)
  \qty(3r_ir_j - \delta_{ij}\abs{\vb* r}^2)$
\end{tabular}
\end{center}
Dabei hat die Quadrupoltensor $Q_{ij}$ nur 5 Freiheitsgraden. 
Mit nur 5 Rechnungen alle (9) Quadrupol Elemente berechnen. 
Es gilt zwar $Q_{ij}=Q_{ji}$ (symmertrisch) und $\text{sp}(\bm Q)=\sum_i Q_{ii}=0$ (spurfrei).

Im allgemeinen berechnet 
man keine weitere Ordnungen analytisch im Bachelorstudium, 
und zum Verständniss bringt dies auch nicht mehr (außer ärger), 
sodass höhere Ordnungen berechnen 
ein Problem ist das man lieber an Computer überlässt.

\subsection{Sphärische Multipolentwicklung}%
\label{ssub:sphaerische-Multipolentwicklung}
Für Ladungsverteilungen die Radial- oder Rotationssymetrisch sind ist die 
Sphärische Multipolentwicklung besonders geeignet, vor allem falls man die 
Ladungsdichte als Linearkombination von Kugelfächen-Funktionen schreiben 
kann. Um die Sphärische Multipolentwicklung zu motivieren machen wir 
zunächst die Annahme (mit dem Seperationsansatz), dass das 
Winkelanteil des Potentials unabhähngig vom Radialanteil ist, also
\begin{equation*}
  \phi(\vb*r) = \phi_r(r)\phi_\Omega(\theta,\varphi)
\end{equation*}
Weil man (wie wir später besprechen werden) die Poisson-Gleichung mit dem
gleichen Ansatz lösen wird, und die Kugelflächen-Funktionen 
$Y_{lm}(\theta,\varphi)$ eine Eigenfunktion der Laplace Operator ist, 
ist es eine gute Idee um dies als Basis für eine Sphärische Entwicklung 
zu benutzen. Dazu formen die Kugelflächen-Funktionen unter Integration über
$d\Omega=\sin\theta d\theta d\varphi$ eine orthonormale Basis. Es gilt die untere Zusammenhang.
\begin{equation*}
  \int d\Omega Y_{lm}(\Omega)Y^*_{l'm'}(\Omega)=\delta_{ll'}\delta_{mm'}
\end{equation*}

Die Entwicklung in Kugelkoordinaten ist rechnerisch etwas aufwändig, und
wird hier erstmal übersprungen, aber sie wird in gute Literatur über
die Elektrodynamik oft gegeben, wie z.B. in Kapitel 2.3.8 von Noltings
``Grundkurs Theoretische Physik 3, Elektrodynamik''. 
Man findet 
\begin{equation}
  \frac{1}{\abs{\vb*r-\vb*r'}}
  =
  \sum_{lm} \frac{4\pi}{2l+1} \frac{r_<^l}{r_>^{l+1}}
  Y^*_{lm}(\Omega')Y_{lm}(\Omega)
\end{equation}
\begin{center}
mit $l=0,1,2,\ldots$ und $m=-l,\ldots,l$\\
$r_<=\min(r,r')$, $r_>=\max(r,r')$.
\end{center}
Setzt man dies in der Definition für das Elektrische Potential ein, 
so findet man
\begin{equation}
  \begin{split}
    \phi(\vb*r) 
    &= k\int d^3r' \rho(\vb*r') \frac{1}{\abs{\vb*r-\vb*r'}} \\
    &= k\sum_{lm} \frac{4\pi}{2l+1} 
    \int_0^\infty dr'{r'}^2 \frac{r_<^l}{r_>^{l+1}} 
    \int_\Omega d\Omega' Y^*_{lm}(\Omega')Y_{lm}(\Omega) \rho(\vb*r')\\
    &= k\sum_{lm} \frac{4\pi}{2l+1} Y_{lm}(\Omega) 
    \qty(
    \int_0^{r} dr'{r'}^2 \frac{{r'}^l}{r^{l+1}} +
    \int_{r}^{\infty} dr'{r'}^2 \frac{r^l}{{r'}^{l+1}} 
    )
    \int_\Omega d\Omega' Y^*_{lm}(\Omega')\rho(\vb*r')\\
  \end{split}
\end{equation}
Sei nun die Ladungsdichte 
$\rho(\vb*r')=\sum_{l'm'}f_{f'm'}(r')Y_{l'm'}(\Omega')$
für beliebige $l'$ und $m'$ (also einfach irgendeine Linearkombination von
beliebige Kugelflächenfunktionen) so vereinfacht sich das Problem weiter,
und kann man für eine Endliche Summe sogar exakte Lösungen finden.
(Es ist nicht notwendig dass man die Ladungsdichte als Linearkombination
von Kugelflächenfunktionen schreibt, nur einfacher).
\begin{equation}
  \begin{split}
    \phi(\vb*r) 
    &= k\sum_{lm} \frac{4\pi}{2l+1} Y_{lm}(\Omega) \sum_{l'm'}
    \qty(
    \int_0^{r} dr'\frac{{r'}^(l+2)}{r^{l+1}}f_{l'm'}(r) +
    \int_{r}^{\infty} dr'\frac{r^l}{{r'}^{l-1}}f_{l'm'}(r) 
    )
    \int_\Omega d\Omega' Y^*_{lm}(\Omega')Y_{l'm'}(\Omega')\\
    &= k\sum_{lm} \frac{4\pi}{2l+1} Y_{lm}(\Omega) \sum_{l'm'}
    \qty(
    \int_0^{r} dr'\frac{{r'}^{l+2}}{r^{l+1}}f_{l'm'}(r) +
    \int_{r}^{\infty} dr'\frac{r^l}{{r'}^{l-1}}f_{l'm'}(r) 
    )
    \delta_{ll'}\delta_{mm'}\\
    &= k\sum_{l'm'} \frac{4\pi}{2l+1} Y_{l'm'}(\Omega)
    \qty(
    \int_0^{r} dr'\frac{{r'}^{l+2}}{r^{l+1}}f_{l'm'}(r) +
    \int_{r}^{\infty} dr'\frac{r^l}{{r'}^{l-1}}f_{l'm'}(r) 
    )
  \end{split}
\end{equation}
d.h.\ alle Terme wo $l\neq l'$ oder $m\neq m'$ fallen wegen dem Kronecker-delta weg. Wir werden aber weiter wieder von eine allgemeine Ladungsdichte 
ausgehen.

Ist die Ladungsdichte nun nach Innen oder Nach außen beschränkt, dann kann
man $r_>$ und $r_<$ eindeutig als $r$ oder $r'$ festlegen in bestimmte 
Raumbereiche. Hat man z.B. eine geladene Kugelschale wobei die 
Ladungsdichte, noch von $\theta$ und $\varphi$ abhängen darf, so kann man
das Potential im Inneren mit nur dem $\int_r^\infty dr$ Integral beschreiben, und
dem Außenraum mit durch das $\int_0^r dr$ Integral. Es macht manchmal 
sogar kein Sinn die Lösungen der beide Raumbereiche gleichzeitig zu 
berechnen weil die Lösung für das Innere Raumbereich im äußeren Raumbereich
divergiren kann oder umgekehrt. Wir Teilen hier also zunächst die 
Raumgebiete auf und finden Multipolmomente für dem innen und außen Räume 
$q_<^{lm}$ und $q_>^{lm}$\\

\noindent
Im Innenraum ($r\approx 0$):
\begin{equation}
  \begin{split}
    \phi(\vb*r) 
    &= k\sum_{lm} \frac{4\pi}{2l+1} Y_{lm}(\Omega)
    \int_0^{r} dr'\frac{{r}^{l}}{{r'}^{l-1}}
    \int_\Omega d\Omega'Y^*_{lm}(\Omega')\rho(\vb*r')\\
    &= k\sum_{lm} \frac{4\pi}{2l+1}r^l q_<^{lm} Y_{lm}(\Omega) 
  \end{split}
\end{equation}
\begin{center}
  mit $\ds q_<^{lm}=\int_0^\infty dr' {r'}^{1-l}
  \int_\Omega d\Omega' Y^*_{lm}(\Omega')\rho(\vb*r)$
\end{center}
\noindent
Im Außenraum ($r\gg0$):
\begin{equation}
  \begin{split}
    \phi(\vb*r) 
    &= k\sum_{lm} \frac{4\pi}{2l+1} Y_{lm}(\Omega)
    \int_0^{r} dr'\frac{{r'}^{l+2}}{r^{l+1}}
    \int_\Omega d\Omega'Y^*_{lm}(\Omega')\rho(\vb*r')\\
    &= k\sum_{lm} \frac{4\pi}{2l+1} 
    \frac{q_>^{lm}}{r^{l+1}} Y_{lm}(\Omega) 
  \end{split}
\end{equation}
\begin{center}
  mit $\ds q_>^{lm}=\int_0^\infty dr' {r'}^{l+2}
  \int_\Omega d\Omega' Y^*_{lm}(\Omega')\rho(\vb*r)$
\end{center}


\newpage
\section{Magnetostatik in Materie}%

\subsection{Magnetische Felstärke und Magnetische Polarisation}%
\label{sub:H-und-M}

Analog wie bei der Elektrostatik teilen wir nun in Medien den Gesammtstrom auf makroskopische ebene auf in eine freie und eine gebundene Stromdichte
\begin{equation}
  \vb*j(\vr) = \vb*j_f(\vr) + \vb*j_b(\vr)
\end{equation}
Dabei sind die Gebundene Ströme im Medium zum Beispiel die Elektronen die um
das Atom kreisen, sowie induzierte Ströme durch die Anwesendheit eines äußeren magnet Feldes, die jedoch auf kleine Raumgebiete beschränkt bleiben.

Nun führen die freie Ströme wieder zu einem Feld, die magnetische Feldstärke $\vb H(\vr)$, dass das magnetische Feld sehr ähnelt. Die gebundene Ströme führen zu eine Dipoldichte oder Magnetisierung $\vb M(\vr)$. Dabei läßt die Dipoldichte sich wieder verstehen als Volumenmittelung von Punktdipolen $\vb*m_i$ und gilt
\begin{equation}
  \vb M(\vr) = \frac{1}{\Delta V(\vr)} \sum_{\vb*m_i\in\Delta V(\vr)}\vb*m_i
\end{equation}
Nun wird die Magnetisierung wieder aufgeteilt in eine Spontane Magnetisierung $\vb M_{\text{sp}}$ und eine induzierte
Magnetisierung $\vb M_{\text{ind}}$ mit
\begin{equation}
  \vb M(\vr) = \vb M_\text{sp}(\vr) + \vb M_\text{ind}(\vr)
\end{equation}
Die Spontane Magnetisierung ist verantwortlich für Permanentmagneten, und ist viel bekannter als die spontane elektrische Polarisation. Obwohl in der klassiche Magnetostatik angenommen wird, dass die Dipoldichte durch Ströme verursacht wird, ist es in der Realität so, daß die Kreisströme von Elektronen
nicht ausreichend sind um Magnetismus zu beschreiben. Dafür ist tatsächlich die Spin des Elektrons zum größten Teil verantwortlich, ist also eine Quantenmechenische Beschreibung erforderlich, aber dies ist nicht Thema dieser Einführungskurs.

Ziemlich analog wie bei der Elektrostatik wird $\vb M_\text{ind}$ von einer äußeren magnetische Feldstärke induziert und man definiere
\begin{equation}
  \vb M_\text{ind}(\vr) = \frac{1}{4\pi k'}\vu*\chi_{M}(\vr, \vb H(\vr)) \cdot \vb H(\vr)
\end{equation}
Für LHI-Medien folgt nun wieder
\begin{equation}
  \label{eq:Mind}
  \vb M_\text{ind}(\vr) = \frac{\chi_M}{4\pi k'}\vb H(\vr) 
\end{equation}
Weiter gilt
\begin{equation}
  \curl \vb M(\vr) = \vb*j_b(\vr)
  \quad\text{und}\quad
  \curl \vb H(\vr) = \vb*j_f(\vr)
\end{equation}
sodass folgt
\begin{equation}
  \begin{split}
    \curl\vb B(\vr) 
    &= 4\pi k'\vb*j(\vr)\\ 
    &= 4\pi k'(\vb*j_f(\vr) +  \vb*j_b(\vr))\\
    &= 4\pi k'(\curl \vb H(\vr) + \curl \vb M(\vr))\\
    \Leftrightarrow \vb B(\vr) &= 4\pi k' (\vb H(\vr) + \vb M(\vr)) 
  \end{split}
\end{equation}
Setzt man $\vb M_\text{sp}=0$, und mit Gleichung~\ref{eq:Mind}
folgt weiter
\begin{equation}
  \begin{split}
    \vb B(\vr) &=  \frac{(1+\chi_M)}{4\pi k'} \vb H(\vr)\\
               &\stackrel{\text{SI}}{=} \mu_r\mu_0 \vb H(\vr) \qquad \mu_r=(1+\chi_M)
  \end{split}
\end{equation}



\newpage
\section{Randwertprobleme in der Magnetostatik}%
Im Vakuum gelten es die Feldgleichungen
\begin{equation}
  \begin{split}
    \div \vb B(\vr) &= 0\\
    \curl \vb B(\vr) &= 4\pi k' \vb*j(\vr)
  \end{split}
\end{equation}
Daraus folgen die Randbedingungen, analog wie beim elektrischen Feld
\begin{equation}
  \begin{split}
    \vb*n\cdot \vb B(\vr) &= 0\\
    \vb*n\times \vb B(\vr) &= 4\pi k'\vb*j^F(\vr)\\
  \end{split}
\end{equation}
Und ist die orthogonalkomponente des magnetischen Feldes immer stetig an einer Randfläche, während die Tangentialkomponente einen Sprung um $4\pi k\vb*j^F(\vr)$ machen kann in der Anwesendheit von einer nichtverschwindenden Flächenstromdichte.

In medien Gelten die Feldgleichungen
\begin{equation}
  \begin{split}
    \div \vb B(\vr) &= 0\\
    \curl \vb H(\vr) &= \vb*j_f(\vr)
  \end{split}
\end{equation}
mit $\vb*j_f(\vr)$ die freie Stromdichte und gelten die analoge Randbedingungen
\begin{equation*}
  \begin{split}
    \vb*n\cdot \vb B(\vr) &= 0\\
    \vb*n\times \vb H(\vr) &= \vb*j_f^F(\vr)\\
  \end{split}
\end{equation*}


\newpage
\section{Formelzettel}%

  
  \newpage
  \chapter{Elektrodynamik}\label{cha:elektrodynamik}
  \minitoc%
  \clearpage
    \section{Dynamische Ergänzung der Maxwellgleichungen}%
\label{sec:dynamische-ergaenzungen}
Endlich sind wir angekommen an dem Punkt, wo wir bewegte Ladungen betrachten
können, ohne zeitliche Einschränkungen. Wir versuchen zunächst aus die
statische Maxwellgleichungen allgemein gültige dynamische Maxwellgleichungen
herzuleiten. Hier unten noch kurz eine Wiederholung der Maxwellgleichungen bis jetzt.
\begin{center}
\begin{tabular}{rll}
  &Elektrostatik&Magnetostatik\\
  inhomogen &$\ds \div \vE=4\pi k\rho$
            &$\ds \curl \vB=4\pi k'\vj$\\
  homogen   &$\ds \curl \vE=0$
            &$\ds \div \vB=0$\\
\end{tabular}\\
\vspace{.2cm}
\begin{tabular}{rcc}
  & $\ds k$ & $\ds k'$\\
  SI    & $\ds1/4\pi\epsilon_0$ & $\ds\mu_0/4\pi$\\
  Gauss & $\ds1$ & $\ds1/c$
\end{tabular}
($\ds \epsilon_0\mu_0 = 1/c^2$)
\end{center}
Wir werden sehen, dass die homogene elektrostatische Maxwellgleichung und
die inhomogene magnetostatische Maxwellgleichung in der Elektrodynamik nicht
mehr gültig sein werden.

\subsection{Maxwellsche Ergängzung des Ampèresche Durchfluttungsgesetzes}%
\label{sub:dyn-kontinuitaetsgleichung}
Bis jetzt konnten wir Ströme als reine Quelle des Magnetfeldes auffassen,
wofür die inhomogene magnetostatische Maxwellgleichung (\textit{Ampèresche Durchfluttungsgesetz}) allgemeingültig war.
\begin{equation*}
  \curl \vBr = 4\pi k'\vjr 
\end{equation*}
Betrachten wir nochmal die Kontinuitätsgleichung
Jetzt ist Zeitabhängigkeit erlaubt, Ladungsdichten dürfen sich Zeitlich ändern
und es gilt dadurch im allgemeinen nicht mehr $\div\vjrt=0$ sondern
\begin{equation}
  \div \vj(\vr, t) + \pt \rho(\vr, t) = 0
\end{equation}
Wenn wir diese Aussage kombinieren mit der inhomogene Maxwellgleichung aus der Elektrostatik finden wir nun
\begin{equation}
    \div \vj(\vr, t) + \frac{1}{4\pi k} \pt (\div\vE(\vr, t)) = 0
\end{equation}
Weil diese Ableitungsoperationen alle Partiell sind, darf $\pt$ mit $\nabla$
vertauscht werden, und man findet
\begin{equation}
    \div \vj(\vr, t) = -\frac{1}{4\pi k} \div(\pt \vE(\vr, t)) 
\end{equation}
Dies führt aber zu einem Problem für die inhomogene Maxwellgleichung der Magnetostatik. Denn wenn wir die Divergenz der Roation des Magnetfeldes nehmen,
sollte das Ergebnis Null sein, es gilt aber erstmal.
\begin{equation}
  \div(\curl \vBrt) = \div (4\pi k' \vjrt) \stackrel!=0
\end{equation}
In der magnetostatik war dies einfach erfüllt, denn $\div\vjr=0$ galt. Jetzt finden wir aber
\begin{equation}
  \div(\curl \vBrt) = -\frac{\cancel{4\pi} k'}{\cancel{4\pi} k} \pt\vErt \stackrel{\text{IA}}{\neq} 0
\end{equation}
was im Allgemeinen nicht gleich Null ist! Deswegen brauchen wir die
dynamische Ergänzung der inhomogene Maxwellgleichung der Magnetostatik,
und finden die erste Elektrodynamische Maxwellgleichung.
\begin{equation}
  \curl \vBrt = 4\pi k' \vjrt +  \frac{k'}{k}\pt\vErt 
\end{equation}
\begin{center}
  \textbf{Maxwell-Ampèresche Durchfluttungsgesetz}
\end{center}
Diese ergänzung wurde von Maxwell in 1861 publiziert. Dabei wird heutzutage $\pt\vErt$ die \textbf{Maxwellsche Verschiebungsstrom} genannt.
Man sieht hier natürlich, dass das \textit{Ampèresche Durchfluttungsgesetz} einen Spezialfall ist für stationäre elektrisches Felder.

\clearpage
\subsection{Faradaysche Induktionsgesetz}%
\label{sub:faradaysche-induktionsgesetz}
Im statischen fall galt bislang die homogene elektrostatische Maxwellgleichung
\begin{equation*}
  \curl \vEr = 0
\end{equation*}
Jedoch wurde in 1831 die elektrische Induktion durch Faraday entdeckt. Das Ströme Magnetfelder ezeugen konnten, war schon bekannt, aber die elektrische Induktion zeigte auch das umgekehrte, zwar das Magnetfelder auch Ströme erzeugen könnten. Obwohl Faraday selber nie die
mathematische Formulierung festgelegt hat, er war zwar mathematisch nicht sehr gelehrt, wurden aus seine Experimente die folgenden Zusammenhänge hergeleitet.

Faraday fand, dass falls man ein permanent Magnet nähe eine geschlossene
Leiterschleife bewegte, sich einen Strom in der Leiterschleife entwickelte.
Sei diese Leiterschleife nun die Randkurve $\partial A$ einer kompakte einfach zusammenhängende Oberfläche $A$, so ist der Magnetischen fluss $\Phi_B$ durch die Oberfläche
\begin{equation}
  \Phi_B = \int_A \vBrt \ddA
\end{equation}
Nun wurde entdeckt, dass die zeitliche Änderung von $\Phi_B$ verantwortlich war für die Induktionsströme. Statt der Strom betrachten wir nun jedoch die
induzierte Spannung $U_\text{ind}$, die natürlich über das \textit{Ohmsche Gesetz} mit einander verknüpft sind. Es wurde beobachtet, dass
\begin{equation}
  U_\text{ind} = -k''\dv t \Phi_B(t)
\end{equation}
Dabei ist sowohl eine Änderung des Magnetfeldes als auch eine Änderung der durchfluteten Oberfläche für die Änderung von $\Phi_B$ verantwortlich.
Die Spannung entlang einer Raumkurve wurde in der Elektrostatik gegeben durch
\begin{equation}
  U_{C} = \int_{C} d\vb*l \cdot \vErt
\end{equation}
sodass man für die Induzierte Spannung entlang $\partial A$ die untere Beziehung findet
\begin{equation}
  \oint_{\partial A} d\vb*l \cdot \vErt = -k''\dv{\Phi_B}{t} (t)
\end{equation}
Unter anwendung des Gesetzes von Stokes findet man nun
\begin{equation}
  \int_{A(t)} \dA(t) \cdot (\curl \vErt) = -k'' \dv t \int_{A(t)} \dA(t) \cdot \vBrt
\end{equation}
In diferentieller form findet man die homogene Maxwellgleichung für die Rotation von $\vE$
\begin{equation}
  \curl \vErt + k''\dv t \vBrt = 0
\end{equation}
\begin{center}
  \textbf{Faradaysche Induktionsgesetz} 
\end{center}
Dabei ist $k''$ wieder eine Konstante, die vom Einheitensystem abhängt. In SI gilt $k''=1$, während in Gausseinheiten gilt $k''=1/c$

\subsection{Maxwell Gleichungen im Vakuum}%
\label{sub:maxwell-gleichungen}
Mit die obere Ergänzungen sind wir jetzt zum Punkt gekommen, wo wir die allgemeingültige Maxwellgleichungen (im Vakuum) zusammenfassen können. Aus die Maxwellgleichungen können alle Elektrodynamische Zusammenhänge hergeleitet werden. Sie Lauten
\begin{center}
\begin{tabular}{rll}
  inhomogen &$\ds \div \vE=4\pi k\rho$
            &$\ds \curl \vB - \frac{k'}{k}\pt\vE=4\pi k'\vj$\\
  homogen   &$\ds \curl \vE + k''\pt \vB =0$
            &$\ds \div \vB=0$\\
\end{tabular}\\
\vspace{.2cm}
\begin{tabular}{rccc}
  & $\ds k$ & $\ds k'$ & $k''$\\
  SI    & $\ds1/4\pi\epsilon_0$ & $\ds\mu_0/4\pi$ & $\ds 1$\\
  Gauss & $\ds1$ & $\ds1/c$ & $\ds 1/c $
\end{tabular}
($\ds \epsilon_0\mu_0 = 1/c^2$)
\end{center}


\newpage
\section{Wellengleichung der Felder}%
\label{sec:wellengleichung}
\subsection{Entkopplungsansatz der EM-Felder}%
\label{sub:entkopplungs-ansatz}
Im letzten Abschnitt wurden die Maxwellgleichungen ergänzt. Es fällt auf, dass die Differentialgleichungen der elektrische und magnetische Felder gekoppelt sind. Nun wollen wir die Maxwellgleichungen entkoppeln. Weil es sowohl für das $\vE$-Feld als das $\vB$-Feld eine gekoppelte, als eine freie Differentialgleichungen gibt, ist es sinnvol uns mal anzuschauen, was passiert, wenn wir die Rotation der Rotation der entsprechende Felder näher betrachten.
\begin{equation}
  \curl(\curl \vE) \qquad \curl(\curl\vB)
\end{equation}
Es gelten für 2-Mal stetig differenzierbare Skalar- $\varphi$ bzw. Vektorfelder $\vb F$ die folgende Gleichungen
\begin{equation}
  \begin{split}
    \trot\trot\varphi &= \tdiv\tgrad\varphi - \Delta \varphi\\
    \trot\trot\vb F &= \tgrad\tdiv \vb F - \Delta \vb F
  \end{split}
\end{equation}
Betrachten wir zunächst Felder im ladungsfreien Vakuum, d.h.
\begin{equation*}
  \rho(\vr, t) = 0 \qquad \vjrt = 0
\end{equation*}
So kann man das $\vE$-Feld wie folgt vom $\vB$-Feld entkoppeln
\begin{equation}
  \begin{split}
    \grad\underbrace{(\div \vE)}_{\propto\rho=0} - \Delta \vE
    =
    \curl (\curl \vE) 
    &= -\curl k''\pt\vB 
    =
    -\pt\qty(\underbrace{4\pi k'\vj}_{=0} + \frac{k'}{k}\pt \vE)\\
    \Leftrightarrow
    \qty(\Delta - \frac{1}{c^2}\pt^2) \vE &=0 
  \end{split}
\end{equation}
Analog geht dies für das $\vB$-Feld
\begin{equation}
  \begin{split}
    \grad\underbrace{(\div \vB)}_{=0} - \Delta \vB
    =
    \curl (\curl \vB) 
    &= \curl(\underbrace{4\pi k'\vj}_{=0} + \frac{k'}{k}\pt\vE)
    = -k''\frac{k'}{k} \pt^2\vB\\
    \Leftrightarrow
    \qty(\Delta - \frac{1}{c^2}\pt^2) \vB &=0 
  \end{split}
\end{equation}
Dabei findet man für sowohl SI als Gausseinheiten die $1/c^2$ Konstante vor
die Zeitableitung. Diese Gleichung wird auch die \textbf{(homogene) Wellengleichung} genannt, denn ebene Wellen lösen diese Gleichung, was wir im nächsten Abschnitt zeigen werden. Weiter wird
\begin{equation}
  \qty(\Delta - \frac{1}{c^2}\pt^2)\equiv\Box
\end{equation}
die \textit{d'Alembertsche Operator} genannt. Kompakt schreibt man
\begin{equation}
  \Box \vE = 0 \qquad \Box \vB=0
\end{equation}

\subsection{Lösung der inhomogene Wellengleichung}%
\label{sub:loesung-wellengleichung}
Gesucht ist eine Lösung der inhomogene Wellengleichung für allgemeine Felder.
\begin{equation}
  \Box \psi(\vr, t) = f(\vr, t)
\end{equation}
Für die EM-Felder werden wir dann den Spezialfall betrachten, wo die Störfunktion $f(\vr, t)=0$ ist. Weil wir später auch inhomogene Wellengleichungen entgegenkommen werden, ist es aber Sinvoll, direkt den
Zur Lösung der Wellengleichung schauen wir uns die \textbf{d'Alambert-Greensche Funktion} an. Dabei sind $G$ und $\delta$ nun Funktionen von $\vr, \vr', t, t'$ sodass
\begin{equation}
  \label{eq:dalambert-green}
  \Box \greent \equiv \delta(\vr-\vr')\delta(t-t')
\end{equation}
Es folgt nun nach die 2.\ Greensche Identität
\begin{equation}
  \label{eq:psi-green}
  \psi(\vr, t) = \int d^3r'\int dt' \greent f(\vr', t')
\end{equation}
Wobei man erstmal annimt, das die Randbedingungen verschwinden. Um die Lösung zu finden, möchten wir $\greent$ im Fourierraum betrachten, denn dies wird das Finden einer Lösung vereinfachen. Per Definition gilt
\begin{equation}
  \label{eq:fourier-dalembert-green}
  \greent = \frac{1}{(2\pi)^2} \int d^3k\int d\omega \fgreent e^{i\vk\cdot(\vr-\vr')}
  e^{-i\omega(t-t')}
\end{equation}
Und weiter ist für die Fouriertransformation der $\delta$-Distributionen
\begin{equation}
  \begin{split}
    \delta(\vr-\vr')
    &= 
    \frac{1}{(2\pi)^{3/2}} 
    \int d^3k e^{i\vk\cdot(\vr-\vr')}\\
    \delta(t-t')
    &= 
    \frac{1}{(2\pi)^{1/2}} 
    \int d\omega e^{i\omega(t-t')}\\
  \end{split}
\end{equation}
sodass zusammen mit Gleichung~\ref{eq:dalambert-green}
\begin{equation}
  \label{eq:fourier-dalembert-green2}
  \Box \greent =
    \frac{1}{(2\pi)^2} 
    \int d^3k e^{i\vk\cdot(\vr-\vr')}
    \int d\omega e^{i\omega(t-t')}
\end{equation}
folgt, was dann zusammen mit Gleichung~\ref{eq:fourier-dalembert-green}
\begin{equation}
  \begin{split}
    \Box \greent 
    &= 
    \frac{1}{(2\pi)^2} \int d^3k\int d\omega \fgreent\Box 
    e^{i\vk\cdot(\vr-\vr')}
    e^{-i\omega(t-t')}
    \quad(\text{mit Gleichung~\ref{eq:fourier-dalembert-green}})\\
    &=
    \frac{1}{(2\pi)^2} \int d^3k\int d\omega \fgreent
    \qty(\frac{\omega^2}{c^2}-\vk^2)
    e^{i\vk\cdot(\vr-\vr')}
    e^{-i\omega(t-t')}\\
    &=
    \frac{1}{(2\pi)^2} 
    \int d^3k \int d\omega e^{i\vk\cdot(\vr-\vr')}
    e^{i\omega(t-t')}
    \quad(\text{mit Gleichung~\ref{eq:fourier-dalembert-green2}})
  \end{split}
\end{equation}
Daraus folgt
\begin{equation}
  \label{eq:helmholtz-gleichung}
  \fgreent \qty(\frac{\omega^2}{c^2}-\vk^2) = 1
\end{equation}
Eine naive Lösung dieser Gleichung ist nun einfach die algebraische Manipulation durchzuführen woraus man
\begin{equation}
  \fgreent = \frac{1}{\qty(\omega^2/c^2-\vk^2)} 
\end{equation}
findet. Diese Lösung entspricht auch tatsächlich die spezifische Lösung für die inhomogene Wellengleichung. Jedoch ist dies nicht die einzige Lösung, denn, mann kann Lösungen der Form
\begin{equation}
  g(\vk, \omega)\qty(\frac{\omega^2}{c^2} -\vk^2)=0
\end{equation}
auf Gleichung~\ref{eq:helmholtz-gleichung} drauf addieren ohne die rechte Seite zu andern. Hieraus folgen tatsächlich zwei weitere Lösungen, welche die Lösung der homogene Wellengleichung beschreiben. Um eine nicht triviale Lösung für 
$g(\vk,\omega)$ zu finden machen wir den Ansatz
\begin{equation}
  g_{\pm}(\vk, \omega)=a_\pm(\vk)\delta(\omega\pm c\abs{\vk})
\end{equation}
Sodass $g(\vk, \omega)$ nur an die Nullstellen von $\qty(\omega^2/c^2-\vk^2)=0$ von null verschieden ist.
Die allgemeineste Lösung dieser Gleichung wird dann gegeben durch die
superposition aller mögliche Lösung
\begin{equation}
  \begin{split}
  \fgreent 
  &= \frac{1}{\qty(\omega^2/c^2-\vk^2)} 
  + a_+(\vk)\delta(\omega+c\abs{\vk})
  + a_-(\vk)\delta(\omega-c\abs{\vk})\\
  &\equiv
  G_\text{inh}(\vk, \omega) + g_-(\vk, \omega) + g_+(\vk, \omega)
  \end{split}
\end{equation}

Nun fehlt uns nur die Rücktransformation nach Ortskoordinaten. Setzen wir die gefundene Greensche funktion in Gleichung~\ref{eq:fourier-dalembert-green} ein und setzt man dies weiter in Gleichung~\ref{eq:psi-green} ein. So findet man
\begin{center}
\begin{equation}
  \begin{split}
    \psi(\vr, t)
    &=\frac{1}{4\pi^2} \vint'\int dt'
    \int d^3k \int d\omega 
    \fgreent
    e^{i\vk\cdot(\vr-\vr')}
    e^{-i\omega(t-t')}
    f(\vr', t')\\
    &=\psi_\text{inh}(\vr, t) + \psi_+(\vr, t) + \psi_-(\vr, t)\\
  \end{split}
\end{equation}
  mit 
\begin{equation}
  \begin{split}
    \psi_\text{inh}(\vr, t)
    &=
    \frac{1}{4\pi^2} \vint'\int dt'
    \int d^3k \int d\omega 
    G_\text{inh}(\vk, \omega)
    e^{i\vk\cdot(\vr-\vr')}
    e^{-i\omega(t-t')}
    f(\vr', t')\\
    \psi_\pm(\vr, t)
    &=
    \frac{1}{4\pi^2} \vint'\int dt'
    \int d^3k \int d\omega 
    g_\pm(\vk, \omega)
    e^{i\vk\cdot(\vr-\vr')}
    e^{-i\omega(t-t')}
    f(\vr', t')\\
  \end{split}
\end{equation}
\end{center} 
Betrachten wir zunächst die Lösung der homogene Wellengleichung
\begin{equation}
  \psi_\text{hom}(\vr, t) = \psi_+(\vr, t) + \psi_-(\vr, t)
\end{equation}
Weil die Greensche Funktion nur für inhomogene Probleme geeignet ist, können wir die 


\newpage
\section{Energie des Elektromagnetischen Feldes}%
\label{sec:energie-EM-feld}

\newpage
\section{Wellengleichung der Potentiale}%
\label{sec:Wellengleichung-}

\newpage
\section{Bewegte und Beschleunigte Ladungen}%
\label{sec:Bewegte und Beschleunigte Ladungen}

\newpage
\section{Wellenleiter}%
\label{sec:Wellenleiter}

\newpage
\section{Wellenpaketen}%
\label{sec:Wellenpaketen}

\newpage
\section{Elektrodynamik in Materie}%
\label{sec:Elektrodynamik in Materie}

  
  \newpage
  \chapter{Relativisitische Elektrodynamik}\label{cha:elektrodynamik}
  \minitoc%
  \clearpage
    \section{Koordinaten Transformationen}%
\label{sec:raum-transformationen}

\newpage
\section{Spezielles Relativitätsprinzip und Lorentz Transformation}%
\label{sec:spezielles-relativitaetsprinzip}

\newpage
\section{Relativistische Elektrodynamik}%
\label{sec:relativistische-elektrodynamik}

\newpage
\section{Relativistische Wirkung}%
\label{sec:relativistische-wirkung}

  
  \newpage
  \chapter{Mathematische Grundlagen}\label{cha:mathematische-grundlagen}
  \minitoc%
  \clearpage
    % TODO write introduction

\section{Dirac $\delta$-Distribution}%
\label{sec:dirac-delta-distribution}
Die Dirac $\delta$-Distribution (manchmal auch $\delta$-Funktion genannt) ist
eine normierte Distribution, die es erschafft, mittels der Kalkulus 
infinitisimale Punkte oder ininitisimal dünne Flächen im Raum zu beschreiben.
Für die Elektrodynamik ist dies wichtig, denn die Ladung ist eine eigenschaft
von Punktteilchen. Auch ist die Delta distribution sehr nützlich um 
Flächenladungen zu beschreiben.

\subsection{Definierende Eigenschaft}%
\label{sub:definierende-eigenschaft}
Die $\delta$-Distribution muss die folgende Bedingungen erfüllen
\begin{equation}
  \begin{split}
    \int_{\mathbb{R}^3}d^3r\delta(\vr-\vr_0) &= 1 \\
    \delta(\vr - \vr_0) &= 
    \begin{cases}
      \infty & \vr=\vr_0\\
      0 & \mathrm{sonst}
    \end{cases}
  \end{split}
\end{equation}
Wobei die untere Zusammenhang vor allem andeutet dass die 
$\delta$-Distribution um $\vr_0$ zentriert ist und unendlich spitz ist.
Es folgt für diese Distribution mit eine beliebige Testfunktion $f(\vr)$.
\begin{equation}
    \int_V d^3r \delta(\vr -\vr_0)f(\vr) = 
    \begin{cases}
      f(\vr_0) & \vr_0\in V\\
      0 & \mathrm{sonst}
    \end{cases}
\end{equation}

\subsection{Herleitung $\delta$-Distribution aus Bekannte Kurven}%
\label{sub:herleitung-delta-distribution}
Die Dirac $\delta$-Funktion kann aus mehrere normierbare Kurven hergeleitet
werden. Wichtig ist die Eigenschaft, dass die Funktion im limes einer
Parameter stehts spitzer wird, aber trotzdem normiert bleit, sodass sich
die Distribution immer mehr auf einem Punkt anhäuft. 
Einige Beispiele solcher Funktionen sind
\begin{enumerate}
  \item \textbf{Kleine Stufe}
    \begin{equation}
      \delta(x - x_0) = \lim_{b \to 0} 
      \begin{cases}
        \frac{1}{b} &x\in[x_0 - \frac{b}{2}, x_0 + \frac{b}{2}] \\
        0 & \textrm{sonst}
      \end{cases}
    \end{equation}
  \item \textbf{Gauß Kurve} 
    \begin{equation}
      \delta(x - x_0) = \lim_{\sigma \to 0} 
      \frac{1}{\sqrt{2\pi}\sigma} e^{\frac{(x-x_0)^2}{2\sigma^2}}
    \end{equation}
  \item \textbf{Lorentz Kurve}
    \begin{equation}
      \delta(x - x_0) = \lim_{\eta \to 0} \frac{1}{\pi} 
      \frac{\eta}{\eta^2 + (x-x_0)^2} 
    \end{equation}
\end{enumerate}
Diese Funktionen werden in der Praxis aber nie direkt angewandt, sind
also nur für die Vorstellung gedacht.

\subsection{Einige Wichtige Identitäten}%
\begin{enumerate}
  \item Fundamentaler Eigenschaft
  \begin{equation*}
    \int d^3 f(\vr)\delta(\vr-\vr_0)=f(\vr_0)
  \end{equation*}

  \item Ableitung der $\delta$-Funktion
    \begin{equation*}
      \delta'(\vr-\vr_0)f(\vr) = \delta(\vr-\vr_0)f'(\vr)
    \end{equation*}

  \item Integration der $\delta$-Funktion (Stufenfunktion)
    \begin{equation*}
      \int_{-\infty}^xdx'\delta(x'-x_0) \equiv \Theta(x-x_0) =
      \begin{cases}
        1 & x > x_0 \\
        0 & x < x_0
      \end{cases}
      \quad \Leftrightarrow\quad \dv{x} \Theta(x-x_0) = \delta(x-x_0)
    \end{equation*}

  \item $\delta$-Funktion einer andere Funktion
    \begin{equation*}
      \delta(g(\vr)) = \sum_{\vr_i} \frac{1}{\abs{g(\vr_i)}}
      \delta(\vr-\vr_i) 
      \quad \text{mit}\ \vr_i\ \text{ Nullstellen von } g(x)
    \end{equation*}

  \item Multidimensionale $\delta$-Funktion Zerlegung in 1-D 
    $\delta$-Funktionen (Karthesisch und Krummlienig)
     \begin{equation*}
       \delta(\vr(x,y,z)) 
       = \delta(x)\delta(y)\delta(z) 
       \quad \delta(\vr(u,v,w)) 
       = \frac{1}{\gamma(u,v,w)}\delta(u)\delta(v)\delta(w) 
     \end{equation*}
     Wobei $\gamma$ die Funktionaldeterminante ist
\end{enumerate}


\newpage
\section{Mehrdimensionale Ableitung: Der Nabla Operator}%
\label{sec:mehrdimensionale-mbleitung}
\input{\subtexdir/mathematische-grundlagen/mehr-dim-abl.tex}

\newpage
\section{Taylor Entwicklung}%
\label{sec:taylor-entwicklung}
Die Taylorentwicklung ist eine Näherungsmethode wobei man Funktionen in 
Polynome zerlegt.

Allgeimeine Taylorentwicklung (Kompakte Darstellung):
\begin{equation}
  f(x) = e^{(x - x_0)\cdot D_{x'}}f(x')\big|_{x'=x_0}
\end{equation}
Wobei $D_{x'}$ eine allgemeine Ableitungsoperator ist. In eine Dimension
folgt für $D_{x'}=\dv{x'}$ die bekannte Zusammenhang.
\begin{equation}
  f(x) = e^{(x-x_0)\dv{x'}}f(x')\big|_{x'=x_0} = 
  \sum_{i=0}^{\infty} \frac{f^{(n)}(x_0)}{n!}{(x-x_0)}^n
\end{equation}

Eine Mehrdimensionale Taylorentwicklung wird analog gegeben wobei 
$D_{x'}=\nabla_{x'}$ und $x\to\vb*x = \vb*x(x_1, x_2, \ldots, x_n)$ und
$\vb*x_0\to\vb*x_0(x_{01},\ldots,x_{0n})$ 
\begin{equation}
  f(x) = e^{(\vb*x-\vb*x_0)\nabla_{\vb*x'}}f(\vb*x')\big|_{x'=x_0} = 
  \sum_{i=0}^{\infty} \frac{\nabla_{\vb*x'}^nf(\vb*x')}{n!}
  {(\vb*x-\vb*x_0)}^n\bigg|_{\vb*x'=\vb*x_0}
\end{equation}

Man kann gegebenenfalls auch um nur ein Teil der Variabelen 
$\{x_i,\ldots,x_j\}\subset\{x_1, \ldots, x_n\}$ entwickeln. Dabei teilen
wir die Variabele  auf in $\vb*y$ und $\vb*z$ und entwickeln dann
um $\vb*z=\vb*z_0$
\begin{equation*}
  \vb*x=\vb*x(\underbrace{x_1, x_2,\ldots, x_i}_{\vb*y}, 
  \underbrace{x_{i+1},\ldots, x_{n}}_{\vb*z})=\vb*x(\vb*y, \vb*z) = \vb*y +
  \vb*z
  \Rightarrow f(\vb*x) = f(\vb*y, \vb*z)
\end{equation*}
Wir finden die Entwicklung
\begin{equation}
  f(\vb*y, \vb*z) = e^{\qty((\vb*y + \vb*z) - \vb*z_0)\nabla_{\vb*z'}}
  f(\vb*y, \vb*z')\big|_{\vb*z'=\vb*z_0} = 
  \sum_{n=0}^{\infty} \frac{\nabla^n_{\vb*z'}f(\vb*y,\vb*z')}{n!}
  (\vb*x-\vb*z_0)^n\bigg|_{\vb*z'=\vb*z_0}
\end{equation}

Schneidet man eine Taylorfunktion ab eine bestimmte Ordnung ab, so erhält
man eine gute Näherung für kleine Gebiete um $x_0$. Die Mehrdimensionale
Taylorentwicklung wird in der Elektrodynamik unter anderem für die
Multipolentwicklung benutzt.

Es ist wichtig aufzumerken, dass nicht alle Funktionen eine 
Taylorentwicklung besitzen, sondern nur analytische (holomorphe) 
Funktionen, aber dies ist in der Regel ein Thema für die 
Mathematikvorlesung.


\newpage
\section{Fourier Reihe und Transformation}%
\input{\subtexdir/mathematische-grundlagen/fourier.tex}

\newpage
\section{Integrale im Euklidischen Raum}%
\label{sec:raum-integrale}

\newpage
\section{Integralsätze und Zerlegungssatz}%
\label{sec:integralsaetze}

\newpage
\section{Zerlegundssatz}%
\label{sec:zerlegundssatz}

  
  \newpage
  \chapter{Ausgewählte Herleitungen}\label{cha:ausgewaehlte-herleitungen}
  \minitoc%
  \clearpage
  %  \section{Ladungen und Ladungsverteilungen}%
\label{sub:Ladungen-und-Ladungsverteilungen}
Die Ursprung der EM-Kraft und EM-Felder ist die \textbf{Ladung}. Mathematisch wird die
Ladungsverteilung eines Systems in Raum durch die \textbf{Ladungsdichte} 
gegeben. Ganz algemein
kann eine Ladungsverteilung von N Punktladungen am Zeit t an den Orten $\vb*x_i$ wie folgt beschrieben werden

\begin{equation}
  \rho(\vb*r, t)= \sum_{i=1}^N q_i \delta(\vb*r - \vb*x_i(t))
\end{equation}
Wobei $q_i$ sowohl positiv oder negativ sein kann.

Prinzipiell ist die Ladung in der Natur eine Eigenschaft von Punktteilchen,
sodass es eigentlich keine kontinuierliche Ladungsverteilungen gibt. Auf
makroskopische Skala ist diese Beschreibung jedoch sehr unhandlich, und ist
eine Mittelung der Ladungsdichte eine wesentlich einfachere Beschreibung.
Eine mögliche Definition für eine kontinuierliche Ladungsverteilung aus einer
diskreten Ladungsverteilung sieht wie folgt aus
\begin{equation}
  \rho_{\textrm{knt.}}(\vb*r, t) 
  = 
  \frac{1}{\Delta V(\vb*r)} 
  \int_{\Delta V(\vr)}d^3r'\rho_{\text{dskr}}(\vr')
  = 
  \frac{1}{\Delta V(\vb*r)} 
  \sum_{q_i\in\Delta V(\vr)}q_i
\end{equation}
wobei man das Volumenkästchen $\Delta V$ um $\vr$ zentriert kontinuierlich verschieben kann, und annimt, daß es viele Ladungen gibt, sodaß die Kontinuierliche Ladungsverteilung praktisch glatt wird.

Für Physikalische Systeme betrachtet man in der Regel immer nur beschränkte 
Ladungsverteilungen, das will sagen, die Ladungsdichte soll nur auf 
ein beschränktes Raumbereich von 0 verschieden sein. 
Dazu dürfen Ladungsverteilungen auch keine Pollstellen haben, 
d.h.\ nie divergieren. 
In manche theoretische Fälle, gibt es auch Ladungsverteilungen die 
nicht beschränkt sind, sondern hinreichend schnell im Unendlichen 
verschwinden (z.B.\ exponentiell unterdrückt), sodass zumindest
\begin{equation*}
  \int d^3r \rho(\vb*r, t) = Q_\textrm{ges} < \infty
\end{equation*}
gilt.

In der Elektrostatik betrachtet man erstmal ruhende Ladungsverteilungen, 
d.h.
\begin{equation}
  \rho(\vb*r, t) = \rho(\vb*r)
\end{equation}
Es gibt also keine Ströme. Manchmal werden in der Elektrostatik auch 
quasi statische Prozeße betrachtet, wo die Effekte von bewegte Ladungen 
erstmal vernachlässicht werden. Die Elektrostatik ist somit ein Spezialfall
der Elektrodynamik.


\newpage
\section{Elektrische Kraft, Elektrisches Feld und Elektrisches Potential}%
\label{sub:Ladungen-und-Ladungsverteilungen}
\subsection{Coulombkraft}%
\label{ssub:Coulombkraft}
Schon in 1785 gelang Charles Augustin de Coulomb eine Beschreibung der
Elektrische Kraft, welche man heutzutage dann auch die 
\textbf{Coulombkraft} 
nennt. Sie ist sehr ähnlich zum Newtons Schwerkraftsgesetz und lautet für 
zwei Punktladungen $q_1$ und $q_2$ an den Orten $\vb*r_1$ und $\vb*r_2$ wie 
folgt (im Vakuum)

\begin{equation}
  \vb F_{1\to2}(\vb*r_1, \vb*r_2, q_1, q_2)= k q_1 q_2 \frac{\vb*r_2-\vb*r_1}{\abs{\vb*r_2-\vb*r_1}^3} 
\end{equation}

\noindent
\begin{center}
Oder in bekanntere Form (nicht vektoriell, Punktladungen auf Abstand r)
\end{center}
\begin{equation*}
  F_c = k \frac{q_1 q_2}{r^2} 
\end{equation*}
wobei $\vb F_{1\to2}$ die Kraft ist, die Teilchen 1 auf Teilchen 2 ausübt. 
Sei $q_1q_2>0$ so wirkt die Kraft abstoßend, 
und sei $q_1q_2<0$, so wirkt sie anziehend. Diese Gleichung wird auch das
\textbf{Coulomb-Gesetz} genannt. 
Dabei ist $k$ eine Konstante die vom Einheitensystem abhängt. In SI gilt z.B. 
$k=\frac{1}{4\pi\epsilon_0}$ mit $\epsilon_0$ die \textit{elekrische Permitivität} des Vakuums. In Gauß-Einheiten gilt $k=1$.

Die Elektrische Kraft ist eine Zentralkraft und deswegen 
\textbf{konservativ}, d.h. 
\begin{equation}
  \oint \vb F(\vb* r) \cdot d\vb*s = 0 \stackrel{*}{\Leftrightarrow} \curl \vb F =0
\end{equation}
\begin{center}
(*Satz von Stokes)
\end{center}
Dies heißt auch, dass die Coulombkraft ein skalares Potential $V(\vb*r)$ besitzt, mit
$\vb F(\vb*r)=-\grad V(\vb*r)$

\subsection{Das Elekrtische Feld}%
\label{ssub:E-feld}
Neben die Kraft, die nur über die Wechselwirkung zweier Teilchen definiert
ist, ist es Sinnvoll das \textbf{Elektrische Feld} zu definieren. Man 
definiert das Elektrische Feld aus ersten Prinzipien aus der Coulombkraft.

\begin{equation*}
  \vb F(q_1, q_2, \vb*r_1, \vb*r_2) 
  = 
  q_1 
  \qty(k q_2 \frac{\vb*r_2 - \vb*r_1}{\abs{\vb*r_2 - \vb*r_1}^3})
  = q_1 \vb E_2(\vb*r_1)
\end{equation*}
Man verstehe dies so, dass Teilchen 1 mit Ladung $q_1$ 
sich am Ort $\vb*r_1$ im äußeren elektrisches Feld $\vb E_2$, 
erzeugt von Teilchen 2 mit Ladung $q_2$ welches am Ort $r_2$ liegt, befindet und daher eine Kraft spürt.

Eine Probeladung $q$ im externen elektrischen Feld $\vb E$ spürt also die Kraft

\begin{equation}
  \vb F(\vb*r) = q \vb E(\vb*r)
\end{equation}
Das Elektrische Feld einer Ladungsverteilung $\rho_1$ ist dann ein 
maß für die Kraft, die eine andere, äußere Ladungsverteilung $\rho_2$ 
wegen der Anwesendheid von $\rho_1$ spüren würde. 

\subsection{Das Elekrtische Potential}%
\label{ssub:E-pot}
Weil die Coulombkraft ein
Skalares Potential $V(\vb*r)$ besitzt, 
besitzt das Elektrische Feld auch ein skalares \textbf{Elektrisches Potential}
$\phi(\vb*r)$ mit
\begin{equation}
  \vb E(\vb*r)=-\grad \phi(\vb*r)
\end{equation}

Nun wollen wir wissen, wie man aus eine beliebige 
Ladungsverteilung $\rho(\vb*r)$ das zugehörige
elekrische Potential $\phi(\vb*r)$ bzw.\ elektrische Feld $\vb E(\vb*r)$ 
findet. Aus der Coulombkraft kann man herleiten, dass das Potential
einer Punktladung $Q$ welches im Ursprung liegt gegeben ist durch
\begin{equation*}
  \phi(\vb*r) = k  \frac{Q}{r} \quad r=\abs{\vb*r}
\end{equation*}
Denn $\vb E(\vb*r)=-\grad\phi(\vb*r)=k \frac{Q}{r^2}\vu e_r $ 
woraus die Coulombkraft $F_c(\vb*r)=k \frac{qQ}{r^2}$ wieder folgt.
Liegt das Teilchen nicht im Ursprung, sondern am beliebigen Ort, $\vb*r_0$
so findet man den Zusammenhang (durch eine einfache Translation)
\begin{equation*}
  \phi(\vb*r) = k \frac{Q}{\abs{\vb*r - \vb*r_0}}
\end{equation*}
Das Potential mehrere Punktteilchen ergibt sich aus der addition für die 
Potenziale der einzelnen Teilchen.
\begin{equation*}
  \phi(\vb*r) = \sum_i \phi_i(\vb*r) = k\sum_i \frac{q_i}{\abs{\vb*r-\vb*r_i}}
\end{equation*}
Daraus lässt sich dann über das Limesprozeß einer Riemannsche Summe das 
Potential einer kontinuierliche Ladungsverteilung $\rho(\vb*r)$ definieren.
\begin{equation}
  \label{eq:potential}
  \phi(\vb*r) = k \int d^3 r' \frac{\rho(\vb*r')}{\abs{\vb*r-\vb*r'}} 
\end{equation}

Man kann auch direkt das E-Feld aus die Ladungsverteilung berechnen, falls 
man den Gradient (nach $\vb*r$, nicht $\vb*r'$) von der obere Formel nimmt, 
und findet
\begin{equation}%
  \label{eq:E-feld}
  \vb E(\vb*r) = k \int d^3r' 
  \rho(\vb*r')\frac{(\vb*r-\vb*r')}{\abs{\vb*r-\vb*r'}^3} 
\end{equation}

Das elektrische Potential ist aber nicht eindeutig fetgelegt, denn die
\textbf{Eichtransformation}
\begin{equation}
  \phi'(\vr) = \phi(\vr) + \phi_0 \quad \phi_0=\const
\end{equation}
läßt das Elektrische Feld invariant (denn $\grad\phi_0=0$). Hierdurch wird im
Allgeimeinen das Potential nur als hilfreiches mathematisches Hilfsmittel
gesehen, obwohl quantenmechanische Effekte wie das Aharanov-Bohm-Effekt auch
darauf weisen können, dass das Potential auch ein physikalisches Feld ist. Dies ist aber erstmal nicht wichtig für die Einführung in der Elektrodynamik.
Meistens wird $\phi_0$ so gewählt, sodass $\phi(\vr\to\infty)\to0$,
dies ist aber nicht notwendig.

\subsection{Die Feldgleichungen der Elektrostatik}%
\label{ssub:Die-Feldgleichungen}
Die Divergenz des elektrischen Feldes gibt nun die Quellendichte des Feldes.
Wir wissen schon aus Erfahrung, dass die elektrische Ladung die Quelle der 
Coulombkraft ist und deswegen auch des elektrischen Feldes. Die genaue Zusammenfang zwischen die Divergenz und die Ladungsdichte folgt aus dem Zerlegungssatz. Zusammen mit die
Rotationsfreiheit der Coulombkraft, und somit auch die Rotationsfreiheit des
E-Feldes folgen die beiden \textbf{Feldgleichungen der Elektrostatik}.
\begin{equation}
  \begin{aligned}
    \div \vb E &= 4\pi k \rho(\vb*r) & \text{(inhomogen)}\\
    \curl \vb E &= 0 & \text{(homogen)}
  \end{aligned}
\end{equation}
Diese Feldgleichungen sind allgemeingültig für alle elektrostatische Felder.

\subsection{Poisson Gleichung}%
Weiter folgt aus den Zusammenhänge $\vb E = -\grad \phi$ und 
$\div E = 4\pi k \rho(\vb*r)$ die \textbf{Poisson Gleichung} 
\begin{equation}
  \Delta \phi(\vb*r) = - 4\pi k \rho(\vb*r)
\end{equation}
Die Poisson-Gleichung ist eine Differentialgleichung 2.\ Ordnung. Unter 
vorgabe von \textbf{Randbedingungen} (das geometrische Analogon zu Anfangsbedinungen von DGLS in der Mechanik) sind die Gleichungen~\ref{eq:potential} und~\ref{eq:E-feld} meistens nicht mehr ausreichend, um korrekte Lösungen zu finden. 
Wir werden Später sehen, das sie nur in dem Spezialfall von verschwindenden
Randbedingungen gültig sind, und die allgemeine Zusammenhänge formaler aus der Poisson-Gleichung herleiten.
Das Lösen der Poisson-Gleichung ist aber ein Thema für ein eigenes Kapitel. 

\subsection{Die Hauptaufgabe der Elektrostatik} 
Die Hauptaufgabe der Elektrostatik ist also das Berechnen von 
elektrostatische Potentiale und Felder unter vorgabe von 
Ladungsverteilungen und Randbedingungen. 
Für Hochsymmetrische Probleme sind im Allgemeinen analytische 
Lösungen möglich, für schwierigere Probleme können meistens nur numerische 
Lösungen gefunden werden. Für uns ist erstmal das Finden von analytische
Lösungen interessant um ein Grundverständnis aufzubauen. Dazu werden auch
Methoden besprochen die uns analytische Näherungen geben können, wie die
Multipolentwicklung.


\newpage
\section{Multipolentwicklung}%
\label{sub:Multipolentwicklung}
Zunächst besprechen wir wie man analytische Näherungen finden kann
unter angabe von komplexere Ladungsverteilungen. 
Das Problem ist meistens das Integrieren der
\begin{equation*}
  \frac{1}{\abs{\vb*r-\vb*r'}}
\end{equation*}
Term (in Kombination mit die Ladungsdichte $\rho(\vb*r')$). 
Deswegen möchten wir in der \textbf{karthesische Multipolentwicklung} 
diesen  Term Taylor-entwickln, damit wir es in eine 
Reihe von Polinomiale Terme umwandeln können, 
weil diese Einfach(er) zu integrieren ist. 
Daneben gibt es noch die \textbf{Kugelfächen-Entwicklung} die sich 
insbesondere für Radial- bzw. Rotationssymmetrische Probleme eignet, 
die wir Später besprechen werden. 

\subsection{Karthesische Multipolentwicklung}%
\label{ssub:Karthesische-Multipolentwicklung}
Die karthesische Multipolentwicklung wird mittels eine Taylor-Entwicklung
hergeleitet. Man muss dabei eine Multidimensionale Taylor-Entwicklung 
durchführen.

Eine allgemeine mehrdimensionale Taylorentwicklung wird gegeben durch:
\begin{equation*}
  f(\vb*\alpha, \vb*\beta)
  =\qty(\exp(\vb*\beta\cdot\nabla_{\vb*\beta'})
  f(\vb*\alpha, \vb*\beta'))
  \bigg|_{\vb*\beta'=\vb*\beta_0}
  =\qty(\sum_{n=0}^\infty
  \frac{(\vb*\beta\cdot\nabla_{\vb*\beta'})^n}{n!}
  f(\vb*\alpha, \vb*\beta'))
  \bigg|_{\vb*\beta'=\vb*\beta_0}
\end{equation*}
Man bemerke dass $\nabla$ einen Operator ist! Die exponential Funktion
dient hier nur zur vereinfachung der Darstellung! Man Taylore nun um $\vb*r'=0$ (dies heißt, daß $\vb*r\gg\vb*r'$ sodass 
$\vb*r-\vb*r'\approx\vb*r$, dafür muss $\vb*r$ natürlich weit
von der Quelle entfernt sein). Die Multipolentwicklung ist also ein 
\textbf{Fernfeldnäherung}.

Man definiere nun $f(\vb*r,\vb*r')=f(r_1,r_2,r_3,r_1',r_2',r_3')
\equiv\frac{1}{\abs{\vb*r-\vb*r'}}$ sodass bis zur 2. Ordnung die Taylorentwicklung für unsere Funktion
wie folgt aussieht:

\begin{equation*}
  \frac{1}{\abs{\vb*r-\vb*r'}}=f(\vb*r, 0)
  + (\vb*r'\cdot \nabla_{\vb*{\bar{r}'}}) f(\vb*r, \vb*{\bar{r}}')\bigg|_{\vb*{\bar{r}}'=0}
  + \frac{1}{2}(\vb*r'\cdot\nabla_{\vb*{\bar r'}})^2
  f(\vb*r,\vb*{\bar r}')
  \bigg|_{\vb*{\bar r'}=0}
  + \ldots
\end{equation*}

\underline{0. Ordnung:}
\begin{equation*}
  \frac{1}{\abs{\vb*r-\vb*{\bar r'}}}\bigg|_{\vb*{\bar r'}=0} 
  = \frac{1}{\abs{\vb*r}}= \frac{1}{r} 
\end{equation*}

\underline{1. Ordnung:}
\begin{equation*}
  \vb*{r'}\cdot
  \nabla_{\vb*{\bar{r}'}}\frac{1}{\abs{\vb*r-\vb*{\bar{r}'}}}
  \bigg|_{\vb*{\bar r'}=0}
  = \frac{\vb*{ r'}\cdot\vb*r}{r^3} 
\end{equation*}

\underline{2. Ordnung:}
\begin{equation*}
  \begin{split}
  \frac{1}{2}(\vb*{\bar r'}\cdot\nabla_{\vb*{\bar{r}'}})^2
  \frac{1}{\abs{\vb*r-\vb*{\bar r'}}}\bigg|_{\vb*{\bar r'}=0}
  &=
  \frac{1}{2}\qty[r_i'\pdv {\bar r_i'}]\qty[r_j'\pdv{\bar r_j'}]
  \frac{1}{\abs{\vb*r-\vb*{\bar r'}}}
  \bigg|_{\vb*{\bar r'}=0}\\
  &=
  \frac{r'_ir'_j}{2} 
  \frac{\partial^2}{\partial_{\bar r_i'}\partial_{\bar r_j'}} 
  \frac{1}{\abs{\vb*r-\vb*{\bar r'}}}\bigg|_{\vb*{\bar r'}=0}\\
  &=
  \frac{r'_ir'_j}{2} \frac{\partial}{\partial_{\bar r_i'}} 
  \frac{x_j}{\abs{\vb*r-\vb*{\bar r'}}^3}\bigg|_{\vb*{\bar r'}=0}
  \qquad x_j\equiv (r_j-\bar r_j')\\
  &=
  \frac{r'_ir'_j}{2} \frac{\partial}{\partial_{\bar r_i'}} 
  \frac{x_j}{\abs{\vb*r-\vb*{\bar r'}}^3}\bigg|_{\vb*{\bar r'}=0}\\
  &=
  \frac{r'_ir'_j}{2} 
  \qty(
    \text{-}\frac{3}{2}
    \frac{\text{-}2x_i x_j}{\abs{\vb*r-\vb*{\bar r'}}^5}
    -\frac{\delta_{ij}}{\abs{\vb*r-\vb*{\bar r'}}^3} 
  )\bigg|_{\vb*{\bar r'}=0}\quad\text{(Produktregel)}\\
  &=
  \frac{1}{2}\sum_{i,j=1}^3 \frac{3r_ir_j-\delta_{ij}\vb*r^2}{r^5}
  r_i'r_j'
  \end{split}
\end{equation*}
\textit{Eine wichtige Bemerkung:}
Aus Symmetrie Grunden gilt
\begin{equation*}
  \frac{1}{2}\sum_{i,j=1}^3 \frac{3r_ir_j-\delta_{ij}\vb*r^2}{r^5}
  r_i'r_j'
  =
  \frac{1}{2}\sum_{i,j=1}^3 \frac{3r'_ir'_j-\delta_{ij}\vb*r'^2}{r^5}
  r_ir_j
\end{equation*}
Wir benutzen im allgemeinen die letzte Definition wenn wir die
elektrische und magnetische Multipole berechnen. Man findet also bis zum
2. Ordnung
\begin{equation}
  \frac{1}{\abs{\vb*r-\vb*r'}}
  \approx 
  \frac{1}{r} 
  + \frac{\vb*r'\cdot\vb*r}{r^3} 
  + \frac{1}{2}\sum_{i,j=1}^3 
    \frac{3r'_ir'_j-\delta_{ij}\vb*r'^2}{r^5}r_ir_j
\end{equation}
Setzt man dies in die Definition für das Elektrische Potential ein, so 
findet man
\begin{equation*}
  \begin{split}
    \phi(\vb*r) 
    &= k \int d^3 r' \frac{\rho(\vb*r')}{\abs{\vb*r - \vb*r'}} \\
    & \approx k\int d^3 r' \rho(\vb*r')
    \qty(
    \frac{1}{r} 
    + \frac{\vb*r'\cdot\vb*r}{r^3} 
    + \frac{1}{2}\sum_{i,j=1}^3 
    \frac{3r'_ir'_j-\delta_{ij}\vb*r'^2}{r^5}r_ir_j)\\
    &\equiv 
    k \frac{Q}{r} + k \frac{\vb*p\cdot\vb*r}{r^3} 
    + \frac{k}{2} \sum_{i,j=1}^{3} \frac{r_ir_j}{r^5}Q_{ij}
  \end{split}
\end{equation*}

\begin{center}
\begin{tabular}{ll}
  Monopol:    & $\ds Q=\int d^3r\rho(\vb* r)$ \quad\text(Gesammtladung)\\
  Dipol:      & $\ds p_i=\int d^3r\rho(\vb* r)r_i
                \quad\vb* p=p_i\vu{e}_i $\\
  Quadrupol:  & $\ds Q_{ij}=\int d^3r\rho(\vb* r)
  \qty(3r_ir_j - \delta_{ij}\abs{\vb* r}^2)$
\end{tabular}
\end{center}
Dabei hat die Quadrupoltensor $Q_{ij}$ nur 5 Freiheitsgraden. 
Mit nur 5 Rechnungen alle (9) Quadrupol Elemente berechnen. 
Es gilt zwar $Q_{ij}=Q_{ji}$ (symmertrisch) und $\text{sp}(\bm Q)=\sum_i Q_{ii}=0$ (spurfrei).

Im allgemeinen berechnet 
man keine weitere Ordnungen analytisch im Bachelorstudium, 
und zum Verständniss bringt dies auch nicht mehr (außer ärger), 
sodass höhere Ordnungen berechnen 
ein Problem ist das man lieber an Computer überlässt.

\subsection{Sphärische Multipolentwicklung}%
\label{ssub:sphaerische-Multipolentwicklung}
Für Ladungsverteilungen die Radial- oder Rotationssymetrisch sind ist die 
Sphärische Multipolentwicklung besonders geeignet, vor allem falls man die 
Ladungsdichte als Linearkombination von Kugelfächen-Funktionen schreiben 
kann. Um die Sphärische Multipolentwicklung zu motivieren machen wir 
zunächst die Annahme (mit dem Seperationsansatz), dass das 
Winkelanteil des Potentials unabhähngig vom Radialanteil ist, also
\begin{equation*}
  \phi(\vb*r) = \phi_r(r)\phi_\Omega(\theta,\varphi)
\end{equation*}
Weil man (wie wir später besprechen werden) die Poisson-Gleichung mit dem
gleichen Ansatz lösen wird, und die Kugelflächen-Funktionen 
$Y_{lm}(\theta,\varphi)$ eine Eigenfunktion der Laplace Operator ist, 
ist es eine gute Idee um dies als Basis für eine Sphärische Entwicklung 
zu benutzen. Dazu formen die Kugelflächen-Funktionen unter Integration über
$d\Omega=\sin\theta d\theta d\varphi$ eine orthonormale Basis. Es gilt die untere Zusammenhang.
\begin{equation*}
  \int d\Omega Y_{lm}(\Omega)Y^*_{l'm'}(\Omega)=\delta_{ll'}\delta_{mm'}
\end{equation*}

Die Entwicklung in Kugelkoordinaten ist rechnerisch etwas aufwändig, und
wird hier erstmal übersprungen, aber sie wird in gute Literatur über
die Elektrodynamik oft gegeben, wie z.B. in Kapitel 2.3.8 von Noltings
``Grundkurs Theoretische Physik 3, Elektrodynamik''. 
Man findet 
\begin{equation}
  \frac{1}{\abs{\vb*r-\vb*r'}}
  =
  \sum_{lm} \frac{4\pi}{2l+1} \frac{r_<^l}{r_>^{l+1}}
  Y^*_{lm}(\Omega')Y_{lm}(\Omega)
\end{equation}
\begin{center}
mit $l=0,1,2,\ldots$ und $m=-l,\ldots,l$\\
$r_<=\min(r,r')$, $r_>=\max(r,r')$.
\end{center}
Setzt man dies in der Definition für das Elektrische Potential ein, 
so findet man
\begin{equation}
  \begin{split}
    \phi(\vb*r) 
    &= k\int d^3r' \rho(\vb*r') \frac{1}{\abs{\vb*r-\vb*r'}} \\
    &= k\sum_{lm} \frac{4\pi}{2l+1} 
    \int_0^\infty dr'{r'}^2 \frac{r_<^l}{r_>^{l+1}} 
    \int_\Omega d\Omega' Y^*_{lm}(\Omega')Y_{lm}(\Omega) \rho(\vb*r')\\
    &= k\sum_{lm} \frac{4\pi}{2l+1} Y_{lm}(\Omega) 
    \qty(
    \int_0^{r} dr'{r'}^2 \frac{{r'}^l}{r^{l+1}} +
    \int_{r}^{\infty} dr'{r'}^2 \frac{r^l}{{r'}^{l+1}} 
    )
    \int_\Omega d\Omega' Y^*_{lm}(\Omega')\rho(\vb*r')\\
  \end{split}
\end{equation}
Sei nun die Ladungsdichte 
$\rho(\vb*r')=\sum_{l'm'}f_{f'm'}(r')Y_{l'm'}(\Omega')$
für beliebige $l'$ und $m'$ (also einfach irgendeine Linearkombination von
beliebige Kugelflächenfunktionen) so vereinfacht sich das Problem weiter,
und kann man für eine Endliche Summe sogar exakte Lösungen finden.
(Es ist nicht notwendig dass man die Ladungsdichte als Linearkombination
von Kugelflächenfunktionen schreibt, nur einfacher).
\begin{equation}
  \begin{split}
    \phi(\vb*r) 
    &= k\sum_{lm} \frac{4\pi}{2l+1} Y_{lm}(\Omega) \sum_{l'm'}
    \qty(
    \int_0^{r} dr'\frac{{r'}^(l+2)}{r^{l+1}}f_{l'm'}(r) +
    \int_{r}^{\infty} dr'\frac{r^l}{{r'}^{l-1}}f_{l'm'}(r) 
    )
    \int_\Omega d\Omega' Y^*_{lm}(\Omega')Y_{l'm'}(\Omega')\\
    &= k\sum_{lm} \frac{4\pi}{2l+1} Y_{lm}(\Omega) \sum_{l'm'}
    \qty(
    \int_0^{r} dr'\frac{{r'}^{l+2}}{r^{l+1}}f_{l'm'}(r) +
    \int_{r}^{\infty} dr'\frac{r^l}{{r'}^{l-1}}f_{l'm'}(r) 
    )
    \delta_{ll'}\delta_{mm'}\\
    &= k\sum_{l'm'} \frac{4\pi}{2l+1} Y_{l'm'}(\Omega)
    \qty(
    \int_0^{r} dr'\frac{{r'}^{l+2}}{r^{l+1}}f_{l'm'}(r) +
    \int_{r}^{\infty} dr'\frac{r^l}{{r'}^{l-1}}f_{l'm'}(r) 
    )
  \end{split}
\end{equation}
d.h.\ alle Terme wo $l\neq l'$ oder $m\neq m'$ fallen wegen dem Kronecker-delta weg. Wir werden aber weiter wieder von eine allgemeine Ladungsdichte 
ausgehen.

Ist die Ladungsdichte nun nach Innen oder Nach außen beschränkt, dann kann
man $r_>$ und $r_<$ eindeutig als $r$ oder $r'$ festlegen in bestimmte 
Raumbereiche. Hat man z.B. eine geladene Kugelschale wobei die 
Ladungsdichte, noch von $\theta$ und $\varphi$ abhängen darf, so kann man
das Potential im Inneren mit nur dem $\int_r^\infty dr$ Integral beschreiben, und
dem Außenraum mit durch das $\int_0^r dr$ Integral. Es macht manchmal 
sogar kein Sinn die Lösungen der beide Raumbereiche gleichzeitig zu 
berechnen weil die Lösung für das Innere Raumbereich im äußeren Raumbereich
divergiren kann oder umgekehrt. Wir Teilen hier also zunächst die 
Raumgebiete auf und finden Multipolmomente für dem innen und außen Räume 
$q_<^{lm}$ und $q_>^{lm}$\\

\noindent
Im Innenraum ($r\approx 0$):
\begin{equation}
  \begin{split}
    \phi(\vb*r) 
    &= k\sum_{lm} \frac{4\pi}{2l+1} Y_{lm}(\Omega)
    \int_0^{r} dr'\frac{{r}^{l}}{{r'}^{l-1}}
    \int_\Omega d\Omega'Y^*_{lm}(\Omega')\rho(\vb*r')\\
    &= k\sum_{lm} \frac{4\pi}{2l+1}r^l q_<^{lm} Y_{lm}(\Omega) 
  \end{split}
\end{equation}
\begin{center}
  mit $\ds q_<^{lm}=\int_0^\infty dr' {r'}^{1-l}
  \int_\Omega d\Omega' Y^*_{lm}(\Omega')\rho(\vb*r)$
\end{center}
\noindent
Im Außenraum ($r\gg0$):
\begin{equation}
  \begin{split}
    \phi(\vb*r) 
    &= k\sum_{lm} \frac{4\pi}{2l+1} Y_{lm}(\Omega)
    \int_0^{r} dr'\frac{{r'}^{l+2}}{r^{l+1}}
    \int_\Omega d\Omega'Y^*_{lm}(\Omega')\rho(\vb*r')\\
    &= k\sum_{lm} \frac{4\pi}{2l+1} 
    \frac{q_>^{lm}}{r^{l+1}} Y_{lm}(\Omega) 
  \end{split}
\end{equation}
\begin{center}
  mit $\ds q_>^{lm}=\int_0^\infty dr' {r'}^{l+2}
  \int_\Omega d\Omega' Y^*_{lm}(\Omega')\rho(\vb*r)$
\end{center}


\newpage
\section{Verhalten von Elektrostatische Felder an Randflächen}%
\label{sub:randflaechen}
In diesem Kapitel überlegen wir uns, wie Elektrische Felder sich an 
\textbf{Randflächen} verhalten. Dabei ist es wichtig dass wir die aus den
Vektorkalkulus folgenden \textbf{Satz von Gauß} und \textbf{Satz von Stokes}
näher betrachten.

\subsection{Randflächen und Randkurven}%

Eine \textbf{Randfläche} ist ganz allgemein eine Fläche im Raum die Zwei 
Raumgebiete trennt, z.B. trennt die Randfläche einer Kugel das innere der
Kugel von dem Außenraum. 
Man kann das Vakuum unendlich viele triviale Randflächen
zuordnen. Was interessanter ist, sind Randflächen zwischen z.B. 
unterschiedliche Medien wie Vakuum, metallische Leiter, (un)geladene 
Dielektrika oder Ränder von Ladungsverteilungen. 

Eine \textbf{Randkurve} ist eine Kurve die eine Oberfläche abschließt. 

\subsection{Bedeutung der Sätze von Gauß und Stokes für Elektrostatische 
Felder}%
Zunächst wiederholen wir wie die Sätze mathematisch ganz 
allgemein aussehen.\\

\noindent
Sei $V\subset \mathbb{R}^n$ eine Kompakte Menge mit glattem Rand 
$\partial V$ mit ein nach äußeren orientierten Normaleneinheitsvektor 
$\vb*n$ bzw. Flächen Element $d\vb A =  \vb*n dA$. Sei ferner das Vektorfeld
$\vb F$ stetig differenzierbar auf einer offenen Menge $U$ 
mit $V\subseteq U$ so gilt
\begin{equation}
  \int_V \div \vb FdV  = \oint_{\partial V} \vb F \cdot d\vb A 
  \qquad \textrm{\textbf{Satz von Gauß}}
\end{equation}
Der Satz von Gauß besagt daß die Quellendichte (Divergenz) eines 
Vektorfeldes $\div \vb F$
integriert über ein Volumen $V$ proportional zur Flußintegrals 
des Feldes durch der Randfläche des Volumen $\partial V$.\\

\noindent
Sei $A\subset \mathbb{R}^n$ eine einfach zusammenhängende Fläche mit
glattem Randkurve $\partial A$ (die gegen dem Urzeigersinn durchlaufen 
wird beim Integrieren). Sei ferner das Vektorfeld
$\vb F$ stetig differenzierbar auf einer offenen Menge $U$ 
mit $A\subseteq U$ so gilt
\begin{equation}
  \int_A (\curl \vb F) \cdot d\vb A= \oint_{\partial A} \vb F \cdot d\vb*l 
  \qquad \textrm{\textbf{Satz von Stokes}}
\end{equation}
Der Satz von Stokes besagt daß das Flußintegral über die Wirbeldichte 
(Rotation) eines
Vektorfeldes $\curl \vb F$ über eine Fläche A proportional zur 
Kurvenintegral entlang die Randkurve der Fläche $\partial A$

In der Regel verhalten physikalische Felder sich immer sehr schön, sodass
wir uns über die Stetige Differentationsbedingung erstmal keine gedanken
machen müssen. Die Sätze von Gauß und Stokes schränken unsere zu betrachten
Volumina und Flächen zwar ein, aber nur sehr exotische Objekte erfüllen
die Bedingungen nicht, sodaß wir uns in der Regel auch keine Gedanken
machen müßen ob die mathematische Bedingungen erfüllt sind.

Zusammen mit die Feldgleichungen der Elektrostatik und die obere Sätze
Folgen direkt die Folgende Aussagen
\begin{equation}
  \begin{split}
    \oint_{\partial V} \vb E \cdot d\vb A 
    &
    =\int_V \underbrace{(\div \vb E)}_{4\pi k \rho(\vb*r)} dV
    =4\pi k Q_{V,\text{ges}}\\
    \oint_{\partial A} \vb E \cdot d\vb*l 
    &
    =\int_A \underbrace{(\curl \vb E)}_{0} \cdot d \vb A
    = 0
  \end{split}
\end{equation}

\subsection{Anwendung auf Randflächen}%
\label{ssub:anwendungen-auf-randflaechen}
Man kann nun eine allgemeine Randfläche im Raum betrachten. Legt man ein
kleines Kästchen (Volumen) auf der Rand, zentriert um dem Rand --- also ein
teil des Kästchens liegt an eine Seite der Randfläche, und ein Teil an 
der andere Seite --- und läßt man dieses Kästchen nun immer kleiner werden,
so betrachtet man annäherend zu die Randfläche selbst. Im limes von $V\to0$
findet man sogar lokale Punkte auf der Oberfläche und gilt
\begin{equation*}
  \lim_{V \to 0} \oint_{\partial V} \vb E \cdot 
  d\vb A 
  = \vb E \cdot \vb*n= 4\pi k \sigma(\vb*r)
\end{equation*}
Wobei $\sigma(\vb*r)$ die lokale Flächenladungsdichte auf der Randfläche 
ist.

Analog kann man eine infinitisimale Flächenstückchen 
--- die orthogonal auf der 
Randfläche steht, mit ein Teil an einer Seite der Randfläche und ein Teil
auf der andere Seite --- um die Randfläche zentrieren. Im limes von 
$A\to0$ findet man
\begin{equation*}
    \lim_{A \to 0} \oint_{\partial A} \vb E \cdot d\vb*l 
    = \vb E \times \vb*n
    = 0
\end{equation*}

Es folgen die \textbf{Randflächenbedingungen}
\begin{equation}
  \begin{split}
    \vb E \cdot \vb*n &= 4\pi k \sigma(\vb*r) \\
    \vb E \times \vb*n &= 0
  \end{split}
\end{equation}
Dies bedeutet weiter, daß die orthogonalkomponente 
$\vb E_{\perp}= \vb E\cdot \vb*n$ (in der Anwesendheit einer 
Flächenladungsdichte) einen Sprung um $4\pi k \sigma(\vb*r)$ macht, während
die tangentialkomponente $\vb E_{\parallel}=\vb E \times \vb*n$ immer
stetig ist.


\newpage
\section{Lösung der Poisson Gleichung und Greensche Funktion}%
\label{sub:poisson-green}
In diesem Kapitel besprechen wir wie man die Poisson gleichung lösen kann.

\subsection{die Poissongleichung}%
\label{ssub:poissongleichung}

Wie wir schon in Kapitel~\ref{ssub:Die-Feldgleichungen} hergeleitet haben
gilt in der Elektrostatik die \textbf{Poissongleichung}
\begin{equation*}
  \Delta \varphi (\vb*r) = -4\pi k \rho(\vb*r)
\end{equation*}
Die aufgabe besteht nun darein, $\varphi$ zu finden. Die Poissongleichung 
stellt eine partielle Differentialgleichung 2. Ordnung dar, die gelößt
werden kann unter vorgabe von Randbedingungen (analog zu Anfangsbedingungen
in der Mechanik). 

Praktische Lösungsmethoden sind meistens abhänging von die Geometrie
des zu lösenden Problems. Zunächst besprechen wir aber die Allgemeinste 
Lösung (jedoch meistens nicht schnellste oder einfachste Lösungsweg) der
Poissongleichung, die \textbf{Greensche Funktion}.

\subsection{Die Greensche Identitäte und die Greensche Funktion}%
\label{sub:green}

Für das Poissonproblem gibt es 3 wichtige Identitäten, die 3 Green'sche 
woraus man die allgemeine Lösung der Poissongleichung finden kann 
mittels die Greensche funktion. Zunächst betrachten wir die allgemeine
mathematische Identitäten und wenden sie danach Spezifisch für das Poisson
Problem an.

\subsubsection{1. Green'sche Identität}%
\label{ssub:green-id-1}
Gäbe es ein das Vektorfeld $\vb F = \psi \grad \varphi \equiv \psi \vb \Gamma$ 
mit $\psi$ und $\varphi$ beliebige Skalarfelder mit $\psi\in C^1$ und
$\varphi\in C^2$ (d.h. 1 bzw.\ 2 mal stetig diffbar) so folgt aus dem 
Satz von Gauß
\begin{equation}
  \int_V \underbrace{\qty(\psi\Delta\varphi + 
  \grad\psi\grad\varphi)}_{\div\vb F}dV =
  \oint_{\partial V} \qty(\psi\grad\varphi) \cdot d\vb A
\end{equation}
wobei man beachtet daß
\begin{equation*}
  \div \vb F =\div (\psi \vb \Gamma) = 
  \grad \psi \vb \Gamma + \psi\div\vb \Gamma = \grad \psi \grad \varphi + 
  \psi \Delta \varphi
\end{equation*}
Die erste Greensche Identität ist also einen Spezialfall des Gauß Gesetzes

\subsubsection{2. Green'sche Identität}%
\label{ssub:green-id-2}
Sei nun auch noch $\psi\in C^2$, und gäbe es ein weiteres $\epsilon\in C^1$
sodass man das Vektorfeld 
$\vb F=\psi(\epsilon\grad\psi) - \varphi(\epsilon\grad\psi)$ 
definiert folgt wieder
unter Anwendung der Satz von Gauß.

\begin{equation}
  \int_V \underbrace{\psi\div(\epsilon\grad\varphi) - \varphi\div(\epsilon\grad\psi)}_{\div \vb F} dV
  = \oint_{\partial V} \epsilon (\psi \grad \varphi - \varphi \grad \psi) 
  \cdot d\vb A
\end{equation}

wobei man beachtet daß
\begin{equation*}
  \div \vb F = \div (\psi (\epsilon\grad\varphi) - \varphi(\epsilon\grad\psi))
  = \cancel{(\grad\psi)(\epsilon\grad\varphi)} 
  + \psi\div(\epsilon\grad\varphi)
  - \cancel{(\grad\varphi)(\epsilon\grad\psi)}
  - \varphi\div(\epsilon\grad\psi)
\end{equation*}

Wir sind erstmal interessiert an dem Spezialfall wo $\epsilon=1$ ist, sodass
folgt
\begin{equation}
  \int_V \psi\Delta\varphi - \varphi\Delta\psi dV= 
  \oint_{\partial V} \psi\grad\varphi - \varphi\grad\psi d\vb A
\end{equation}

\subsubsection{3. Green'sche Identität}%
\label{ssub:green-id-3}

Man definiere nun die \textbf{Greensche Funktion} sodass gilt
\begin{equation}
  \Delta \green= \delta(\vb*r - \vb*r')
\end{equation}
Für den Laplace Operator $\Delta$ erfüllt die Funktion 
$\green=-\frac{1}{4\pi}\frac{1}{\abs{\vb*r-\vb*r'}}$ 
diese Bedingung. Die 
Green'sche Funktion ist aber nicht eindeutich definiert, denn die Funktion 
$\green
=-\frac{1}{4\pi}\frac{1}{\abs{\vb*r-\vb*r'}} + F(\vb*r,\vb*r')$ erfüllt die
Bedingung auch, falls $\Delta F(\vb*r, \vb*r')=0$

Setzt man nun $\psi=\green$ und $\varphi=\phi(\vr')$ in die die 2.
Greensche Identität ein, so folgt die 3. Greensche Identität (hier schon
als Spezialfal für das elektrische Potenzial)
\begin{equation}
  \begin{split}
    \int_V G(\vb*r,\vb*r')\underbrace{\Delta\phi(\vb*r')}_{%
    -4\pi k\rho(\vb*r')} 
    - \phi(\vb*r')\underbrace{\Delta \green}_{\delta(\vb*r - \vb*r')}
    dV'
    &= -4\pi k\int_V \rho(\vb*r') \green dV' -\phi(\vb*r)\\
    &= \oint_{\partial V} (\green\grad\phi(\vr')-\phi(\vr')\grad\green)\ddA'\\
  \end{split}
\end{equation}
Es folgt die Allgemeine Lösung für das Potenzial
\begin{equation*}
  \Leftrightarrow \phi(\vr) = -4\pi k\int_V \rho(\vr')\green dV' 
  - \oint_{\partial V} (\green\grad\phi(\vr')-\phi(\vr')\grad\green)\ddA
\end{equation*}

Bemerke, dass falls $\phi$ und $\green$ im Unendlichen nach null abfallen,
mit $\green=-\frac{1}{4\pi} \frr $ einfach Funktion~\ref{eq:potential} folgt.
Dies ist also ein Spezialfall der allgemeine Lösung von $\phi(\vr)$, wo
keine Randbedingungen vorgegeben sind, bzw.\ wo die Randbedingungen im
Unendlichen liegen und deswegen verschwinden.

Ist nun auf die Randfläche $\phi$ vorgegeben (Dirichlet Randbedingungen), 
so kann man $G$ so wählen, 
sodass $\green[D]\equiv\green\big|_{\vr'\in\partial V}= 0$ gilt, und vereinfacht sich 
die 3. Greensche Identität sich zu
\begin{equation}
  \phi(\vb*r') = -4\pi k\int_V \rho(\vr')\green[D] dV' 
  + \oint_{\partial V} \phi(\vr')\grad\green[D] \ddA
\end{equation}
vereinfacht


\newpage
\section{Spiegelladungsmethode}%

\newpage
\section{Elektrostatik in Materie}%

\newpage
\section{Energie von Elektrostatische Felder}%


\end{document}
