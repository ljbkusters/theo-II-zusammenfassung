\subsubsection{Coulombkraft}%
\label{ssub:Coulombkraft}
Schon in 1785 gelang Charles Augustin de Coulomb eine Beschreibung der
Elektrische Kraft, welche man heutzutage dann auch die 
\textbf{Coulombkraft} 
nennt. Sie ist sehr ähnlich zum Newtons Schwerkraftsgesetz und lautet für 
zwei Punktladungen $q_1$ und $q_2$ an den Orten $\vb*r_1$ und $\vb*r_2$ wie 
folgt (im Vakuum)

\begin{equation}
  \vb F_{1\to2}(\vb*r_1, \vb*r_2, q_1, q_2)= k q_1 q_2 \frac{\vb*r_2-\vb*r_1}{\abs{\vb*r_2-\vb*r_1}^3} 
\end{equation}

\noindent
\begin{center}
Oder in bekanntere Form (nicht vektoriell, Punktladungen auf Abstand r)
\end{center}
\begin{equation*}
  F_c = k \frac{q_1 q_2}{r^2} 
\end{equation*}
wobei $\vb F_{1\to2}$ die Kraft ist, die Teilchen 1 auf Teilchen 2 ausübt. 
Sei $q_1q_2>0$ so wirkt die Kraft abstoßend, 
und sei $q_1q_2<0$, so wirkt sie natürlich anziehend.

\noindent
Die Elektrische Kraft ist eine Zentralkraft und deswegen 
\textbf{konservativ}, d.h. 
\begin{equation}
  \oint \vb F(\vb* r) \cdot d\vb*s = 0 \stackrel{*}{\Leftrightarrow} \curl \vb F =0
\end{equation}
\begin{center}
(*Satz von Stokes)
\end{center}
Dies heißt auch, dass die Coulombkraft ein skalares Potential $V(\vb*r)$ besitzt, mit
$\vb F(\vb*r)=-\grad V(\vb*r)$

\subsubsection{Das Elekrtische Feld}%
\label{ssub:E-feld}
Neben die Kraft, die nur über die Wechselwirkung zweier Teilchen definiert
ist, ist es Sinnvoll das \textbf{Elektrische Feld} zu definieren. Man 
definiert das Elektrische Feld aus ersten Prinzipien aus der Coulombkraft.

\begin{equation*}
  \vb F(q_1, q_2, \vb*r_1, \vb*r_2) 
  = 
  q_1 
  \qty(k q_2 \frac{\vb*r_2 - \vb*r_1}{\abs{\vb*r_2 - \vb*r_1}^3})
  = q_1 \vb E_2(\vb*r_1)
\end{equation*}
Man verstehe dies so, dass Teilchen 1 mit Ladung $q_1$ 
sich am Ort $\vb*r_1$ im äußeren Elektrisches Feld $\vb E_2$, 
erzeugt von Teilchen 2 mit 
Ladung $q_2$ das am ort $r_2$ liegt, befindet und daher eine Kraft spürt.

Eine Probeladung $q$ im externen elektrischen Feld $\vb E$ spürt also die Kraft

\begin{equation}
  \vb F(\vb*r) = q \vb E(\vb*r)
\end{equation}

Das Elektrische Feld einer Ladungsverteilung $\rho_1$ ist dann ein 
maß für die Kraft, die eine andere, äußere Ladungsverteilung $\rho_2$ 
wegen der Anwesendheid von $\rho_1$ spüren würde. 

\subsubsection{Das Elekrtische Potential}%
\label{ssub:E-pot}
Weil die Coulombkraft ein
Skalares Potential $V(\vb*r)$ besitzt, 
besitzt das Elektrische Feld auch ein skalares \textbf{Elektrisches Potential}
$\phi(\vb*r)$ mit
\begin{equation}
  \vb E(\vb*r)=-\grad \phi(\vb*r)
\end{equation}

Dabei ist $k$ die gleiche Konstante die auch im Coulombkraft auftauchte. Diese
konstate ist abhängig vom gewählten Einheitssystem. In SI gilt z.B. 
$k=\frac{1}{4\pi\epsilon_0}$ mit $\epsilon_0$ die elekrische Permitivität des
Vakuums, und in Gauß-Einheiten gilt $k=1$

Nun wollen wir wissen, wie man aus eine beliebige 
Ladungsverteilung $\rho(\vb*r)$ das zugehörige
elekrische Potential $\phi(\vb*r)$ bzw.\ elektrische Feld $\vb E(\vb*r)$ 
findet. Aus der Coulombkraft kann man herleiten, dass das Potential
einer Punktladung $Q$ welches im Ursprung liegt gegeben ist durch
\begin{equation*}
  \phi(\vb*r) = k  \frac{Q}{r} \quad r=\abs{\vb*r}
\end{equation*}
Denn $\vb E(\vb*r)=-\grad\phi(\vb*r)=k \frac{Q}{r^2}\vu e_r $ 
woraus die Coulombkraft $F_c(\vb*r)=k \frac{qQ}{r^2}$ wieder folgt.
Liegt das Teilchen nicht im Ursprung, sondern am beliebigen Ort, $\vb*r_0$
so findet man den Zusammenhang (durch eine einfache Translation)
\begin{equation*}
  \phi(\vb*r) = k \frac{Q}{\abs{\vb*r - \vb*r_0}}
\end{equation*}
Das Potential mehrere Punktteilchen ergibt sich aus der addition für die 
Potenziale der einzelnen Teilchen.
\begin{equation*}
  \phi(\vb*r) = \sum_i \phi_i(\vb*r) = k\sum_i \frac{q_i}{\abs{\vb*r-\vb*r_i}}
\end{equation*}
Daraus lässt sich dann über das Limesprozeß einer Riemannsche Summe das 
Potential einer kontinuierliche Ladungsverteilung $\rho(\vb*r)$ definieren
\begin{equation}
  \phi(\vb*r) = k \int d^3 r' \frac{\rho(\vb*r')}{\abs{\vb*r-\vb*r'}} 
\end{equation}
Man kann auch direkt das E-Feld aus die Ladungsverteilung berechnen, falls 
man den Gradient (nach $\vb*r$, nicht $\vb*r'$) von der obere Formel nimmt, 
und findet
\begin{equation}
  \vb E(\vb*r) = k \int d^3r' 
  \rho(\vb*r')\frac{(\vb*r-\vb*r')}{\abs{\vb*r-\vb*r'}^3} 
\end{equation}

\subsubsection{Die Feldgleichungen der Elektrostatik}%
\label{ssub:Die-Feldgleichungen}
Die Divergenz des elektrischen Feldes gibt nun die Quellendichte des Feldes.
Wir wissen schon aus Erfahrung, dass die elektrische Ladung die Quelle der 
Coulombkraft ist und deswegen auch des elektrischen Feldes. Zusammen mit die
Rotationsfreiheit der Coulombkraft, und somit auch die Rotationsfreiheit des
E-Feldes folgen die beiden \textbf{Feldgleichungen der Elektrostatik}.
\begin{equation}
  \begin{aligned}
    \div \vb E &= 4\pi k \rho(\vb*r)\\
    \curl \vb E &= 0
  \end{aligned}
\end{equation}
Diese Feldgleichungen sind allgemeingültig für alle elektrostatische Felder.

Weiter folgt aus den Zusammenhänge $\vb E = -\grad \phi$ und 
$\div E = 4\pi k \rho(\vb*r)$ die \textbf{Poisson-Gleichung} 
\begin{equation}
  \Delta \phi(\vb*r) = - 4\pi k \rho(\vb*r)
\end{equation}
Die Poisson-Gleichung ist eine Differentialgleichung 2.\ Ordnung. Unter 
vorgabe von Randbedingungen kann man auch ohne die Ladungsverteilung zu 
kennen das Potential und daraus das Elekrtische Feld berechnen. Das Lösen
der Poisson-Gleichung ist ein Thema für ein eigenes Kapitel. 

\textbf{Die Hauptaufgabe der Elektrostatik} ist also das Berechnen von 
elektrostatische Potentiale und Felder unter vorgabe von 
Ladungsverteilungen oder Randbedingungen. 
Für Hochsymmetrische Probleme sind im Allgemeinen analytische 
Lösungen möglich, für schwierigere Probleme können meistens nur numerische 
Lösungen gefunden werden. Für uns ist erstmal das Finden von analytische
Lösungen interessant um ein Grundverständnis aufzubauen. Dazu werden auch
Methoden besprochen die uns analytische Näherungen geben können, wie die
Multipolentwicklung.
