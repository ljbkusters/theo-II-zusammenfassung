Die bisherige hergeleitete Formeln sind alle Gülltig im Vakuum. In Medien
sowie Dielektrika oder Leiter können jedoch mikroskopische Interaktionen
entstehen, weil Atome und Moleküle aus geladenen Teilchen bestehen, die
Elektrische Felder verändern können. 
Eine Mikroskopische Beschreibung von Vielteilchen Systeme ist aber nicht
das Ziel der makroskopische Elektrodynamik. 
Deswegen werden zunächst auf Makroskopische Skala
vereinfachungen eingeführt, woraus man ein sehr 
praktisches und verständliches Bild der Elektrodynamik herleiten kann.

\subsection{Materialarten}%
Man unterscheided zunächst zwischen zwei Arten von Medien, zwar 
(Metallische) 
\textbf{Leiter} und \textbf{Dielektrika} (auch Isolatoren genannt). 

\begin{enumerate}
  \item In Leiter gibt es, 
neben gebundenen Ladungen (Valenzelektronen, Protonen), 
\textbf{freie Ladungen} (Leitungselektronen) die 
frei durch dem Medium bewegen können. Die freie Ladungen können angelegte
elektrische Felde entgegenwirken und deswegen aufheben.
  \item In Dielektrika sind alle Ladungen gebunden. Die einzige Gegenreaktion 
    die die gebundene Ladungen auf äußere Felder haben können, 
sind \textbf{Polarisationseffekte} (denke z.B. an Verformung/Verschiebung der 
Elektronenwolke um dem Atom). Gitterstrukturen von Atomen können weiter
dafür sorgen das diese Polarisationseffekte richtungsabängig sind. Neben
induzierte Dipole können auch spontane (das heißt nich von äußere felder
ausgelöste) Dipole in Dielektrika auftreten (z.B. inherentes 
Dipolmoment vom Wassermolekül).
\end{enumerate}
(In Leiter können auch Polarisationseffekte auftreten, jedoch sind diese
meistens einige größenordungen schwacher als die Effekte der freien Ladungen)

Um die aufspaltung von freie und gebundene Ladungen mathematisch zu 
beschreiben
führen wir die zunächst die freie bzw. die gebundene Ladungsdichte 
$\rho_\text{f}(\vr)$ und $\rho_\text{geb}(\vr)$ ein 
wobei natürlich gilt, dass
\begin{equation}
  \label{eq:ladungsdiche-geb-frei}
  \rho(\vr) = \rho_\text{f}(\vr) + \rho_\text{geb}(\vr)
\end{equation}


\subsection{Polarisation und dielektrische Verschiebung}%
\label{sub:Polarisation}
Mikroskopisch gibt es in einem Polarisierten Medium einen Satz von Dipole,
die als Punktdipole $\vb*p_i$ betrachtet werden können. Wir füren jetzt die
Größe $\vb P$ ein, die \textbf{Polarisation}, die ein Maß ist für die 
Polarisationsdichte in einem Raumbereich und definieren es formal als 
Raummittelung von Punktdipole (analog zu Punktladung und Ladungsdichte) 
\begin{equation}
  \vb P(\vr) = \frac{1}{V(\vr)}\sum_{\vb*p_i\in V(\vr)} \vb*p_i
\end{equation}

Die Polarisation kann aufgeteilt werden in die spontane Polarisation 
$\vb P_\text{sp}$ und eine induzierte Polarisation $\vb P_\text{ind}$ sodass
\begin{equation}
  \vb P(\vr) = \vb P_\text{sp}(\vr) + \vb P_\text{ind}(\vr)
\end{equation}
Dabei wird die induzierte Polarisation durch äußere elektrische Felder erzeugt
und es gilt ganz allgeimen
\begin{equation}
  \vb P_\text{ind} = \frac{\vu*\chi\qty(\vr, \vb E(\vr))}{4\pi k} \cdot \vb E(\vr)
\end{equation}
Wobei $\vu*\chi$ eine $3\times3$-Matrix ist die Richtungsabhängigkeit der
Polarisation zulässt. $\vu*\chi$ ist weiter ganz allgemein eine Funktion
von $\vr$ und $\vb E(\vr)$. Obwohl in viele Medien tatsächlich eine Orts-
und elektrisches Feldabhängigkeit anwesend sein kann, werden in diese in
der Regel nicht in Einführungskurse der Elektrodynamik behandelt. Man 
betrachtet stattdessen meistens \textbf{lineare}, \textbf{homogene} und 
\textbf{isotrope} Medien (LHI-Medien). Das heißt,
\begin{enumerate}
  \item isotropie: Raumrichtungsunabhängigkeit $\rightarrow$ 
    $\vu*\chi=\mathbb{1}\chi$ bzw.\ die $3\times3$-Matrix 
    durch eine skalare Funktion ersetzt werden.
  \item homogenität: Ortsunabhängigkeit $\rightarrow$ 
    $\vu*\chi$ unabängig von $\vr$
  \item liniarität: Elektrische Feldstärke unabhängig $\rightarrow$
    $\vu*\chi$ unabhänging (oder sehr sehr schwach abhänging) 
    von $\vb*E(\vr)$ sodaß $\abs{\vb P} 
    \propto \abs{\vb E}$ eine lineare Zusammenhang erfüllt
\end{enumerate}
Als Spezialfall für LHI-Medien gilt deswegen
\begin{equation}
  \vb P_\text{ind} = \frac{\chi}{4\pi k} \vb E(\vr)
\end{equation}
wobei $\chi$ eine Materialabähnige Konstante ist.

Die Polarisation hat als Quelle die gebundene Ladungsdichte und es gilt
\begin{equation}
  \div \vb P(\vr) = -\rho_\text{geb}(\vr)
\end{equation}
wobei das Vorzeichen eine mathematische Formalität ist, um anzudeuten dass
die Polarisation das Elektrische Feld entgegenwirkt.

Die freie Ladungsdichte $\rho_\text{f}$ verursacht nun noch ein Zusätzliches
Feld das sich sehr ähnlich zum Elektrischen Feld verhält. Dieses Feld
wird die \textbf{dielektrische Verschiebung} $\vb D$ genannt und es gilt
\begin{equation}
  \div \vb D(\vr) = \rho_f(\vr)
\end{equation}

Mit Gleichung~\ref{eq:ladungsdiche-geb-frei} und $\div\vb E=4\pi k \rho$
folgt nun
\begin{equation*}
  \div \vb E(\vr) = 4\pi k \rho(\vr)=4\pi k(\div\vb D(\vr)-\div P(\vr))
\end{equation*}
sodass folgt
\begin{equation}
  \vb E(\vr) = 4\pi k (\vb D(\vr) - \vb P(\vr))
\end{equation}
Vernachlässigen wir zunächst die spontane Polarisation (d.h.\ wir setzen  
$\vb P_\text{sp}(\vr)=0$) und betrachten wir LHI-Medien so folgt
\begin{equation}
  \begin{split}
    \vb E(\vr) 
    &= 4\pi k \qty(\vb D(\vr) - \frac{\chi}{4\pi k} \vb E(\vr))\\
    \Leftrightarrow \frac{(1+\chi)}{4\pi k} \vb E(\vr) 
    &= \vb D(\vr)\\
    \stackrel{\text{SI}}{\Leftrightarrow} 
    \vb \epsilon_0\epsilon_r \vb E(\vr) 
    &\equiv 
    \epsilon \vb E(\vr)
    = \vb D(\vr) \quad \text{mit}\quad \epsilon_r = (1+\chi) \\
  \end{split}
\end{equation}
Für LHI-Medien ohne spontane Polarisation gilt also (in SI einheiten)
\begin{equation*}
  \epsilon \vb E(\vr) = \vb D(\vr) 
\end{equation*}

\subsection{Randbedingungen an Dielektrika}%
\label{sub:randbedingungen-an-dielektrika}
Es gelten an Dielektrika die Feldgleichungen
\begin{equation}
  \begin{split}
    \div \vb D(\vr) &= \rho_\text{f}(\vb r)\\
    \curl \vb E(\vr) &= 0\\
  \end{split}
\end{equation}
Das heißt dann, dass an eine Randfläche die 
Tangentialkomponente von $\vb E$ stetig ist, und
die Orthogonalkomponente von $\vb D$ einen Sprung um $\rho_\text{f}$ 
machen kann.
