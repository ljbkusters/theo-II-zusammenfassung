In diesem Kapitel besprechen wir wie die Energie von Ladungsverteilungen 
in externe Felder aussieht, sowie die Wechselwirkungsenergie zweier
Ladungsverteilungen.

\subsection{Energie von Ladungen im Externen Feld}%
Bewegt sich eine Ladung $q$ 
durch ein äußeres Elektrisches Feld $\vb E_{\mathrm{a}}(\vr)$ mit
Potential $\phi_{\mathrm{a}}(\vr)$, so spürt es die Kraft $\vb F(\vr)=q\vb E(\vr)$ und wird
die Arbeit
\begin{equation*}
  W 
  = -\int_C \vb F(\vr) \cdot d\vb l 
  = -q\int_C \vb E_{\mathrm{a}}(\vr) \cdot d\vb l 
  = q\int_{C} \grad\phi_{\mathrm{a}}(\vr) \cdot d\vb l 
  = q(\phi_{\mathrm{a}}(\vr_1)-\phi_{\mathrm{a}}(\vr_0))
  = qU_{\vr_0\to\vr_1}
\end{equation*}
verrichtet, wobei $C$ eine beliebige Raumkurve von $\vr_0$ nach $\vr_1$ ist.
Betrachtet man statt eine Punktladung eine Ladungsdichte $\rho(\vr)$ im
äußeren E-Feld, so muss man den oberen Zusammenhang über dem Volumen
integrieren. Setzt man $\phi_{\mathrm a}(\vr_0)=0$ (Potential
wird immer relativ definiert), so findet man für
die Potentielle Energie den untere Zusammenhang
\begin{equation}
  \label{eq:Epot}
  E_{\mathrm{pot}}=\int d^3r \rho(\vr)\phi_{\mathrm{a}}(\vr)
\end{equation}

Sei die Ladungsverteilungen auf einem kleinen Raumbereich um $\vr_0=0$ 
beschränkt, so kann man das äußere Potential Taylorentwickeln um eine
gute näherung für die Potentielle Energie zu finden.
\begin{equation*}
  T(\phi_{\mathrm{a}}(\vr); \vr_0=0) = 
  \phi_{\mathrm{a}}(0) 
  - \vb E_{\mathrm{a}}(0)\cdot\vr 
  - \frac{1}{2} \sum_{ij} \del j {E_{\mathrm{a},i}}(0)r_ir_j
  + \ldots
\end{equation*}
Setzt man dies in Gleichung~\ref{eq:Epot} ein, so findet man
\begin{equation}
  E_{\mathrm{pot}}\approx 
  \phi_{a}(0)Q 
  - \vb E_{a}(0) \cdot \vb*p
  - \frac{1}{6} \sum_{ij} \del j E_i(0)Q_{ij}
\end{equation}
Wobei $Q$, $\vb*p$ und $Q_{ij}$ die Multipolmomente sind. Man bemerke also,
dass (näherungsweise für ausgedehnte Ladungsverteilungen) die 
elektrostatische potentielle Energie verursacht wird durch die Kopplung
(d.h. ``mathematische interaktion'') von die Gesammtladung mit dem äußeren 
Potential und den Dipol mit dem E-Feld usw.

\subsection{Feldenergie und Selbstinteraction}%
\label{sub:feldenergie}
Bis jetzt wurden die Effekte die die Ladungsverteilungen selbst verursacht
vernachlässigt. Jetzt möchten wir uns aber anschauen wie wir die gesammte
energie eines Systems berechnen können, wenn wir nicht ausgehen von äußere
felder. Sei nun $\rho(\vr)$ die Ladungsverteilungen des gesammten Raumes,
mit das zugehörige Elektrische Potential $\phi_\rho(\vr)$.

Es gilt noch immer die Zusammenhang aus Gleichung~\ref{eq:Epot}, stellt 
nun jedoch keine potentielle Energie in einem äußeren Feld mehr dar 
sondern eine Wechselwirkungsenergie und es gilt,
\begin{equation}
  E_{\mathrm{WW}}=\int d^3r\rho(\vr)\phi_\rho(\vr)
  =\int d^3r\rho(\vr)\int d^3 r' \frac{\rho(\vr')}{\rr}
  =\int d^3r\int d^3 r' \frac{\rho(\vr)\rho(\vr')}{\rr}
\end{equation}
Über die Poissongleichung $\Delta\phi(\vr)=-4\pi k \rho(\vr)$ 
kann man dies weiter umformen nach
\begin{equation}
   \begin{split}
      E_{\mathrm{WW}}
      &=\int_{\mathbb{R}^3} d^3r\rho(\vr)\phi(\vr)
      =-\frac{1}{8\pi k} \int_{\mathbb{R}^3} 
      d^3r\phi(\vr)\Delta\phi(\vr) \\
      &\stackrel*= 
      \frac{1}{8\pi k} \int_{\mathbb{R}^3} d^3r
      \underbrace{\grad\phi(\vr)\grad\phi(\vr)}_{\vb E^2(\vr)}
      -\frac{1}{8\pi k}\underbrace{\oint_{\partial\mathbb{R}^3}
        \phi(\vr)\grad\phi(\vr) \ddA}_{=0\ (\phi(\vr\to\infty)\to0)} \\
      &=\frac{1}{8\pi k} \int d^3r \vb E^2(\vr) 
   \end{split}
\end{equation}
\begin{center}
  (* 1. Greensche Identität)
\end{center}
