In diesem Kapitel besprechen wir wie man die Poisson gleichung lösen kann.

\subsection{die Poissongleichung}%
\label{ssub:poissongleichung}

Wie wir schon in Kapitel~\ref{ssub:Die-Feldgleichungen} hergeleitet haben
gilt in der Elektrostatik die \textbf{Poissongleichung}
\begin{equation*}
  \Delta \phi (\vb*r) = -4\pi k \rho(\vb*r)
\end{equation*}
Die aufgabe besteht nun darein, $\phi$ zu finden. Die Poissongleichung 
stellt eine partielle Differentialgleichung 2. Ordnung dar, die gelößt
werden kann unter vorgabe von \textbf{Randbedingungen}.

Praktische Lösungsmethoden sind meistens abhänging von die Geometrie
des zu lösenden Problems und die vorgegebene Randbedingungen. 
Zunächst besprechen wir aber die Allgemeinste 
Lösung der Poissongleichung, über die \textbf{Greensche Funktion} der Poissongleichung.

\subsection{Die Greensche Identitäte und die Greensche Funktion}%
\label{sub:green}

Für das Poissonproblem gibt es 3 wichtige Identitäten, 
woraus man die allgemeine Lösung der Poissongleichung finden kann 
mittels die Greensche funktion. Zunächst betrachten wir die allgemeine
mathematische Identitäten und wenden sie danach Spezifisch für das Poisson
Problem an.

\subsubsection{1. Greensche Identität}%
\label{ssub:green-id-1}
Gäbe es ein Vektorfeld $\vb F = \psi \grad \varphi \equiv \psi \vb \Gamma$ 
mit $\psi$ und $\varphi$ beliebige Skalarfelder mit $\psi\in C^1$ und
$\varphi\in C^2$ (d.h. 1 bzw.\ 2 mal stetig diffbar) so folgt aus dem 
Satz von Gauß
\begin{equation}
  \int_V \underbrace{\qty(\psi\Delta\varphi + 
  \grad\psi\grad\varphi)}_{\div\vb F}dV =
  \oint_{\partial V} \qty(\psi\grad\varphi) \cdot d\vb A
\end{equation}
wobei man beachtet daß
\begin{equation*}
  \div \vb F =\div (\psi \vb \Gamma) = 
  \grad \psi \vb \Gamma + \psi\div\vb \Gamma = \grad \psi \grad \varphi + 
  \psi \Delta \varphi
\end{equation*}
Die erste Greensche Identität ist also einen Spezialfall des Gauß Gesetzes

\subsubsection{2. Greensche Identität}%
\label{ssub:green-id-2}
Sei nun auch noch $\psi\in C^2$, und gäbe es ein weiteres $\epsilon\in C^1$
sodass man das Vektorfeld 
$\vb F=\psi(\epsilon\grad\psi) - \varphi(\epsilon\grad\psi)$ 
definiert folgt wieder
unter Anwendung der Satz von Gauß.

\begin{equation}
  \int_V \underbrace{\psi\div(\epsilon\grad\varphi) - \varphi\div(\epsilon\grad\psi)}_{\div \vb F} dV
  = \oint_{\partial V} \epsilon (\psi \grad \varphi - \varphi \grad \psi) 
  \cdot d\vb A
\end{equation}

wobei man beachtet daß
\begin{equation*}
  \div \vb F = \div (\psi (\epsilon\grad\varphi) - \varphi(\epsilon\grad\psi))
  = \cancel{(\grad\psi)(\epsilon\grad\varphi)} 
  + \psi\div(\epsilon\grad\varphi)
  - \cancel{(\grad\varphi)(\epsilon\grad\psi)}
  - \varphi\div(\epsilon\grad\psi)
\end{equation*}

Wir sind erstmal interessiert an dem Spezialfall wo $\epsilon=1$ ist, sodass
folgt
\begin{equation}
  \int_V \psi\Delta\varphi - \varphi\Delta\psi dV= 
  \oint_{\partial V} \psi\grad\varphi - \varphi\grad\psi d\vb A
\end{equation}

\subsubsection{3. Greensche Identität}%
\label{ssub:green-id-3}

Die Greensche Funktion $G$ ist eine Allgemeine Name für Funktionen, worauf unter Anwendung eines lineares differentieles Operators $L$ auf diesen Funktion, sich die Dirac'sche $\delta$-Distribution ergibt. Es gilt also
\begin{equation}
  LG = \delta 
\end{equation}

Man definiere nun die \textbf{Poisson-Greensche Funktion} (weiter erstmal einfach Greensche Funktion genannt) zum Laplace Operator $\Delta$ sodass gilt
\begin{equation}
  \Delta \green= \delta(\vb*r - \vb*r')
\end{equation}
Für den Laplace Operator erfüllt die Funktion 
$\green=-\frac{1}{4\pi}\frac{1}{\abs{\vb*r-\vb*r'}}$ 
diese Bedingung. Diese 
Greensche Funktion ist aber nicht eindeutich definiert, denn die Funktion 
$\green
=-\frac{1}{4\pi}\frac{1}{\abs{\vb*r-\vb*r'}} + F(\vb*r,\vb*r')$ erfüllt die
Bedingung auch, falls $\Delta F(\vb*r, \vb*r')=0$

Setzt man nun $\psi=\green$ und $\varphi=\phi(\vr')$ in die die 2.
Greensche Identität ein, so folgt die 3. Greensche Identität (hier schon
als Spezialfal für das elektrische Potenzial)
\begin{equation}
  \begin{split}
    \int_V G(\vb*r,\vb*r')\underbrace{\Delta\phi(\vb*r')}_{%
    -4\pi k\rho(\vb*r')} 
    - \phi(\vb*r')\underbrace{\Delta \green}_{\delta(\vb*r - \vb*r')}
    dV'
    &= -4\pi k\int_V \rho(\vb*r') \green dV' -\phi(\vb*r)\\
    &= \oint_{\partial V} (\green\grad\phi(\vr')-\phi(\vr')\grad\green)\ddA'\\
  \end{split}
\end{equation}
Es folgt die Allgemeine Lösung für das Potenzial
\begin{equation}
  \label{eq:allgemeine-potential}
  \Leftrightarrow \phi(\vr) = -4\pi k\int_V \rho(\vr')\green dV' 
  - \oint_{\partial V} (\green\grad\phi(\vr')-\phi(\vr')\grad\green)\ddA
\end{equation}

Bemerke, dass falls $\phi$ und $\green$ im Unendlichen nach null abfallen,
mit $\green=-\frac{1}{4\pi} \frr $ einfach Funktion~\ref{eq:potential} folgt.
Dies ist also ein Spezialfall der allgemeine Lösung von $\phi(\vr)$, wo
keine Randbedingungen vorgegeben sind, bzw.\ wo die Randbedingungen im
Unendlichen liegen und deswegen verschwinden.

Ist nun auf die Randfläche $\phi$ vorgegeben (Dirichlet Randbedingungen), 
so wählt man $\green$ meistens so,
sodass $\green\big|_{\vr'\in\partial V}= 0$ gilt, also auf dem Rand 
verschwindet, und vereinfacht sich Funktion~\ref{eq:allgemeine-potential} zu
\begin{equation}
  \label{eq:dirichlet-green-potential}
  \phi(\vb*r') = -4\pi k\int_V \rho(\vr')\green[D] dV' 
  + \oint_{\partial V} \phi(\vr')\grad\green[D] \ddA
\end{equation}
wobei $\green[D]$ dann die obere Bedingung erfüllt, und manchmal
auch die Dirichlet-Greensche Funktion genannt wird.

Ist statt dem Potential das E-Feld oder eine Ladungsdichte vorgegeben 
(von Neumann Randbedingungen, man beachte dass $\sigma_F=\vb*n\cdot\vb E$,
sodass vorgabe von Flächenladungsdichte und E-feld auf dem Rand equivalent 
sind), so wählt man meistens dass $\grad\green\big|_{\vr\in\partial V}=0$,
sodass Funktion~\ref{eq:allgemeine-potential} sich veireinfacht zu
\begin{equation}
  \phi(\vb*r') = -4\pi k\int_V \rho(\vr')\green[N] dV' 
  + \oint_{\partial V} \green[N]\grad\phi(\vr') \ddA
\end{equation}
wobei $\green[N]$ dann die obere Bedingung erfüllt, und manchmal
auch die Von-Neumann-Greensche Funktion genannt wird.

\subsection{Praktische Lösungswege}%
\label{sub:spiegelladungsmethode}
In diesen Abschnitt werden wir uns mit Praktische Lösungsmethoden der
Poisson-Gleichung auseinandersetzen.

\subsubsection{Spiegelladungsmethode}%
Das Problem welches betrachtet wird, ist eine Ladung auf einen abstand a, z.B.
auf der z-Achse, vor eine Unendlich ausgedehnte Leiterschicht, z.B. in der 
x-y-Ebene. Allgemein gilt für Leiter, wo keinen strom durch fließt (was
in der Elektrostatik der Fall ist) dass das Potential überall in dem Leiter
konstant ist. Man muss also jetzt eine Lösung finden für dem die freie 
Ladungsdichte $\rho(\vb*r)=q\delta(\vr-\vr_0)$ und die Randbedingung 
$\phi(x,y,z=0)=\phi_0=\const$. 

Eine konstantes Potential in der x-y-Ebene
ist einfach zu realisieren, falls man eine 2. Ladung bei $\bar{\vr}_0=
(x_0,y_0,-z_0)$ und mit Ladung $-q$ legt. Das Potential oberhalb der 
Leiterebene erfüllt dann die Randbedingungen, jedoch ist aufzumerken, dass
es diese negative Ladung in der Wirklichkeit gar nicht gibt, sodass diese 
Lösung nur für eine Hälfte des Raumgebietes richtig ist. Die Greensche
Funktion zu diesem Problem ist $\green=-\frac{1}{4\pi} \qty[\frr-\frac{1}{\abs{\vr-\bar{\vr}}}]$, sodass man dieses Problem auch formeller mit 
Formel~\ref{eq:dirichlet-green-potential} lösen könnte.

Die Spiegelladungsmethode ist auch fur andere Geometrieen geignet, sowie
eine Punktladung vor eine metallische Kugel oder Zylinder, und die
Spiegelladung und ort der Spiegelladung sind immer entsprechend für das
Problem zu wählen.

\subsubsection{Seperationsansatz}%
Die Seperationsansatz ist einen Ansatz, wo man annimt, dass das Koordinaten
im Potential produktmäßich unabhänging sind, d.h.
\begin{equation*}
  \phi(u,v,w)=\phi_u(u)\phi_v(v)\phi_w(w)
\end{equation*}
Mit diesem Ansatz gelingt es in manche Fälle, die Multidimensionale 
Poissongleichung in 2 oder 3 unabhängige DGLs 2. Ordnung zu entkoppeln.

\subsubsection{Allgemeine Seperationsansatz in Kugelkoordinaten}%
Für Rotations- oder Kugelsymmetrische Probleme eignet es sich, eine 
Separationsansatz zu machen, wo 
$\phi(\vr)=\phi_r(r)\phi_\Omega(\theta,\varphi)$.
Der Laplace Operator muss dan auch entsprechend in Kugelkoordinaten definiert
werden. Es gilt 
\begin{equation}
  \Delta=\Delta_r + \frac{1}{r^2}\Delta_\Omega=
  \frac{1}{r^2}\del r \qty(r^2\del r) + 
  \frac{1}{r^2}\qty[\frac{1}{\sin\theta}\del\theta\qty(\sin\theta\del\theta)+
  \frac{1}{\sin^2\theta}\del\varphi^2] 
\end{equation}

Dabei sind die Kugelflächenfunktionen $\Y$ eine eigenfunktion der
Raumwinkel-Laplace-Operator $\Delta_\Omega$ wobei gilt dass
$\Delta_\Omega\Y=l(l+1)\Y$. Deswegen sind die Kugelflächenfunktionen einen
guten Wahl für unseren Ansatz. Weil die Kugelflächenfunktionen für alle
$l$ un $m$ Eigenfunktion sind, muss man die Superposition aller Lösungen
nehmen. Zu jeden $l$ und $m$ könnte nun aber eine unterschiedliche Funktion
$\phi_r(r)$ gehören, sodass die allgemeine Lösung gegeben wird durch
\begin{equation*}
  \phi(r,\theta,\varphi) = \sum_{lm} f_{lm}(r)\Y \qquad
  \l\in\mathbb{N}_0, m\in\{-l,-l+1,\ldots,l-1,l\}
\end{equation*}
Die Herleitung von $f_{lm}(r)$ wird hier zunächst übersprungen, ist aber
in viele Skripte und gute Literatur zu finden. Man findet am Ende die
allgemeine Lösung
\begin{equation}
  \phi(r,\theta,\varphi)=\sum_{l=0}^\infty\sum_{m=-l}^{m=l}
  \qty(a_{lm}r^l + b_{lm}r^{-l-1})\Y
\end{equation}

Sei das Problem nicht unabhängig von $\varphi$, so kann man auch stattdessen
die Lösung über die Legendre Polynome $P_l(\cos\theta)$ definieren 
(man beachte das $\Y\propto P_{lm}(\cos\theta)e^{im\phi}$, sodass die
$\varphi$ abhängigkeit für m=0 wegfällt) und findet für rotationssymetrische
Probleme den vereinfachten Ansatz
\begin{equation}
  \phi(r,\theta,\varphi)=\sum_{l=0}^\infty
  \qty(a_{l}r^l + b_{l}r^{-l-1})P_l(\cos\theta)
\end{equation}

Das lösen dieser Ansatze wird dann meistens durch Koeffizientenvergleich
mit den vorgegebenen Randbedingungen gemacht.
