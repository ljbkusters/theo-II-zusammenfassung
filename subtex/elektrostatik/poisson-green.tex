In diesem Kapitel besprechen wir wie man die Poisson gleichung lösen kann.

\subsection{die Poissongleichung}%
\label{ssub:poissongleichung}

Wie wir schon in Kapitel~\ref{ssub:Die-Feldgleichungen} hergeleitet haben
gilt in der Elektrostatik die \textbf{Poissongleichung}
\begin{equation*}
  \Delta \varphi (\vb*r) = -4\pi k \rho(\vb*r)
\end{equation*}
Die aufgabe besteht nun darein, $\varphi$ zu finden. Die Poissongleichung 
stellt eine partielle Differentialgleichung 2. Ordnung dar, die gelößt
werden kann unter vorgabe von Randbedingungen (analog zu Anfangsbedingungen
in der Mechanik). 

Praktische Lösungsmethoden sind meistens abhänging von die Geometrie
des zu lösenden Problems. Zunächst besprechen wir aber die Allgemeinste 
Lösung (jedoch meistens nicht schnellste oder einfachste Lösungsweg) der
Poissongleichung, die \textbf{Greensche Funktion}.

\subsection{Die Greensche Identitäte und die Greensche Funktion}%
\label{sub:green}

Für das Poissonproblem gibt es 3 wichtige Identitäten, die 3 Green'sche 
woraus man die allgemeine Lösung der Poissongleichung finden kann 
mittels die Greensche funktion. Zunächst betrachten wir die allgemeine
mathematische Identitäten und wenden sie danach Spezifisch für das Poisson
Problem an.

\subsubsection{1. Green'sche Identität}%
\label{ssub:green-id-1}
Gäbe es ein das Vektorfeld $\vb F = \psi \grad \varphi \equiv \psi \vb \Gamma$ 
mit $\psi$ und $\varphi$ beliebige Skalarfelder mit $\psi\in C^1$ und
$\varphi\in C^2$ (d.h. 1 bzw.\ 2 mal stetig diffbar) so folgt aus dem 
Satz von Gauß
\begin{equation}
  \int_V \underbrace{\qty(\psi\Delta\varphi + 
  \grad\psi\grad\varphi)}_{\div\vb F}dV =
  \oint_{\partial V} \qty(\psi\grad\varphi) \cdot d\vb A
\end{equation}
wobei man beachtet daß
\begin{equation*}
  \div \vb F =\div (\psi \vb \Gamma) = 
  \grad \psi \vb \Gamma + \psi\div\vb \Gamma = \grad \psi \grad \varphi + 
  \psi \Delta \varphi
\end{equation*}
Die erste Greensche Identität ist also einen Spezialfall des Gauß Gesetzes

\subsubsection{2. Green'sche Identität}%
\label{ssub:green-id-2}
Sei nun auch noch $\psi\in C^2$, und gäbe es ein weiteres $\epsilon\in C^1$
sodass man das Vektorfeld 
$\vb F=\psi(\epsilon\grad\psi) - \varphi(\epsilon\grad\psi)$ 
definiert folgt wieder
unter Anwendung der Satz von Gauß.

\begin{equation}
  \int_V \underbrace{\psi\div(\epsilon\grad\varphi) - \varphi\div(\epsilon\grad\psi)}_{\div \vb F} dV
  = \oint_{\partial V} \epsilon (\psi \grad \varphi - \varphi \grad \psi) 
  \cdot d\vb A
\end{equation}

wobei man beachtet daß
\begin{equation*}
  \div \vb F = \div (\psi (\epsilon\grad\varphi) - \varphi(\epsilon\grad\psi))
  = \cancel{(\grad\psi)(\epsilon\grad\varphi)} 
  + \psi\div(\epsilon\grad\varphi)
  - \cancel{(\grad\varphi)(\epsilon\grad\psi)}
  - \varphi\div(\epsilon\grad\psi)
\end{equation*}

Wir sind erstmal interessiert an dem Spezialfall wo $\epsilon=1$ ist, sodass
folgt
\begin{equation}
  \int_V \psi\Delta\varphi - \varphi\Delta\psi dV= 
  \oint_{\partial V} \psi\grad\varphi - \varphi\grad\psi d\vb A
\end{equation}

\subsubsection{3. Green'sche Identität}%
\label{ssub:green-id-3}

Man definiere nun die \textbf{Greensche Funktion} sodass gilt
\begin{equation}
  \Delta \green= \delta(\vb*r - \vb*r')
\end{equation}
Für den Laplace Operator $\Delta$ erfüllt die Funktion 
$\green=-\frac{1}{4\pi}\frac{1}{\abs{\vb*r-\vb*r'}}$ 
diese Bedingung. Die 
Green'sche Funktion ist aber nicht eindeutich definiert, denn die Funktion 
$\green
=-\frac{1}{4\pi}\frac{1}{\abs{\vb*r-\vb*r'}} + F(\vb*r,\vb*r')$ erfüllt die
Bedingung auch, falls $\Delta F(\vb*r, \vb*r')=0$

Setzt man nun $\psi=\green$ und $\varphi=\phi(\vr')$ in die die 2.
Greensche Identität ein, so folgt die 3. Greensche Identität (hier schon
als Spezialfal für das elektrische Potenzial)
\begin{equation}
  \begin{split}
    \int_V G(\vb*r,\vb*r')\underbrace{\Delta\phi(\vb*r')}_{%
    -4\pi k\rho(\vb*r')} 
    - \phi(\vb*r')\underbrace{\Delta \green}_{\delta(\vb*r - \vb*r')}
    dV'
    &= -4\pi k\int_V \rho(\vb*r') \green dV' -\phi(\vb*r)\\
    &= \oint_{\partial V} (\green\grad\phi(\vr')-\phi(\vr')\grad\green)\ddA'\\
  \end{split}
\end{equation}
Es folgt die Allgemeine Lösung für das Potenzial
\begin{equation*}
  \Leftrightarrow \phi(\vr) = -4\pi k\int_V \rho(\vr')\green dV' 
  - \oint_{\partial V} (\green\grad\phi(\vr')-\phi(\vr')\grad\green)\ddA
\end{equation*}

Bemerke, dass falls $\phi$ und $\green$ im Unendlichen nach null abfallen,
mit $\green=-\frac{1}{4\pi} \frr $ einfach Funktion~\ref{eq:potential} folgt.
Dies ist also ein Spezialfall der allgemeine Lösung von $\phi(\vr)$, wo
keine Randbedingungen vorgegeben sind, bzw.\ wo die Randbedingungen im
Unendlichen liegen und deswegen verschwinden.

Ist nun auf die Randfläche $\phi$ vorgegeben (Dirichlet Randbedingungen), 
so kann man $G$ so wählen, 
sodass $\green[D]\equiv\green\big|_{\vr'\in\partial V}= 0$ gilt, und vereinfacht sich 
die 3. Greensche Identität sich zu
\begin{equation}
  \phi(\vb*r') = -4\pi k\int_V \rho(\vr')\green[D] dV' 
  + \oint_{\partial V} \phi(\vr')\grad\green[D] \ddA
\end{equation}
vereinfacht
