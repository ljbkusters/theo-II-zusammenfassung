In diesem Kapitel überlegen wir uns, wie Elektrische Felder sich an 
\textbf{Randflächen} verhalten. Dabei ist es wichtig dass wir die aus den
Vektorkalkulus folgenden \textbf{Satz von Gauß} und \textbf{Satz von Stokes}
näher betrachten.

\subsubsection{Randflächen und Randkurven}%

Eine \textbf{Randfläche} ist ganz allgemein eine Fläche im Raum die Zwei 
Raumgebiete trennt, z.B. trennt die Randfläche einer Kugel das innere der
Kugel von dem Außenraum. 
Man kann das Vakuum unendlich viele triviale Randflächen
zuordnen. Was interessanter ist, sind Randflächen zwischen z.B. 
unterschiedliche Medien wie Vakuum, metallische Leiter, (un)geladene 
Dielektrika oder Ränder von Ladungsverteilungen. 

Eine \textbf{Randkurve} ist eine Kurve die eine Oberfläche abschließt. 

\subsubsection{Bedeutung der Sätze von Gauß und Stokes für Elektrostatische 
Felder}%
Zunächst wiederholen wir wie die Sätze mathematisch ganz 
allgemein aussehen.\\

\noindent
Sei $V\subset \mathbb{R}^n$ eine Kompakte Menge mit glattem Rand 
$\partial V$ mit ein nach äußeren orientierten Normaleneinheitsvektor 
$\vb*n$ bzw. Flächen Element $d\vb A =  \vb*n dA$. Sei ferner das Vektorfeld
$\vb F$ stetig differenzierbar auf einer offenen Menge $U$ 
mit $V\subseteq U$ so gilt
\begin{equation}
  \int_V \div \vb FdV  = \oint_{\partial V} \vb F \cdot d\vb A 
  \qquad \textrm{\textbf{Satz von Gauß}}
\end{equation}
Der Satz von Gauß besagt daß die Quellendichte (Divergenz) eines 
Vektorfeldes $\div \vb F$
integriert über ein Volumen $V$ proportional zur Flußintegrals 
des Feldes durch der Randfläche des Volumen $\partial V$.\\

\noindent
Sei $A\subset \mathbb{R}^n$ eine einfach zusammenhängende Fläche mit
glattem Randkurve $\partial A$ (die gegen dem Urzeigersinn durchlaufen 
wird beim Integrieren). Sei ferner das Vektorfeld
$\vb F$ stetig differenzierbar auf einer offenen Menge $U$ 
mit $A\subseteq U$ so gilt
\begin{equation}
  \int_A (\curl \vb F) \cdot d\vb A= \oint_{\partial A} \vb F \cdot d\vb*l 
  \qquad \textrm{\textbf{Satz von Stokes}}
\end{equation}
Der Satz von Stokes besagt daß das Flußintegral über die Wirbeldichte 
(Rotation) eines
Vektorfeldes $\curl \vb F$ über eine Fläche A proportional zur 
Kurvenintegral entlang die Randkurve der Fläche $\partial A$

In der Regel verhalten physikalische Felder sich immer sehr schön, sodass
wir uns über die Stetige Differentationsbedingung erstmal keine gedanken
machen müssen. Die Sätze von Gauß und Stokes schränken unsere zu betrachten
Volumina und Flächen zwar ein, aber nur sehr exotische Objekte erfüllen
die Bedingungen nicht, sodaß wir uns in der Regel auch keine Gedanken
machen müßen ob die mathematische Bedingungen erfüllt sind.

Zusammen mit die Feldgleichungen der Elektrostatik und die obere Sätze
Folgen direkt die Folgende Aussagen
\begin{equation}
  \begin{split}
    \oint_{\partial V} \vb E \cdot d\vb A 
    &
    =\int_V \underbrace{(\div \vb E)}_{4\pi k \rho(\vb*r)} dV
    =4\pi k Q_{V,\text{ges}}\\
    \oint_{\partial A} \vb E \cdot d\vb*l 
    &
    =\int_A \underbrace{(\curl \vb E)}_{0} \cdot d \vb A
    = 0
  \end{split}
\end{equation}

\subsubsection{Anwendung auf Randflächen}%
\label{ssub:anwendungen-auf-randflaechen}
Man kann nun eine allgemeine Randfläche im Raum betrachten. Legt man ein
kleines Kästchen (Volumen) auf der Rand, zentriert um dem Rand --- also ein
teil des Kästchens liegt an eine Seite der Randfläche, und ein Teil an 
der andere Seite --- und läßt man dieses Kästchen nun immer kleiner werden,
so betrachtet man annäherend zu die Randfläche selbst. Im limes von $V\to0$
findet man sogar lokale Punkte auf der Oberfläche und gilt
\begin{equation*}
  \lim_{V \to 0} \oint_{\partial V} \vb E \cdot 
  d\vb A 
  = \vb E \cdot \vb*n= 4\pi k \sigma(\vb*r)
\end{equation*}
Wobei $\sigma(\vb*r)$ die lokale Flächenladungsdichte auf der Randfläche 
ist.

Analog kann man eine infinitisimale Flächenstückchen 
--- die orthogonal auf der 
Randfläche steht, mit ein Teil an einer Seite der Randfläche und ein Teil
auf der andere Seite --- um die Randfläche zentrieren. Im limes von 
$A\to0$ findet man
\begin{equation*}
    \lim_{A \to 0} \oint_{\partial A} \vb E \cdot d\vb*l 
    = \vb E \times \vb*n
    = 0
\end{equation*}

Es folgen die \textbf{Randflächenbedingungen}
\begin{equation}
  \begin{split}
    \vb E \cdot \vb*n &= 4\pi k \sigma(\vb*r) \\
    \vb E \times \vb*n &= 0
  \end{split}
\end{equation}
Dies bedeutet weiter, daß die orthogonalkomponente 
$\vb E_{\perp}= \vb E\cdot \vb*n$ (in der Anwesendheit einer 
Flächenladungsdichte) einen Sprung um $4\pi k \sigma(\vb*r)$ macht, während
die tangentialkomponente $\vb E_{\parallel}=\vb E \times \vb*n$ immer
stetig ist.
