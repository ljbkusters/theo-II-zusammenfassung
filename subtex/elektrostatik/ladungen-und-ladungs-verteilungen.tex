Die Ursprung der EM-Kraft und EM-Felder ist die \textbf{Ladung}. Mathematisch wird die
Ladungsverteilung eines Systems in Raum durch die \textbf{Ladungsdichte} 
gegeben. Ganz algemein
kann eine Ladungsverteilung von N Punktladungen am Zeit t an den Orten $\vb*x_i$ wie folgt beschrieben werden

\begin{equation}
  \rho(\vb*r, t)= \sum_{i=1}^N q_i \delta(\vb*r - \vb*x_i(t))
\end{equation}
Wobei $q_i$ sowohl positiv oder negativ sein kann.

Prinzipiell ist die Ladung in der Natur eine Eigenschaft von Punktteilchen,
sodass es eigentlich keine kontinuierliche Ladungsverteilungen gibt. Auf
makroskopische Skala ist diese Beschreibung jedoch sehr unhandlich, und ist
eine Mittelung der Ladungsdichte eine wesentlich einfachere Beschreibung.
Eine mögliche Definition für eine kontinuierliche Ladungsverteilung aus einer
diskreten Ladungsverteilung sieht wie folgt aus
\begin{equation}
  \rho_{\textrm{knt.}}(\vb*r, t) 
  = 
  \frac{1}{\Delta V(\vb*r)} 
  \int_{\Delta V(\vr)}d^3r'\rho_{\text{dskr}}(\vr')
  = 
  \frac{1}{\Delta V(\vb*r)} 
  \sum_{q_i\in\Delta V(\vr)}q_i
\end{equation}
wobei man das Volumenkästchen $\Delta V$ um $\vr$ zentriert kontinuierlich verschieben kann, und annimt, daß es viele Ladungen gibt, sodaß die Kontinuierliche Ladungsverteilung praktisch glatt wird.

Für Physikalische Systeme betrachtet man in der Regel immer nur beschränkte 
Ladungsverteilungen, das will sagen, die Ladungsdichte soll nur auf 
ein beschränktes Raumbereich von 0 verschieden sein. 
Dazu dürfen Ladungsverteilungen auch keine Pollstellen haben, 
d.h.\ nie divergieren. 
In manche theoretische Fälle, gibt es auch Ladungsverteilungen die 
nicht beschränkt sind, sondern hinreichend schnell im Unendlichen 
verschwinden (z.B.\ exponentiell unterdrückt), sodass zumindest
\begin{equation*}
  \int d^3r \rho(\vb*r, t) = Q_\textrm{ges} < \infty
\end{equation*}
gilt.

In der Elektrostatik betrachtet man erstmal ruhende Ladungsverteilungen, 
d.h.
\begin{equation}
  \rho(\vb*r, t) = \rho(\vb*r)
\end{equation}
Es gibt also keine Ströme. Manchmal werden in der Elektrostatik auch 
quasi statische Prozeße betrachtet, wo die Effekte von bewegte Ladungen 
erstmal vernachlässicht werden. Die Elektrostatik ist somit ein Spezialfall
der Elektrodynamik.
