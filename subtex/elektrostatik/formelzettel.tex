\subsection{Ladungen und Ladungsverteilungen}%
\begin{equation*}
  \rho(\vr, t) = \rho(\vr) \quad (\text{Elektrostatik})
\end{equation*}

\subsection{Elektrische Kraft, Elektrisches Feld und Elektrisches Potential}%
\begin{multicols}{2}
\noindent
Coulombkraft einer Punktladung im äußeren E-Feld
\begin{equation*}
  \vb F_c(\vr) = q\vb E(\vr)
\end{equation*}

\noindent
Zusammenhang E-Feld und Elektrisches Potential
\begin{equation*}
  \vb E = -\grad\phi(\vr)
\end{equation*}

\noindent
Berechnung des E-Feld und Potential aus Ladungsverteilung 
\begin{equation*}
  \begin{split}
    \phi(\vr)&=\int d^3r' \frac{\rho(\vr')}{\rr}\\
    \vb E(\vr)&=\int d^3r' \rho(\vr')\frac{(\vr-\vr')}{\rr^3}\\
  \end{split}
\end{equation*}

\noindent
Feldgleichungen der E-Statik
\begin{equation*}
  \begin{split}
    \div\vb E &= 4\pi k \rho(\vr)\\ 
    \curl\vb E &= 0\\ 
  \end{split}
\end{equation*}

\noindent
Poisson-Gleichung
\begin{equation*}
  \Delta\phi(\vr) = -4\pi k \rho(\vr)
\end{equation*}
\end{multicols}

\subsection{Multipolentwicklung}%

\subsection{Verhalten von Elektrostatische Felder an Randflächen}%

\subsection{Lösung der Poisson Gleichung und Greensche Funktion}%

\subsection{Elektrostatische Energie}%

\subsection{Elektrostatik in Materie}%
