\subsection{Ladungen und Ladungsverteilungen}%
\begin{equation*}
  \rho(\vr, t) = \rho(\vr) \quad (\text{Elektrostatik})
\end{equation*}

\subsection{Elektrische Kraft, Elektrisches Feld und Elektrisches Potential}%
\begin{multicols}{2}
  \noindent
  \begin{tabular}{p{4.5cm}p{5cm}}
    Coulombkraft Punktladung im äußeren E-Feld
    & 
    $\displaystyle \vb F_c(\vr) = q\vb E(\vr)$ 
    \\\\
    Zusammenhang E-Feld und Pot.
    &
    $\displaystyle \vb E = -\grad\phi(\vr)$
    \\\\
    Poisson-Gleichung
    & 
    \(\displaystyle
      \Delta\phi(\vr) = -4\pi k \rho(\vr)
    \)\\\\
    Feldgleichungen der E-Statik
    & 
    \(\displaystyle
    \div\vb E = 4\pi k \rho(\vr)
    \)\newline
    \(\displaystyle
      \curl\vb E = 0
    \) \\\\
    Konstante $k$
    &
    \(\displaystyle
      k=\frac{1}{4\pi \epsilon_0}\quad (\text{SI}) \newline
      k=1\quad (\text{Gauß})
    \)
  \end{tabular}
  
  \noindent
  \begin{tabular}{p{3cm}p{5cm}}
    Pot.\ einer Ladungsverteilung*
    &
    \(\displaystyle
      \phi(\vr)=k\int d^3r' \frac{\rho(\vr')}{\rr}
    \)
    \\\\
    E-Feld einer Ladungsverteilung*
    &
    \(\displaystyle
        \vb E(\vr)=k\int d^3r' \rho(\vr')\frac{(\vr-\vr')}{\rr^3}
    \)\\\\
    Pot.\ einer Punktladung*
    &
    \(\displaystyle
    \phi(\vr) = k \frac{Q}{\abs{\vr-\vr_0}} 
    \)\\\\
    E-Feld einer Punktladung*
    &
    \(\displaystyle
    \vb E(\vr) = kQ \frac{\vr-\vr_0}{\abs{\vr-\vr_0}^3} 
    \)\\\\
  \end{tabular}
  * (Bei verschwindende Randbedingungen)
\end{multicols}
\subsection{Multipolentwicklung}%
  \noindent
  \begin{tabular}{p{4.5cm}p{12cm}}
    Grundlegende Entwicklung (Karthesisch)
    & 
    \(\displaystyle
      \frac{1}{\rr}\bigg|_{\vr\gg\vr'} 
      \approx \frac{1}{r} + \frac{\vr'\cdot\vr}{r^3} + \frac{1}{2}
      \sum_{i,j=1}^{3} \frac{3r_i'r_j'-\delta_{ij}\vr'^2}{r^5}r_ir_j 
    \)\\\\
    Karthesische Multipolentwicklung
    & 
    \(\displaystyle
    \phi(\vr)
    \approx k\frac{Q}{r} + k\frac{\vb*p\cdot\vr}{r^3} + \frac{k}{2}
    \sum_{i,j=1}^{3} \frac{Q_{ij}}{r^5}r_ir_j 
    \)\\\\
    Karthesische Multipolmomente
    &
    \(\displaystyle
      Q = \int d^3r' \rho(r')
    \)\newline
    \(\displaystyle
      \vb*p = \int d^3r'\vr' \rho(r')
    \)\newline
    \(\displaystyle
      Q_{ij} = \int d^3r'(3r_i'r_j'-\delta_{ij}\vr'^2) \rho(r')
    \)\\\\
    Grundlegende Entwicklung (Kugelfl.fkt.)
    &
    \(\displaystyle
    \frac{1}{\abs{\vb*r-\vb*r'}}
    =
    \sum_{lm} \frac{4\pi}{2l+1} \frac{r_<^l}{r_>^{l+1}}
    Y^*_{lm}(\Omega')Y_{lm}(\Omega)\newline r_>=\max(r,r')\ r_<=\min(r,r')
    \)\\\\
    Kugelfl. Entwicklung Nah am Ursprung
    &
    \(\displaystyle
    \phi(\vb*r) 
    = k\sum_{lm} \frac{4\pi}{2l+1}r^l q_<^{lm} Y_{lm}(\Omega) 
    \quad q_<^{lm}=\int_0^\infty dr' {r'}^{1-l}
  \int_\Omega d\Omega' Y^*_{lm}(\Omega')\rho(\vb*r')
    \)\\
    Kugelfl. Entwicklung Weit weg vom Ursprung
    &
    \(\displaystyle
    \phi(\vb*r) 
    = k\sum_{lm} \frac{4\pi}{2l+1} 
    \frac{q_>^{lm}}{r^{l+1}} Y_{lm}(\Omega)\quad
    q_>^{lm}=\int_0^\infty dr' {r'}^{l+2}
    \int_\Omega d\Omega' Y^*_{lm}(\Omega')\rho(\vb*r')
    \)\\\\
  \end{tabular}
   
  \noindent
  

\subsection{Verhalten von Elektrostatische Felder an Randflächen}%

\subsection{Lösung der Poisson Gleichung und Greensche Funktion}%

\subsection{Elektrostatische Energie}%

\subsection{Elektrostatik in Materie}%
