Zunächst besprechen wir wie man analytische Näherungen finden kann
unter angabe von komplexere Ladungsverteilungen. 
Das Problem ist meistens das Integrieren der
\begin{equation*}
  \frac{1}{\abs{\vb*r-\vb*r'}}
\end{equation*}
Term (in Kombination mit die Ladungsdichte $\rho(\vb*r')$). 
Deswegen möchten wir in der \textbf{karthesische Multipolentwicklung} 
diesen  Term Taylor-entwickln, damit wir es in eine 
Reihe von Polinomiale Terme umwandeln können, 
weil diese Einfach(er) zu integrieren ist. 
Daneben gibt es noch die \textbf{Kugelfächen-Entwicklung} die sich 
insbesondere für Radial- bzw. Rotationssymmetrische Probleme eignet, 
die wir Später besprechen werden. 

\subsection{Karthesische Multipolentwicklung}%
\label{ssub:Karthesische-Multipolentwicklung}
Die karthesische Multipolentwicklung wird mittels eine Taylor-Entwicklung
hergeleitet. Man muss dabei eine Multidimensionale Taylor-Entwicklung 
durchführen.

Eine allgemeine mehrdimensionale Taylorentwicklung wird gegeben durch:
\begin{equation*}
  f(\vb*\alpha, \vb*\beta)
  =\qty(\exp(\vb*\beta\cdot\nabla_{\vb*\beta'})
  f(\vb*\alpha, \vb*\beta'))
  \bigg|_{\vb*\beta'=\vb*\beta_0}
  =\qty(\sum_{n=0}^\infty
  \frac{(\vb*\beta\cdot\nabla_{\vb*\beta'})^n}{n!}
  f(\vb*\alpha, \vb*\beta'))
  \bigg|_{\vb*\beta'=\vb*\beta_0}
\end{equation*}
Man bemerke dass $\nabla$ einen Operator ist! Die exponential Funktion
dient hier nur zur vereinfachung der Darstellung! Man Taylore nun um $\vb*r'=0$ (dies heißt, daß $\vb*r\gg\vb*r'$ sodass 
$\vb*r-\vb*r'\approx\vb*r$, dafür muss $\vb*r$ natürlich weit
von der Quelle entfernt sein). Die Multipolentwicklung ist also ein 
\textbf{Fernfeldnäherung}.

Man definiere nun $f(\vb*r,\vb*r')=f(r_1,r_2,r_3,r_1',r_2',r_3')
\equiv\frac{1}{\abs{\vb*r-\vb*r'}}$ sodass bis zur 2. Ordnung die Taylorentwicklung für unsere Funktion
wie folgt aussieht:

\begin{equation*}
  \frac{1}{\abs{\vb*r-\vb*r'}}=f(\vb*r, 0)
  + (\vb*r'\cdot \nabla_{\vb*{\bar{r}'}}) f(\vb*r, \vb*{\bar{r}}')\bigg|_{\vb*{\bar{r}}'=0}
  + \frac{1}{2}(\vb*r'\cdot\nabla_{\vb*{\bar r'}})^2
  f(\vb*r,\vb*{\bar r}')
  \bigg|_{\vb*{\bar r'}=0}
  + \ldots
\end{equation*}

\underline{0. Ordnung:}
\begin{equation*}
  \frac{1}{\abs{\vb*r-\vb*{\bar r'}}}\bigg|_{\vb*{\bar r'}=0} 
  = \frac{1}{\abs{\vb*r}}= \frac{1}{r} 
\end{equation*}

\underline{1. Ordnung:}
\begin{equation*}
  \vb*{r'}\cdot
  \nabla_{\vb*{\bar{r}'}}\frac{1}{\abs{\vb*r-\vb*{\bar{r}'}}}
  \bigg|_{\vb*{\bar r'}=0}
  = \frac{\vb*{ r'}\cdot\vb*r}{r^3} 
\end{equation*}

\underline{2. Ordnung:}
\begin{equation*}
  \begin{split}
  \frac{1}{2}(\vb*{\bar r'}\cdot\nabla_{\vb*{\bar{r}'}})^2
  \frac{1}{\abs{\vb*r-\vb*{\bar r'}}}\bigg|_{\vb*{\bar r'}=0}
  &=
  \frac{1}{2}\qty[r_i'\pdv {\bar r_i'}]\qty[r_j'\pdv{\bar r_j'}]
  \frac{1}{\abs{\vb*r-\vb*{\bar r'}}}
  \bigg|_{\vb*{\bar r'}=0}\\
  &=
  \frac{r'_ir'_j}{2} 
  \frac{\partial^2}{\partial_{\bar r_i'}\partial_{\bar r_j'}} 
  \frac{1}{\abs{\vb*r-\vb*{\bar r'}}}\bigg|_{\vb*{\bar r'}=0}\\
  &=
  \frac{r'_ir'_j}{2} \frac{\partial}{\partial_{\bar r_i'}} 
  \frac{x_j}{\abs{\vb*r-\vb*{\bar r'}}^3}\bigg|_{\vb*{\bar r'}=0}
  \qquad x_j\equiv (r_j-\bar r_j')\\
  &=
  \frac{r'_ir'_j}{2} \frac{\partial}{\partial_{\bar r_i'}} 
  \frac{x_j}{\abs{\vb*r-\vb*{\bar r'}}^3}\bigg|_{\vb*{\bar r'}=0}\\
  &=
  \frac{r'_ir'_j}{2} 
  \qty(
    \text{-}\frac{3}{2}
    \frac{\text{-}2x_i x_j}{\abs{\vb*r-\vb*{\bar r'}}^5}
    -\frac{\delta_{ij}}{\abs{\vb*r-\vb*{\bar r'}}^3} 
  )\bigg|_{\vb*{\bar r'}=0}\quad\text{(Produktregel)}\\
  &=
  \frac{1}{2}\sum_{i,j=1}^3 \frac{3r_ir_j-\delta_{ij}\vb*r^2}{r^5}
  r_i'r_j'
  \end{split}
\end{equation*}
\textit{Eine wichtige Bemerkung:}
Aus Symmetrie Grunden gilt
\begin{equation*}
  \frac{1}{2}\sum_{i,j=1}^3 \frac{3r_ir_j-\delta_{ij}\vb*r^2}{r^5}
  r_i'r_j'
  =
  \frac{1}{2}\sum_{i,j=1}^3 \frac{3r'_ir'_j-\delta_{ij}\vb*r'^2}{r^5}
  r_ir_j
\end{equation*}
Wir benutzen im allgemeinen die letzte Definition wenn wir die
elektrische und magnetische Multipole berechnen. Man findet also bis zum
2. Ordnung
\begin{equation}
  \frac{1}{\abs{\vb*r-\vb*r'}}
  \approx 
  \frac{1}{r} 
  + \frac{\vb*r'\cdot\vb*r}{r^3} 
  + \frac{1}{2}\sum_{i,j=1}^3 
    \frac{3r'_ir'_j-\delta_{ij}\vb*r'^2}{r^5}r_ir_j
\end{equation}
Setzt man dies in die Definition für das Elektrische Potential ein, so 
findet man
\begin{equation*}
  \begin{split}
    \phi(\vb*r) 
    &= k \int d^3 r' \frac{\rho(\vb*r')}{\abs{\vb*r - \vb*r'}} \\
    & \approx k\int d^3 r' \rho(\vb*r')
    \qty(
    \frac{1}{r} 
    + \frac{\vb*r'\cdot\vb*r}{r^3} 
    + \frac{1}{2}\sum_{i,j=1}^3 
    \frac{3r'_ir'_j-\delta_{ij}\vb*r'^2}{r^5}r_ir_j)\\
    &\equiv 
    k \frac{Q}{r} + k \frac{\vb*p\cdot\vb*r}{r^3} 
    + \frac{k}{2} \sum_{i,j=1}^{3} \frac{r_ir_j}{r^5}Q_{ij}
  \end{split}
\end{equation*}

\begin{center}
\begin{tabular}{ll}
  Monopol:    & $\ds Q=\int d^3r\rho(\vb* r)$ \quad\text(Gesammtladung)\\
  Dipol:      & $\ds p_i=\int d^3r\rho(\vb* r)r_i
                \quad\vb* p=p_i\vu{e}_i $\\
  Quadrupol:  & $\ds Q_{ij}=\int d^3r\rho(\vb* r)
  \qty(3r_ir_j - \delta_{ij}\abs{\vb* r}^2)$
\end{tabular}
\end{center}
Dabei hat die Quadrupoltensor $Q_{ij}$ nur 5 Freiheitsgraden. 
Mit nur 5 Rechnungen alle (9) Quadrupol Elemente berechnen. 
Es gilt zwar $Q_{ij}=Q_{ji}$ (symmertrisch) und $\text{sp}(\bm Q)=\sum_i Q_{ii}=0$ (spurfrei).

Im allgemeinen berechnet 
man keine weitere Ordnungen analytisch im Bachelorstudium, 
und zum Verständniss bringt dies auch nicht mehr (außer ärger), 
sodass höhere Ordnungen berechnen 
ein Problem ist das man lieber an Computer überlässt.

\subsection{Sphärische Multipolentwicklung}%
\label{ssub:sphaerische-Multipolentwicklung}
Für Ladungsverteilungen die Radial- oder Rotationssymetrisch sind ist die 
Sphärische Multipolentwicklung besonders geeignet, vor allem falls man die 
Ladungsdichte als Linearkombination von Kugelfächen-Funktionen schreiben 
kann. Um die Sphärische Multipolentwicklung zu motivieren machen wir 
zunächst die Annahme (mit dem Seperationsansatz), dass das 
Winkelanteil des Potentials unabhähngig vom Radialanteil ist, also
\begin{equation*}
  \phi(\vb*r) = \phi_r(r)\phi_\Omega(\theta,\varphi)
\end{equation*}
Weil man (wie wir später besprechen werden) die Poisson-Gleichung mit dem
gleichen Ansatz lösen wird, und die Kugelflächen-Funktionen 
$Y_{lm}(\theta,\varphi)$ eine Eigenfunktion der Laplace Operator ist, 
ist es eine gute Idee um dies als Basis für eine Sphärische Entwicklung 
zu benutzen. Dazu formen die Kugelflächen-Funktionen unter Integration über
$d\Omega=\sin\theta d\theta d\varphi$ eine orthonormale Basis. Es gilt die untere Zusammenhang.
\begin{equation*}
  \int d\Omega Y_{lm}(\Omega)Y^*_{l'm'}(\Omega)=\delta_{ll'}\delta_{mm'}
\end{equation*}

Die Entwicklung in Kugelkoordinaten ist rechnerisch etwas aufwändig, und
wird hier erstmal übersprungen, aber sie wird in gute Literatur über
die Elektrodynamik oft gegeben, wie z.B. in Kapitel 2.3.8 von Noltings
``Grundkurs Theoretische Physik 3, Elektrodynamik''. 
Man findet 
\begin{equation}
  \frac{1}{\abs{\vb*r-\vb*r'}}
  =
  \sum_{lm} \frac{4\pi}{2l+1} \frac{r_<^l}{r_>^{l+1}}
  Y^*_{lm}(\Omega')Y_{lm}(\Omega)
\end{equation}
\begin{center}
mit $l=0,1,2,\ldots$ und $m=-l,\ldots,l$\\
$r_<=\min(r,r')$, $r_>=\max(r,r')$.
\end{center}
Setzt man dies in der Definition für das Elektrische Potential ein, 
so findet man
\begin{equation}
  \begin{split}
    \phi(\vb*r) 
    &= k\int d^3r' \rho(\vb*r') \frac{1}{\abs{\vb*r-\vb*r'}} \\
    &= k\sum_{lm} \frac{4\pi}{2l+1} 
    \int_0^\infty dr'{r'}^2 \frac{r_<^l}{r_>^{l+1}} 
    \int_\Omega d\Omega' Y^*_{lm}(\Omega')Y_{lm}(\Omega) \rho(\vb*r')\\
    &= k\sum_{lm} \frac{4\pi}{2l+1} Y_{lm}(\Omega) 
    \qty(
    \int_0^{r} dr'{r'}^2 \frac{{r'}^l}{r^{l+1}} +
    \int_{r}^{\infty} dr'{r'}^2 \frac{r^l}{{r'}^{l+1}} 
    )
    \int_\Omega d\Omega' Y^*_{lm}(\Omega')\rho(\vb*r')\\
  \end{split}
\end{equation}
Sei nun die Ladungsdichte 
$\rho(\vb*r')=\sum_{l'm'}f_{f'm'}(r')Y_{l'm'}(\Omega')$
für beliebige $l'$ und $m'$ (also einfach irgendeine Linearkombination von
beliebige Kugelflächenfunktionen) so vereinfacht sich das Problem weiter,
und kann man für eine Endliche Summe sogar exakte Lösungen finden.
(Es ist nicht notwendig dass man die Ladungsdichte als Linearkombination
von Kugelflächenfunktionen schreibt, nur einfacher).
\begin{equation}
  \begin{split}
    \phi(\vb*r) 
    &= k\sum_{lm} \frac{4\pi}{2l+1} Y_{lm}(\Omega) \sum_{l'm'}
    \qty(
    \int_0^{r} dr'\frac{{r'}^(l+2)}{r^{l+1}}f_{l'm'}(r) +
    \int_{r}^{\infty} dr'\frac{r^l}{{r'}^{l-1}}f_{l'm'}(r) 
    )
    \int_\Omega d\Omega' Y^*_{lm}(\Omega')Y_{l'm'}(\Omega')\\
    &= k\sum_{lm} \frac{4\pi}{2l+1} Y_{lm}(\Omega) \sum_{l'm'}
    \qty(
    \int_0^{r} dr'\frac{{r'}^{l+2}}{r^{l+1}}f_{l'm'}(r) +
    \int_{r}^{\infty} dr'\frac{r^l}{{r'}^{l-1}}f_{l'm'}(r) 
    )
    \delta_{ll'}\delta_{mm'}\\
    &= k\sum_{l'm'} \frac{4\pi}{2l+1} Y_{l'm'}(\Omega)
    \qty(
    \int_0^{r} dr'\frac{{r'}^{l+2}}{r^{l+1}}f_{l'm'}(r) +
    \int_{r}^{\infty} dr'\frac{r^l}{{r'}^{l-1}}f_{l'm'}(r) 
    )
  \end{split}
\end{equation}
d.h.\ alle Terme wo $l\neq l'$ oder $m\neq m'$ fallen wegen dem Kronecker-delta weg. Wir werden aber weiter wieder von eine allgemeine Ladungsdichte 
ausgehen.

Ist die Ladungsdichte nun nach Innen oder Nach außen beschränkt, dann kann
man $r_>$ und $r_<$ eindeutig als $r$ oder $r'$ festlegen in bestimmte 
Raumbereiche. Hat man z.B. eine geladene Kugelschale wobei die 
Ladungsdichte, noch von $\theta$ und $\varphi$ abhängen darf, so kann man
das Potential im Inneren mit nur dem $\int_r^\infty dr$ Integral beschreiben, und
dem Außenraum mit durch das $\int_0^r dr$ Integral. Es macht manchmal 
sogar kein Sinn die Lösungen der beide Raumbereiche gleichzeitig zu 
berechnen weil die Lösung für das Innere Raumbereich im äußeren Raumbereich
divergiren kann oder umgekehrt. Wir Teilen hier also zunächst die 
Raumgebiete auf und finden Multipolmomente für dem innen und außen Räume 
$q_<^{lm}$ und $q_>^{lm}$\\

\noindent
Im Innenraum ($r\approx 0$):
\begin{equation}
  \begin{split}
    \phi(\vb*r) 
    &= k\sum_{lm} \frac{4\pi}{2l+1} Y_{lm}(\Omega)
    \int_0^{r} dr'\frac{{r}^{l}}{{r'}^{l-1}}
    \int_\Omega d\Omega'Y^*_{lm}(\Omega')\rho(\vb*r')\\
    &= k\sum_{lm} \frac{4\pi}{2l+1}r^l q_<^{lm} Y_{lm}(\Omega) 
  \end{split}
\end{equation}
\begin{center}
  mit $\ds q_<^{lm}=\int_0^\infty dr' {r'}^{1-l}
  \int_\Omega d\Omega' Y^*_{lm}(\Omega')\rho(\vb*r')$
\end{center}
\noindent
Im Außenraum ($r\gg0$):
\begin{equation}
  \begin{split}
    \phi(\vb*r) 
    &= k\sum_{lm} \frac{4\pi}{2l+1} Y_{lm}(\Omega)
    \int_0^{r} dr'\frac{{r'}^{l+2}}{r^{l+1}}
    \int_\Omega d\Omega'Y^*_{lm}(\Omega')\rho(\vb*r')\\
    &= k\sum_{lm} \frac{4\pi}{2l+1} 
    \frac{q_>^{lm}}{r^{l+1}} Y_{lm}(\Omega) 
  \end{split}
\end{equation}
\begin{center}
  mit $\ds q_>^{lm}=\int_0^\infty dr' {r'}^{l+2}
  \int_\Omega d\Omega' Y^*_{lm}(\Omega')\rho(\vb*r')$
\end{center}
