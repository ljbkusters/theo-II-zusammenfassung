Die Dirac $\delta$-Distribution (manchmal auch $\delta$-Funktion genannt) ist
eine normierte Distribution, die es erschafft, mittels der Kalkulus 
infinitisimale Punkte oder ininitisimal dünne Flächen im Raum zu beschreiben.
Für die Elektrodynamik ist dies wichtig, denn die Ladung ist eine eigenschaft
von Punktteilchen. Auch ist die Delta distribution sehr nützlich um 
Flächenladungen zu beschreiben.

\subsection{Definierende Eigenschaft}%
\label{sub:definierende-eigenschaft}
Die $\delta$-Distribution muss die folgende Bedingungen erfüllen
\begin{equation}
  \begin{split}
    \int_{\mathbb{R}^3}d^3r\delta(\vr-\vr_0) &= 1 \\
    \delta(\vr - \vr_0) &= 
    \begin{cases}
      \infty & \vr=\vr_0\\
      0 & \mathrm{sonst}
    \end{cases}
  \end{split}
\end{equation}
Wobei die untere Zusammenhang vor allem andeutet dass die 
$\delta$-Distribution um $\vr_0$ zentriert ist und unendlich spitz ist.
Es folgt für diese Distribution mit eine beliebige Testfunktion $f(\vr)$.
\begin{equation}
    \int_V d^3r \delta(\vr -\vr_0)f(\vr) = 
    \begin{cases}
      f(\vr_0) & \vr_0\in V\\
      0 & \mathrm{sonst}
    \end{cases}
\end{equation}

\subsection{Herleitung $\delta$-Distribution aus Bekannte Kurven}%
\label{sub:herleitung-delta-distribution}
Die Dirac $\delta$-Funktion kann aus mehrere normierbare Kurven hergeleitet
werden. Wichtig ist die Eigenschaft, dass die Funktion im limes einer
Parameter stehts spitzer wird, aber trotzdem normiert bleit, sodass sich
die Distribution immer mehr auf einem Punkt anhäuft. 
Einige Beispiele solcher Funktionen sind
\begin{enumerate}
  \item \textbf{Kleine Stufe}
    \begin{equation}
      \delta(x - x_0) = \lim_{b \to 0} 
      \begin{cases}
        \frac{1}{b} &x\in[x_0 - \frac{b}{2}, x_0 + \frac{b}{2}] \\
        0 & \textrm{sonst}
      \end{cases}
    \end{equation}
  \item \textbf{Gauß Kurve} 
    \begin{equation}
      \delta(x - x_0) = \lim_{\sigma \to 0} 
      \frac{1}{\sqrt{2\pi}\sigma} e^{\frac{(x-x_0)^2}{2\sigma^2}}
    \end{equation}
  \item \textbf{Lorentz Kurve}
    \begin{equation}
      \delta(x - x_0) = \lim_{\eta \to 0} \frac{1}{\pi} 
      \frac{\eta}{\eta^2 + (x-x_0)^2} 
    \end{equation}
\end{enumerate}
Diese Funktionen werden in der Praxis aber nie direkt angewandt, sind
also nur für die Vorstellung gedacht.

\subsection{Einige Wichtige Identitäten}%
\begin{enumerate}
  \item Fundamentaler Eigenschaft
  \begin{equation*}
    \int d^3 f(\vr)\delta(\vr-\vr_0)=f(\vr_0)
  \end{equation*}

  \item Ableitung der $\delta$-Funktion
    \begin{equation*}
      \delta'(\vr-\vr_0)f(\vr) = \delta(\vr-\vr_0)f'(\vr)
    \end{equation*}

  \item Integration der $\delta$-Funktion (Stufenfunktion)
    \begin{equation*}
      \int_{-\infty}^xdx'\delta(x'-x_0) \equiv \Theta(x-x_0) =
      \begin{cases}
        1 & x > x_0 \\
        0 & x < x_0
      \end{cases}
      \quad \Leftrightarrow\quad \dv{x} \Theta(x-x_0) = \delta(x-x_0)
    \end{equation*}

  \item $\delta$-Funktion einer andere Funktion
    \begin{equation*}
      \delta(g(\vr)) = \sum_{\vr_i} \frac{1}{\abs{g(\vr_i)}}
      \delta(\vr-\vr_i) 
      \quad \text{mit}\ \vr_i\ \text{ Nullstellen von } g(x)
    \end{equation*}

  \item Multidimensionale $\delta$-Funktion Zerlegung in 1-D 
    $\delta$-Funktionen (Karthesisch und Krummlienig)
     \begin{equation*}
       \delta(\vr(x,y,z)) 
       = \delta(x)\delta(y)\delta(z) 
       \quad \delta(\vr(u,v,w)) 
       = \frac{1}{\gamma(u,v,w)}\delta(u)\delta(v)\delta(w) 
     \end{equation*}
     Wobei $\gamma$ die Funktionaldeterminante ist
\end{enumerate}
