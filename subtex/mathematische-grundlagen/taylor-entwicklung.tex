Die Taylorentwicklung ist eine Näherungsmethode wobei man Funktionen in 
Polynome zerlegt.

Allgeimeine Taylorentwicklung (Kompakte Darstellung):
\begin{equation}
  f(x) = e^{(x - x_0)\cdot D_{x'}}f(x')\big|_{x'=x_0}
\end{equation}
Wobei $D_{x'}$ eine allgemeine Ableitungsoperator ist. In eine Dimension
folgt für $D_{x'}=\dv{x'}$ die bekannte Zusammenhang.
\begin{equation}
  f(x) = e^{(x-x_0)\dv{x'}}f(x')\big|_{x'=x_0} = 
  \sum_{i=0}^{\infty} \frac{f^{(n)}(x_0)}{n!}{(x-x_0)}^n
\end{equation}

Eine Mehrdimensionale Taylorentwicklung wird analog gegeben wobei 
$D_{x'}=\nabla_{x'}$ und $x\to\vb*x = \vb*x(x_1, x_2, \ldots, x_n)$ und
$\vb*x_0\to\vb*x_0(x_{01},\ldots,x_{0n})$ 
\begin{equation}
  f(x) = e^{(\vb*x-\vb*x_0)\nabla_{\vb*x'}}f(\vb*x')\big|_{x'=x_0} = 
  \sum_{i=0}^{\infty} \frac{\nabla_{\vb*x'}^nf(\vb*x')}{n!}
  {(\vb*x-\vb*x_0)}^n\bigg|_{\vb*x'=\vb*x_0}
\end{equation}

Man kann gegebenenfalls auch um nur ein Teil der Variabelen 
$\{x_i,\ldots,x_j\}\subset\{x_1, \ldots, x_n\}$ entwickeln. Dabei teilen
wir die Variabele  auf in $\vb*y$ und $\vb*z$ und entwickeln dann
um $\vb*z=\vb*z_0$
\begin{equation*}
  \vb*x=\vb*x(\underbrace{x_1, x_2,\ldots, x_i}_{\vb*y}, 
  \underbrace{x_{i+1},\ldots, x_{n}}_{\vb*z})=\vb*x(\vb*y, \vb*z) = \vb*y +
  \vb*z
  \Rightarrow f(\vb*x) = f(\vb*y, \vb*z)
\end{equation*}
Wir finden die Entwicklung
\begin{equation}
  f(\vb*y, \vb*z) = e^{\qty((\vb*y + \vb*z) - \vb*z_0)\nabla_{\vb*z'}}
  f(\vb*y, \vb*z')\big|_{\vb*z'=\vb*z_0} = 
  \sum_{n=0}^{\infty} \frac{\nabla^n_{\vb*z'}f(\vb*y,\vb*z')}{n!}
  (\vb*x-\vb*z_0)^n\bigg|_{\vb*z'=\vb*z_0}
\end{equation}

Schneidet man eine Taylorfunktion ab eine bestimmte Ordnung ab, so erhält
man eine gute Näherung für kleine Gebiete um $x_0$. Die Mehrdimensionale
Taylorentwicklung wird in der Elektrodynamik unter anderem für die
Multipolentwicklung benutzt.

Es ist wichtig aufzumerken, dass nicht alle Funktionen eine 
Taylorentwicklung besitzen, sondern nur analytische (holomorphe) 
Funktionen, aber dies ist in der Regel ein Thema für die 
Mathematikvorlesung.
