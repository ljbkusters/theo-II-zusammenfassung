Im Vakuum gelten es die Feldgleichungen
\begin{equation}
  \begin{split}
    \div \vb B(\vr) &= 0\\
    \curl \vb B(\vr) &= 4\pi k' \vb*j(\vr)
  \end{split}
\end{equation}
Daraus folgen die Randbedingungen, analog wie beim elektrischen Feld
\begin{equation}
  \begin{split}
    \vb*n\cdot \vb B(\vr) &= 0\\
    \vb*n\times \vb B(\vr) &= 4\pi k'\vb*j^F(\vr)\\
  \end{split}
\end{equation}
Und ist die orthogonalkomponente des magnetischen Feldes immer stetig an einer Randfläche, während die Tangentialkomponente einen Sprung um $4\pi k\vb*j^F(\vr)$ machen kann in der Anwesendheit von einer nichtverschwindenden Flächenstromdichte.

In medien Gelten die Feldgleichungen
\begin{equation}
  \begin{split}
    \div \vb B(\vr) &= 0\\
    \curl \vb H(\vr) &= \vb*j_f(\vr)
  \end{split}
\end{equation}
mit $\vb*j_f(\vr)$ die freie Stromdichte und gelten die analoge Randbedingungen
\begin{equation*}
  \begin{split}
    \vb*n\cdot \vb B(\vr) &= 0\\
    \vb*n\times \vb H(\vr) &= \vb*j_f^F(\vr)\\
  \end{split}
\end{equation*}
