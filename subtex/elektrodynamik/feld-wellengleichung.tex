\subsection{Entkopplungsansatz der EM-Felder}%
\label{sub:entkopplungs-ansatz}
Im letzten Abschnitt wurden die Maxwellgleichungen ergänzt. Es fällt auf, dass die Differentialgleichungen der elektrische und magnetische Felder gekoppelt sind. Nun wollen wir die Maxwellgleichungen entkoppeln. Weil es sowohl für das $\vE$-Feld als das $\vB$-Feld eine gekoppelte, als eine freie Differentialgleichungen gibt, ist es sinnvol uns mal anzuschauen, was passiert, wenn wir die Rotation der Rotation der entsprechende Felder näher betrachten.
\begin{equation}
  \curl(\curl \vE) \qquad \curl(\curl\vB)
\end{equation}
Es gelten für 2-Mal stetig differenzierbare Skalar- $\varphi$ bzw. Vektorfelder $\vb F$ die folgende Gleichungen
\begin{equation}
  \begin{split}
    \trot\trot\varphi &= \tdiv\tgrad\varphi - \Delta \varphi\\
    \trot\trot\vb F &= \tgrad\tdiv \vb F - \Delta \vb F
  \end{split}
\end{equation}
Betrachten wir zunächst Felder im ladungsfreien Vakuum, d.h.
\begin{equation*}
  \rho(\vr, t) = 0 \qquad \vjrt = 0
\end{equation*}
So kann man das $\vE$-Feld wie folgt vom $\vB$-Feld entkoppeln
\begin{equation}
  \begin{split}
    \grad\underbrace{(\div \vE)}_{\propto\rho=0} - \Delta \vE
    =
    \curl (\curl \vE) 
    &= -\curl k''\pt\vB 
    =
    -\pt\qty(\underbrace{4\pi k'\vj}_{=0} + \frac{k'}{k}\pt \vE)\\
    \Leftrightarrow
    \qty(\Delta - \frac{1}{c^2}\pt^2) \vE &=0 
  \end{split}
\end{equation}
Analog geht dies für das $\vB$-Feld
\begin{equation}
  \begin{split}
    \grad\underbrace{(\div \vB)}_{=0} - \Delta \vB
    =
    \curl (\curl \vB) 
    &= \curl(\underbrace{4\pi k'\vj}_{=0} + \frac{k'}{k}\pt\vE)
    = -k''\frac{k'}{k} \pt^2\vB\\
    \Leftrightarrow
    \qty(\Delta - \frac{1}{c^2}\pt^2) \vB &=0 
  \end{split}
\end{equation}
Dabei findet man für sowohl SI als Gausseinheiten die $1/c^2$ Konstante vor
die Zeitableitung. Diese Gleichung wird auch die \textbf{(homogene) Wellengleichung} genannt, denn ebene Wellen lösen diese Gleichung, was wir im nächsten Abschnitt zeigen werden. Weiter wird
\begin{equation}
  \qty(\Delta - \frac{1}{c^2}\pt^2)\equiv\Box
\end{equation}
die \textit{d'Alembertsche Operator} genannt. Kompakt schreibt man
\begin{equation}
  \Box \vE = 0 \qquad \Box \vB=0
\end{equation}

\subsection{Lösung der inhomogene Wellengleichung}%
\label{sub:loesung-wellengleichung}
Gesucht ist eine Lösung der inhomogene Wellengleichung für allgemeine Felder.
\begin{equation}
  \Box \psi(\vr, t) = f(\vr, t)
\end{equation}
Für die EM-Felder werden wir dann den Spezialfall betrachten, wo die Störfunktion $f(\vr, t)=0$ ist. Weil wir später auch inhomogene Wellengleichungen entgegenkommen werden, ist es aber Sinvoll, direkt den
Zur Lösung der Wellengleichung schauen wir uns die \textbf{d'Alambert-Greensche Funktion} an. Dabei sind $G$ und $\delta$ nun Funktionen von $\vr, \vr', t, t'$ sodass
\begin{equation}
  \label{eq:dalambert-green}
  \Box \greent \equiv \delta(\vr-\vr')\delta(t-t')
\end{equation}
Es folgt nun nach die 2.\ Greensche Identität
\begin{equation}
  \label{eq:psi-green}
  \psi(\vr, t) = \int d^3r'\int dt' \greent f(\vr', t')
\end{equation}
Wobei man erstmal annimt, das die Randbedingungen verschwinden. Um die Lösung zu finden, möchten wir $\greent$ im Fourierraum betrachten, denn dies wird das Finden einer Lösung vereinfachen. Per Definition gilt
\begin{equation}
  \label{eq:fourier-dalembert-green}
  \greent = \frac{1}{(2\pi)^2} \int d^3k\int d\omega \fgreent e^{i\vk\cdot(\vr-\vr')}
  e^{-i\omega(t-t')}
\end{equation}
Und weiter ist für die Fouriertransformation der $\delta$-Distributionen
\begin{equation}
  \begin{split}
    \delta(\vr-\vr')
    &= 
    \frac{1}{(2\pi)^{3/2}} 
    \int d^3k e^{i\vk\cdot(\vr-\vr')}\\
    \delta(t-t')
    &= 
    \frac{1}{(2\pi)^{1/2}} 
    \int d\omega e^{i\omega(t-t')}\\
  \end{split}
\end{equation}
sodass zusammen mit Gleichung~\ref{eq:dalambert-green}
\begin{equation}
  \label{eq:fourier-dalembert-green2}
  \Box \greent =
    \frac{1}{(2\pi)^2} 
    \int d^3k e^{i\vk\cdot(\vr-\vr')}
    \int d\omega e^{i\omega(t-t')}
\end{equation}
folgt, was dann zusammen mit Gleichung~\ref{eq:fourier-dalembert-green}
\begin{equation}
  \begin{split}
    \Box \greent 
    &= 
    \frac{1}{(2\pi)^2} \int d^3k\int d\omega \fgreent\Box 
    e^{i\vk\cdot(\vr-\vr')}
    e^{-i\omega(t-t')}
    \quad(\text{mit Gleichung~\ref{eq:fourier-dalembert-green}})\\
    &=
    \frac{1}{(2\pi)^2} \int d^3k\int d\omega \fgreent
    \qty(\frac{\omega^2}{c^2}-\vk^2)
    e^{i\vk\cdot(\vr-\vr')}
    e^{-i\omega(t-t')}\\
    &=
    \frac{1}{(2\pi)^2} 
    \int d^3k \int d\omega e^{i\vk\cdot(\vr-\vr')}
    e^{i\omega(t-t')}
    \quad(\text{mit Gleichung~\ref{eq:fourier-dalembert-green2}})
  \end{split}
\end{equation}
Daraus folgt
\begin{equation}
  \label{eq:helmholtz-gleichung}
  \fgreent \qty(\frac{\omega^2}{c^2}-\vk^2) = 1
\end{equation}
Eine naive Lösung dieser Gleichung ist nun einfach die algebraische Manipulation durchzuführen woraus man
\begin{equation}
  \fgreent = \frac{1}{\qty(\omega^2/c^2-\vk^2)} 
\end{equation}
findet. Diese Lösung entspricht auch tatsächlich die spezifische Lösung für die inhomogene Wellengleichung. Jedoch ist dies nicht die einzige Lösung, denn, mann kann Lösungen der Form
\begin{equation}
  g(\vk, \omega)\qty(\frac{\omega^2}{c^2} -\vk^2)=0
\end{equation}
auf Gleichung~\ref{eq:helmholtz-gleichung} drauf addieren ohne die rechte Seite zu andern. Hieraus folgen tatsächlich zwei weitere Lösungen, welche die Lösung der homogene Wellengleichung beschreiben. Um eine nicht triviale Lösung für 
$g(\vk,\omega)$ zu finden machen wir den Ansatz
\begin{equation}
  g_{\pm}(\vk, \omega)=a_\pm(\vk)\delta(\omega\pm c\abs{\vk})
\end{equation}
Sodass $g(\vk, \omega)$ nur an die Nullstellen von $\qty(\omega^2/c^2-\vk^2)=0$ von null verschieden ist.
Die allgemeineste Lösung dieser Gleichung wird dann gegeben durch die
superposition aller mögliche Lösung
\begin{equation}
  \begin{split}
  \fgreent 
  &= \frac{1}{\qty(\omega^2/c^2-\vk^2)} 
  + a_+(\vk)\delta(\omega+c\abs{\vk})
  + a_-(\vk)\delta(\omega-c\abs{\vk})\\
  &\equiv
  G_\text{inh}(\vk, \omega) + g_-(\vk, \omega) + g_+(\vk, \omega)
  \end{split}
\end{equation}

Nun fehlt uns nur die Rücktransformation nach Ortskoordinaten. Setzen wir die gefundene Greensche funktion in Gleichung~\ref{eq:fourier-dalembert-green} ein und setzt man dies weiter in Gleichung~\ref{eq:psi-green} ein. So findet man
\begin{center}
\begin{equation}
  \begin{split}
    \psi(\vr, t)
    &=\frac{1}{4\pi^2} \vint'\int dt'
    \int d^3k \int d\omega 
    \fgreent
    e^{i\vk\cdot(\vr-\vr')}
    e^{-i\omega(t-t')}
    f(\vr', t')\\
    &=\psi_\text{inh}(\vr, t) + \psi_+(\vr, t) + \psi_-(\vr, t)\\
  \end{split}
\end{equation}
  mit 
\begin{equation}
  \begin{split}
    \psi_\text{inh}(\vr, t)
    &=
    \frac{1}{4\pi^2} \vint'\int dt'
    \int d^3k \int d\omega 
    G_\text{inh}(\vk, \omega)
    e^{i\vk\cdot(\vr-\vr')}
    e^{-i\omega(t-t')}
    f(\vr', t')\\
    \psi_\pm(\vr, t)
    &=
    \frac{1}{4\pi^2} \vint'\int dt'
    \int d^3k \int d\omega 
    g_\pm(\vk, \omega)
    e^{i\vk\cdot(\vr-\vr')}
    e^{-i\omega(t-t')}
    f(\vr', t')\\
  \end{split}
\end{equation}
\end{center} 
Betrachten wir zunächst die Lösung der homogene Wellengleichung
\begin{equation}
  \psi_\text{hom}(\vr, t) = \psi_+(\vr, t) + \psi_-(\vr, t)
\end{equation}
Weil die Greensche Funktion nur für inhomogene Probleme geeignet ist, können wir die 
