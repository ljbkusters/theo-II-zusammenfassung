\subsection{Entkopplungsansatz der EM-Felder}%
\label{sub:entkopplungs-ansatz}
Im letzten Abschnitt wurden die Maxwellgleichungen ergänzt. Es fällt auf, dass die Differentialgleichungen der elektrische und magnetische Felder gekoppelt sind. Nun wollen wir die Maxwellgleichungen entkoppeln. Weil es sowohl für das $\vE$-Feld als das $\vB$-Feld eine gekoppelte, als eine freie Differentialgleichungen gibt, ist es sinnvol uns mal anzuschauen, was passiert, wenn wir die Rotation der Rotation der entsprechende Felder näher betrachten.
\begin{equation}
  \curl(\curl \vE) \qquad \curl(\curl\vB)
\end{equation}
Es gelten für 2-Mal stetig differenzierbare Skalar- $\varphi$ bzw. Vektorfelder $\vb F$ die folgende Gleichungen
\begin{equation}
  \begin{split}
    \trot\trot\varphi &= \tdiv\tgrad\varphi - \Delta \varphi\\
    \trot\trot\vb F &= \tgrad\tdiv \vb F - \Delta \vb F
  \end{split}
\end{equation}
Betrachten wir zunächst Felder im ladungsfreien Vakuum, d.h.
\begin{equation*}
  \rho(\vr, t) = 0 \qquad \vjrt = 0
\end{equation*}
So kann man das $\vE$-Feld wie folgt vom $\vB$-Feld entkoppeln
\begin{equation}
  \begin{split}
    \grad\underbrace{(\div \vE)}_{\propto\rho=0} - \Delta \vE
    =
    \curl (\curl \vE) 
    &= -\curl k''\pt\vB 
    =
    -\pt\qty(\underbrace{4\pi k'\vj}_{=0} + \frac{k'}{k}\pt \vE)\\
    \Leftrightarrow
    \qty(\Delta - \frac{1}{c^2}\pt^2) \vE &=0 
  \end{split}
\end{equation}
Analog geht dies für das $\vB$-Feld
\begin{equation}
  \begin{split}
    \grad\underbrace{(\div \vB)}_{=0} - \Delta \vB
    =
    \curl (\curl \vB) 
    &= \curl(\underbrace{4\pi k'\vj}_{=0} + \frac{k'}{k}\pt\vE)
    = -k''\frac{k'}{k} \pt^2\vB\\
    \Leftrightarrow
    \qty(\Delta - \frac{1}{c^2}\pt^2) \vB &=0 
  \end{split}
\end{equation}
Dabei findet man für sowohl SI als Gausseinheiten die $1/c^2$ Konstante vor
die Zeitableitung. Diese Gleichung wird auch die \textbf{(homogene) Wellengleichung} genannt, denn ebene Wellen lösen diese Gleichung, was wir im nächsten Abschnitt zeigen werden. Weiter wird
\begin{equation}
  \qty(\Delta - \frac{1}{c^2}\pt^2)\equiv\Box
\end{equation}
die \textit{d'Alembertsche Operator} genannt. Kompakt schreibt man
\begin{equation}
  \Box \vE = 0 \qquad \Box \vB=0
\end{equation}

\subsection{Allegmeine Lösung der Wellengleichung}%
\label{sub:loesung-wellengleichung}
\subsubsection{Lösung der homogene Wellengleichung}%
\label{ssub:homogene-wellengleichung}
Zunächst betrachten wir die Lösung der homogene Wellengleichung für allgemeine Felder, denn die Lösung wird unahängig davon sein ob wir sie durchführen mit einem Skalaren Feld oder einem Vektorfeld. Wir betrachten das Problem
\begin{equation}
  \Box \psi(\vr, t) = 0
\end{equation}
Um dieses Problem zu Lösen wenden wir die Fouriertransformation an. Es gilt zwar
\begin{equation}
  \label{eq:fourier-psi}
  \psi(\vr, t) = F^{-1}[\hat{\psi}(\vk, \omega)]
  = \frac{1}{(2\pi)^2} \int d^3k \int d\omega \,
  \hat{\psi}(\vk,\omega) e^{i\vk\cdot\vr} e^{-i\omega t}
\end{equation}
Wenden wir nun Links und rechts den d'Alambertschen Operator an, so folgt
\begin{equation}
  \Box \psi(\vr, t) = 0 
  = \frac{1}{(2\pi)^2} \int d^3k \int d\omega \,
  (\omega^2/c^2-\vk^2)\hat{\psi}(\vk,\omega) e^{i\vk\cdot\vr} e^{-i\omega t}
\end{equation}
daraus folgt dann
\begin{equation}
  \hat{\psi}(\vk, \omega)\qty(\omega^2/c^2-\vk^2)=0
\end{equation}
Um eine nicht triviale Lösung für 
$\hat{\psi}(\vk,\omega)$ zu finden machen wir den Ansatz
\begin{equation}
  \hat{\psi}(\vk, \omega)=a_\pm(\vk)\delta(\omega\pm c\abs{\vk})
\end{equation}
Sodass $\hat{\psi}_\pm(\vk, \omega)$ nur an die Nullstellen von $\qty(\omega^2/c^2-\vk^2)=0$ von null verschieden ist. Setzt man dies in
Gleichung~\ref{eq:fourier-psi} ein, so folgt
\begin{equation}
  \begin{split}
    \psi(\vr, t) 
    &=
    \frac{1}{4\pi^2} 
    \int d^3k \int d\omega \,
    (\hat\psi_+(\vk, \omega) + \hat\psi_-(\vk, \omega) )
    e^{i\vk\cdot\vr}
    e^{-i\omega t}\\
    &=
    \frac{1}{4\pi^2} 
    \int d^3k \int d\omega \,
    (a_+(\vk)\delta(\omega+c\abs{\vk})+a_-(\vk)\delta(\omega-c\abs{\vk})) 
    e^{i\vk\cdot\vr}
    e^{-i\omega t}\\
    &=
    \frac{1}{4\pi^2} 
    \int d^3k \,
    \qty(a_+(\vk)e^{i(\vk\cdot\vr-c\abs{\vk}t)}+
    a_-(\vk)e^{i(\vk\cdot\vr+c\abs{\vk}t)})
  \end{split}
\end{equation}
Mit den Anfangsbedingungen $\psi(\vr, t=0)=\psi_0(\vr)$ und $\dv t \psi(\vr, t=0)=\nu_0(\vr)$ folgt
\begin{equation}
  \psi_0(\vr)
  =
  \frac{1}{4\pi^2} 
  \int d^3k \,
  \qty[a_+(\vk)+
  a_-(\vk)]e^{i(\vk\cdot\vr)}
\end{equation}
und
\begin{equation}
  \nu_0(\vr)
  =
  \frac{ic\abs{\vk}}{4\pi^2} 
  \int d^3k \,
  \qty[a_+(\vk)-
  a_-(\vk)]e^{i(\vk\cdot\vr)}
\end{equation}
Damit sind man $a_\pm(\vk)$ festgelegt, und folgt nach etwas umformen und
eine Fourier-Rücktransformation
\begin{equation}
  a_\pm(\vk) =
  \frac{1}{\sqrt{2\pi}} \int d^3r e^{-i\vk\cdot\vr}
  \qty(
  \psi_0(\vr) \mp \frac{i}{c\abs{\vk}} \nu_0(\vr)
  )
\end{equation}
Damit ist die Lösung der Wellengleichung für beliebige Anfangsbedingungen festgelegt. Es ist nun leicht zu zeigen dass
\begin{equation}
  \psi(\vr, t) = A e^{i(\vk\cdot\vr-\omega t + \varphi)}
  \quad
  \text{\textbf{Ebene Welle}}
\end{equation}
mit $A\in \mathbb{C}$ und $\vk, \vr, t, \omega, \varphi\in \mathbb{R}$

\subsubsection{Lösung der inhomogene Wellengleichung}%
\label{ssub:inhomogene-wellengleichung}
Gesucht ist nun eine Lösung der inhomogene Wellengleichung für allgemeine Felder.
\begin{equation}
  \Box \psi(\vr, t) = f(\vr, t)
\end{equation}
Für die EM-Felder werden wir dann den Spezialfall betrachten, wo die Störfunktion $f(\vr, t)=0$ ist. Weil wir später auch inhomogene Wellengleichungen entgegenkommen werden, ist es aber Sinvoll, direkt den
Zur Lösung der Wellengleichung schauen wir uns die \textbf{d'Alambert-Greensche Funktion} an. Dabei sind $G$ und $\delta$ nun Funktionen von $\vr, \vr', t, t'$ sodass
\begin{equation}
  \label{eq:dalambert-green}
  \Box \greent \equiv \delta(\vr-\vr')\delta(t-t')
\end{equation}
Es folgt nun nach die 2.\ Greensche Identität
\begin{equation}
  \label{eq:psi-green}
  \psi(\vr, t) = \int d^3r'\int dt' \greent f(\vr', t')
\end{equation}
Wobei man erstmal annimt, das die Randbedingungen verschwinden. Um die Lösung zu finden, möchten wir $\greent$ im Fourierraum betrachten, denn dies wird das Finden einer Lösung vereinfachen. Per Definition gilt
\begin{equation}
  \label{eq:fourier-dalembert-green}
  \greent = \frac{1}{(2\pi)^2} \int d^3k\int d\omega \fgreent e^{i\vk\cdot(\vr-\vr')}
  e^{-i\omega(t-t')}
\end{equation}
Und weiter ist für die Fouriertransformation der $\delta$-Distributionen
\begin{equation}
  \begin{split}
    \delta(\vr-\vr')
    &= 
    \frac{1}{(2\pi)^{3/2}} 
    \int d^3k e^{i\vk\cdot(\vr-\vr')}\\
    \delta(t-t')
    &= 
    \frac{1}{(2\pi)^{1/2}} 
    \int d\omega e^{i\omega(t-t')}\\
  \end{split}
\end{equation}
sodass zusammen mit Gleichung~\ref{eq:dalambert-green}
\begin{equation}
  \label{eq:fourier-dalembert-green2}
  \Box \greent =
    \frac{1}{(2\pi)^2} 
    \int d^3k e^{i\vk\cdot(\vr-\vr')}
    \int d\omega e^{i\omega(t-t')}
\end{equation}
folgt, was dann zusammen mit Gleichung~\ref{eq:fourier-dalembert-green}
\begin{equation}
  \begin{split}
    \Box \greent 
    &= 
    \frac{1}{(2\pi)^2} \int d^3k\int d\omega \fgreent\Box 
    e^{i\vk\cdot(\vr-\vr')}
    e^{-i\omega(t-t')}
    \quad(\text{mit Gleichung~\ref{eq:fourier-dalembert-green}})\\
    &=
    \frac{1}{(2\pi)^2} \int d^3k\int d\omega \fgreent
    \qty(\frac{\omega^2}{c^2}-\vk^2)
    e^{i\vk\cdot(\vr-\vr')}
    e^{-i\omega(t-t')}\\
    &=
    \frac{1}{(2\pi)^2} 
    \int d^3k \int d\omega e^{i\vk\cdot(\vr-\vr')}
    e^{i\omega(t-t')}
    \quad(\text{mit Gleichung~\ref{eq:fourier-dalembert-green2}})
  \end{split}
\end{equation}
Daraus folgt
\begin{equation}
  \label{eq:helmholtz-gleichung}
  \fgreent \qty(\frac{\omega^2}{c^2}-\vk^2) = 1
\end{equation}
Eine naive Lösung dieser Gleichung ist nun einfach die algebraische Manipulation durchzuführen woraus man
\begin{equation}
  \fgreent = \frac{1}{\qty(\omega^2/c^2-\vk^2)} 
\end{equation}
findet. Diese Lösung entspricht auch tatsächlich die spezifische Lösung für die inhomogene Wellengleichung. Jedoch ist dies nicht die einzige Lösung, denn, mann kann Lösungen der Form
\begin{equation}
  g(\vk, \omega)\qty(\frac{\omega^2}{c^2} -\vk^2)=0
\end{equation}
auf Gleichung~\ref{eq:helmholtz-gleichung} drauf addieren ohne die rechte Seite zu andern. Hieraus folgen tatsächlich zwei weitere Lösungen, welche die Lösung der homogene Wellengleichung beschreiben. 

Die allgemeineste Lösung dieser Gleichung wird dann gegeben durch die
superposition aller mögliche Lösung
\begin{equation}
  \begin{split}
  \fgreent 
  &= \frac{1}{\qty(\omega^2/c^2-\vk^2)} 
  + a_+(\vk)\delta(\omega+c\abs{\vk})
  + a_-(\vk)\delta(\omega-c\abs{\vk})\\
  &\equiv
  G_\text{inh}(\vk, \omega) + g_-(\vk, \omega) + g_+(\vk, \omega)
  \end{split}
\end{equation}

Nun fehlt uns nur die Rücktransformation nach Ortskoordinaten. Setzen wir die gefundene Greensche funktion in Gleichung~\ref{eq:fourier-dalembert-green} ein und setzt man dies weiter in Gleichung~\ref{eq:psi-green} ein. So findet man
\begin{center}
\begin{equation}
  \begin{split}
    \psi(\vr, t)
    &=\frac{1}{4\pi^2} \vint'\int dt'
    \int d^3k \int d\omega 
    \fgreent
    e^{i\vk\cdot(\vr-\vr')}
    e^{-i\omega(t-t')}
    f(\vr', t')\\
    &=\psi_\text{inh}(\vr, t) + \psi_+(\vr, t) + \psi_-(\vr, t)\\
  \end{split}
\end{equation}
  mit 
\begin{equation}
  \begin{split}
    \psi_\text{inh}(\vr, t)
    &=
    \frac{1}{4\pi^2} \vint'\int dt'
    \int d^3k \int d\omega 
    G_\text{inh}(\vk, \omega)
    e^{i\vk\cdot(\vr-\vr')}
    e^{-i\omega(t-t')}
    f(\vr', t')\\
    \psi_\pm(\vr, t)
    &=
    \frac{1}{4\pi^2} \vint'\int dt'
    \int d^3k \int d\omega 
    g_\pm(\vk, \omega)
    e^{i\vk\cdot(\vr-\vr')}
    e^{-i\omega(t-t')}
    f(\vr', t')\\
  \end{split}
\end{equation}
\end{center} 
Betrachten wir zunächst die Lösung der homogene Wellengleichung
\begin{equation}
  \psi_\text{hom}(\vr, t) = \psi_+(\vr, t) + \psi_-(\vr, t)
\end{equation}
