In der Magnetostatik betrachtet man nur \textbf{Zeitunabhängige} Stromdichten (daher Magneto\textit{statik}). Dies führt dazu, sowie in der
Elektrostatik, das man Stationäre Felder betrachtet, und das Zeitunabhängige
Effekte vernachlässigt werden können. Zwischen der Elektrostatik und
der Magnetostatik gibt es viele Ähnlichkeiten. Deswegen werden wir nur
kurz alle wichtige Themen besprechen ohne Herleitungen zu wiederholen.

\section{Stromdichten und Kontinuitätsgleichung}%
\label{sec:stromdichten}
In der Magnetostatik dürfen sich Ladungen endlich bewegen. Dies führt zu
\textbf{Ladungsströme}, die mathematisch durch die \textbf{Stromdichte}
\(\vb*j(\vr, t)\) beschrieben werden. In der Magnetostatik betrachtet man
zunächst Zeitunabhängige Stromdichten $\vb*j(\vr, t)=\vb*j(\vr)$. Weiter
ist auch noch immer die Ladungsverteilung im Raum zeitunabhängig. Weil die
Ladung eine physikalische Erhaltungsgröße ist (\textbf{Ladungserhaltung}),
und Ladung sowohl die Ursprung der Ladungsdichte als die Stromdichte ist,
sind die Größen $\rho$ und $\vb*j$ natürlich verknüpft. Die Ladungserhaltung
wird Mathemathisch beschrieben durch die \textbf{Kontinuitätsgleichung}
\begin{equation}
  -\div\vb*j(\vr, t) = \partial_t \rho(\vr, t)
\end{equation}
Man kann dies So auffassen, dass die Fluß von Ladung in durch die Randfläche
einer Raumbereich rein (\(-\div\vb*j(\vr, t)\)) gleich die Zeitliche Änderung
(\(\partial_t\rho(\vr, t)\)) der Ladung im Raumbereich ist. Bemerke, dass das Minus da ist, weil für eine positive Ladungszuname die man für einen strom
von positiven Ladungen, $\div\vb*j(\vr, t)$ eine Senke
darstellen soll, und nicht eine Quelle!

Weil die Ladungsdichte zeitunabhängig sein muss in der Magnetostatik, folgt
direkt dass $\div\vb*j(\vr)=0$. Dies bedeutet also, dass die Stromdichte
in einem Raumbereich rein also immer gleich sein muss als die Stromdichte aus
dem Raumbereich raus, sowie zum Beispiel in einem Leiterdraht oder Leiterschleife.

Die Ladungsdichte die durch eine Punktladung erzeugt wird ist einfach die
Ladungsdichte der Punktladung (Gleichung~\ref{eq:}) mal seine Momentane geschwindigkeit.
\begin{equation*}
  \vb*j(\vr) = q \dot{\vr}(t) \delta(\vr -\vr(t))
\end{equation*}
Für mehrere Ladungsverteilungen folgt natürlich
\begin{equation}
  \vb*j(\vr) = \sum_i q_i \dot{\vr}_i(t) \delta(\vr-\vr_i(t))
\end{equation}
Man kann dies dann wieder Räumlich mitteln um eine kontinuierliche Ladungsverteilung zu erhalten.
\begin{equation}
  \vb*j_{\text{kont}}(\vr) = \frac{1}{V(\vr)}\sum_{\vb*j_i\in V(\vr)}\vb*j_i 
\end{equation}
