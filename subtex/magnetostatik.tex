In der Magnetostatik betrachtet man nur \textbf{Zeitunabhängige} Stromdichten (daher Magneto\textit{statik}). Dies führt dazu, sowie in der
Elektrostatik, das man Stationäre Felder betrachtet, und das Zeitunabhängige
Effekte vernachlässigt werden können. Zwischen der Elektrostatik und
der Magnetostatik gibt es viele Ähnlichkeiten. Deswegen werden wir nur
kurz alle wichtige Themen besprechen ohne Herleitungen zu wiederholen.

\section{Stromdichten und Kontinuitätsgleichung}%
\label{sec:stromdichten}
In der Magnetostatik dürfen sich Ladungen endlich bewegen. Dies führt zu
\textbf{Ladungsströme}, die mathematisch durch die \textbf{Stromdichte}
\(\vb*j(\vr, t)\) beschrieben werden. In der Magnetostatik betrachtet man
zunächst Zeitunabhängige Stromdichten $\vb*j(\vr, t)=\vb*j(\vr)$. Weiter
ist auch noch immer die Ladungsverteilung im Raum zeitunabhängig. Weil die
Ladung eine physikalische Erhaltungsgröße ist (\textbf{Ladungserhaltung}),
und Ladung sowohl die Ursprung der Ladungsdichte als die Stromdichte ist,
sind die Größen $\rho$ und $\vb*j$ natürlich verknüpft. Die Ladungserhaltung
wird Mathemathisch beschrieben durch die \textbf{Kontinuitätsgleichung}
\begin{equation}
  -\div\vb*j(\vr, t) = \partial_t \rho(\vr, t)
\end{equation}
Man kann dies So auffassen, dass die Fluß von Ladung in durch die Randfläche
einer Raumbereich rein (\(-\div\vb*j(\vr, t)\)) gleich die Zeitliche Änderung
(\(\partial_t\rho(\vr, t)\)) der Ladung im Raumbereich ist. Bemerke, dass das Minus da ist, weil für eine positive Ladungszuname die man für einen strom
von positiven Ladungen, $\div\vb*j(\vr, t)$ eine Senke
darstellen soll, und nicht eine Quelle!

Weil die Ladungsdichte zeitunabhängig sein muss in der Magnetostatik, folgt
direkt dass $\div\vb*j(\vr)=0$. Dies bedeutet also, dass die Stromdichte
in einem Raumbereich rein also immer gleich sein muss als die Stromdichte aus
dem Raumbereich raus, sowie zum Beispiel in einem Leiterdraht oder Leiterschleife.

Die Ladungsdichte die durch eine Punktladung erzeugt wird ist einfach die
Ladungsdichte der Punktladung (Gleichung~\ref{eq:}) mal seine Momentane geschwindigkeit.
\begin{equation*}
  \vb*j(\vr) = q \dot{\vr}(t) \delta(\vr -\vr(t))
\end{equation*}
Für mehrere Ladungsverteilungen folgt natürlich
\begin{equation}
  \vb*j(\vr) = \sum_i q_i \dot{\vr}_i(t) \delta(\vr-\vr_i(t))
\end{equation}
Man kann dies dann wieder Räumlich mitteln um eine kontinuierliche Ladungsverteilung zu erhalten.
\begin{equation}
  \vb*j_{\text{kont}}(\vr) = \frac{1}{V(\vr)}\sum_{\vb*j_i\in V(\vr)}\vb*j_i 
\end{equation}
Der gesammte Strom durch eine Oberfläche wird auch als $I$ angedeutet. Es
gilt
\begin{equation}
  I_A = \int_A \vb*j(\vr) \ddA
\end{equation}


\newpage
\section{Magnetfeld und Biot-Savart-Gesetz}%
\label{sec:magnetfeld}
\subsection{Ampèresche Kraft Gesetz}%
\label{sub:amperesche-gesetz}
Analog zu wie die Ladungsdichte die Quelle für das Elektrostatische-Feld, ist
die die Stromdichte die Quelle für das Magnetfeld. Im frühen 19. Jahrhundert wurde beobachtet dass Zwei stromdurchflossenen Leiterschleifen Kräfte auf einander ausübten. Die Kraft wirkt Anziehend falls die Ströme gleich ausgerichtet sind und vise versa. Diese magnetische Kraft wurde in 1823 von André Marie Ampère zuerst mathematisch beschrieben, und ist gegeben durch das \textbf{Ampèresche Kraft Gesetz} (nicht zu verwechseln
mit dem Ampèresche Durchfluttungsgesetz).

Gäbe es zwei Leiterschleifen die durch die einfache, geschlossene und disjunkte Raumkurven $C_1$ bzw. $C_2$ beschrieben werden, die durch die ströme $I_1$ bzw. $I_2$ durchflossen werden.
Sind $\vr_1$ und $\vr_2$ weiter zwei unabhängige Ortsvektoren die vom Ursprung
aus auf Punkte der Kurven $C_1$ bzw. $C_2$ weisen, so gilt für die Kraft die
Leiterschleife 1 auf Leiterschleife 2 ausübt. 
\begin{equation} 
  \vb F_{1\to2}=k'I_1I_2\oint_{C1}\oint_{C2} \frac{d\vr_1\times(d\vr_2
  \times \vr_{12})}{r_{12}^3}=k'I_1I_2\oint_{C1}\oint_{C2} d\vr_1 \cdot d\vr_2 \frac{\vr_{12}}{r_{12}^3} \qquad \vr_{12}=\vr_1-\vr_2
\end{equation}
$k'$ Ist wieder eine
Konstante die vom gewählten Einheitensystem abhängt. In SI gilt $k'=\frac{\mu_0}{4\pi}$. In Gausseinheiten gilt $k'=\frac{1}{c}$ mit $c$ die
Lichtgeschwindigkeit. Wie wir in der Elektrodynamik sehen werden ist auch $\mu_0$ zusammen mit $\epsilon_0$ durch mit $c$ verknüpft ($c^2=\frac{1}{\mu_0\epsilon_0}$ in SI). Eine relativistische beschreibung der Elektrodynamik weißt eine Äquivalenz zwischen dem Coulomb Gesetz und
das Ampèresche Gesetz auf. Ob eine Elektrische oder eine Magnetische Kraft
wirkt, wird dann abhängig vom gewählten Bezugssystem sein.

\subsection{Das Magnetfeld}%
\label{sub:magnetfeld}
Es ist nun wieder sinnvol, eine Beschreibung zu finden, analog zur Definition des Elektrischen Feldes, wobei die magnetische Kraft durch beschrieben wird durch ein äußeres (Magnet-)Feld das lokal auf eine Stromverteilung wirkt. Man definiere das Magnetfeld aus ersten Prinzipien
aus die Ampèresche Kraft sodass
\begin{equation}
  \vb F_{12} = I_1\oint_{C_1} d\vr_1\times \vb B_2(\vr_1)
  \quad\text{mit}\quad
  \vb B_1(\vr_1) = k'I_2 \oint_{C_2} d\vr_2 \times \frac{\vr_{12}}{r^3_{12}} 
\end{equation}
Das Magnetfeld läßt sich dann zu beliebige Stromverteilungen veralgemeinern, und wird das \textbf{Biot-Savart-Gesetz} genannt
\begin{equation}
  \vb B(\vr) = k'\int d^3 r' \vb*j(\vr')\times\frac{(\vr-\vr')}{\rr^3}
\end{equation}
Dies entspricht, sowie in der Elektrostatik, wieder eine lösung für
verschwindende Randbedingungen.
Die gesammte magnetische Kraft auf eine beliebige Ladungsverteilung im außeren Magnetfeld wird zu
\begin{equation}
  \vb F = \int d^3r \vb*j(\vr) \times \vb B(\vr)
\end{equation}
Das Magnetfeld ist nun also ein Maß, für die Kraft die ein infinitisimales Leiterstück bzw.\ eine bewegte Ladung lokal spüren Würde, durch die
Anwesendheit eines Äußeren Stroms.

\subsection{Das Vektorpotential}%
\label{sub:Vektorpotential}
Es ist nun offensichtlich dass das Magnetfeld ein reines Rotationsfeld ist,
indem man sieht dass
\begin{equation}
  \vb \nabla_r \times \frac{\vb*j(\vr')}{\rr} 
  = \vb*j(\vr')\times\frac{(\vr-\vr')}{\rr^3} 
\end{equation}
Aus dem Biot-Savart-Gesetz folgt also nun
\begin{equation}
    \label{eq:vektorpotential}
    \vb B(\vr) = \curl \vb A(\vr)\quad\text{mit}
    \quad\vb A(\vr) = k'\int d^3r' \frac{\vb*j(\vr')}{\rr}
\end{equation} 

\subsection{Feldgleichungen der Magnetostatik}%
\label{sub:feldgleichungen-magnetostatik}
Sowie schon im letzten Abschnitt besprochen ist das Magnetfeld ein reines
Rotationsfeld. Daraus folgt die homogene Feldgleichung der Magnetostatik.
Aus dem Zerlegungssatz folgt zusammen mit die homogene Feldgleichung auch
den Zusammenhang zwischen die Rotation des Magnetfeldes $\vb B(\vr)$ und die Stromdichte $\vb*j(\vr)$
\begin{equation}
  \begin{aligned}
    \div \vb B(\vr) &= 0 & \text{(homogen)}\\
    \curl \vb B(\vr) &= 4\pi k'\vb*j(\vr) & \text{(inhomogen)}\\
  \end{aligned}
\end{equation}

\subsection{Eichtransformation und Inhomogene Poisson-Gleichung}%
\label{sub:eichtransformation}
Das Vektorpotential ist nicht eindeutig Festgelegt, denn
\begin{equation}
  \vb B(\vr) = \curl \vb A'(\vr)= \curl (\vb A(\vr) + \vb F(\vr)) 
  \quad\text{falls}\quad\curl \vb F(\vr)=0
  \Leftrightarrow \vb F(\vr) \equiv \grad \chi(\vr)
\end{equation}
Für ein beliebiges Skalarfeld $\chi$.

Das heißt das für die Magnetostatik die \textbf{Eichtransformation}
\begin{equation}
  \vb A'(\vr) = \vb A(\vr) + \grad \chi(\vr)
\end{equation}
das Magnetfeld invariant läßt. In der Magnetostatik ist die \textbf{Coulomb-Eichung} üblich, wobei $\div \vb A=0$ gewählt wird, sodaß $\Delta \chi(\vr)=0$. In der Coulomb-Eichung folgt somit zusammen mit die magnetostatische Maxwellgleichungen die inhomogene Poisson-Gleichung
\begin{equation}
  \Delta \vb A(\vr) = -4\pi k' \vb*j(\vr)
\end{equation}

\subsection{Die Hauptaufgabe der Magnetostatik} 
Die Hauptaufgabe der Magnetostatik ist also das Berechnen von 
magnetostatische Potentiale und Felder unter vorgabe von 
Stromverteilungen und Randbedingungen. 
Sowie bei der Elektrostatik sind für Hochsymmetrische Probleme analytische 
Lösungen möglich, für schwierigere Probleme können wieder meistens nur numerische 
Lösungen gefunden werden. Für uns ist erstmal wieder das Finden von analytische
Lösungen interessant um ein Grundverständnis aufzubauen. Dazu werden weiter auch wieder Methoden besprochen die uns analytische Näherungen geben können, wie die Multipolentwicklung.


\newpage
\section{Multipolentwicklung: Magnetischer Dipol}%
Zunächst besprechen wir wie man analytische Näherungen finden kann
unter angabe von komplexere Ladungsverteilungen. 
Das Problem ist meistens das Integrieren der
\begin{equation*}
  \frac{1}{\abs{\vb*r-\vb*r'}}
\end{equation*}
Term (in Kombination mit die Ladungsdichte $\rho(\vb*r')$). 
Deswegen möchten wir in der \textbf{karthesische Multipolentwicklung} 
diesen  Term Taylor-entwickln, damit wir es in eine 
Reihe von Polinomiale Terme umwandeln können, 
weil diese Einfach(er) zu integrieren ist. 
Daneben gibt es noch die \textbf{Kugelfächen-Entwicklung} die sich 
insbesondere für Radial- bzw. Rotationssymmetrische Probleme eignet, 
die wir Später besprechen werden. 

\subsection{Karthesische Multipolentwicklung}%
\label{ssub:Karthesische-Multipolentwicklung}
Die karthesische Multipolentwicklung wird mittels eine Taylor-Entwicklung
hergeleitet. Man muss dabei eine Multidimensionale Taylor-Entwicklung 
durchführen.

Eine allgemeine mehrdimensionale Taylorentwicklung wird gegeben durch:
\begin{equation*}
  f(\vb*\alpha, \vb*\beta)
  =\qty(\exp(\vb*\beta\cdot\nabla_{\vb*\beta'})
  f(\vb*\alpha, \vb*\beta'))
  \bigg|_{\vb*\beta'=\vb*\beta_0}
  =\qty(\sum_{n=0}^\infty
  \frac{(\vb*\beta\cdot\nabla_{\vb*\beta'})^n}{n!}
  f(\vb*\alpha, \vb*\beta'))
  \bigg|_{\vb*\beta'=\vb*\beta_0}
\end{equation*}
Man bemerke dass $\nabla$ einen Operator ist! Die exponential Funktion
dient hier nur zur vereinfachung der Darstellung! Man Taylore nun um $\vb*r'=0$ (dies heißt, daß $\vb*r\gg\vb*r'$ sodass 
$\vb*r-\vb*r'\approx\vb*r$, dafür muss $\vb*r$ natürlich weit
von der Quelle entfernt sein). Die Multipolentwicklung ist also ein 
\textbf{Fernfeldnäherung}.

Man definiere nun $f(\vb*r,\vb*r')=f(r_1,r_2,r_3,r_1',r_2',r_3')
\equiv\frac{1}{\abs{\vb*r-\vb*r'}}$ sodass bis zur 2. Ordnung die Taylorentwicklung für unsere Funktion
wie folgt aussieht:

\begin{equation*}
  \frac{1}{\abs{\vb*r-\vb*r'}}=f(\vb*r, 0)
  + (\vb*r'\cdot \nabla_{\vb*{\bar{r}'}}) f(\vb*r, \vb*{\bar{r}}')\bigg|_{\vb*{\bar{r}}'=0}
  + \frac{1}{2}(\vb*r'\cdot\nabla_{\vb*{\bar r'}})^2
  f(\vb*r,\vb*{\bar r}')
  \bigg|_{\vb*{\bar r'}=0}
  + \ldots
\end{equation*}

\underline{0. Ordnung:}
\begin{equation*}
  \frac{1}{\abs{\vb*r-\vb*{\bar r'}}}\bigg|_{\vb*{\bar r'}=0} 
  = \frac{1}{\abs{\vb*r}}= \frac{1}{r} 
\end{equation*}

\underline{1. Ordnung:}
\begin{equation*}
  \vb*{r'}\cdot
  \nabla_{\vb*{\bar{r}'}}\frac{1}{\abs{\vb*r-\vb*{\bar{r}'}}}
  \bigg|_{\vb*{\bar r'}=0}
  = \frac{\vb*{ r'}\cdot\vb*r}{r^3} 
\end{equation*}

\underline{2. Ordnung:}
\begin{equation*}
  \begin{split}
  \frac{1}{2}(\vb*{\bar r'}\cdot\nabla_{\vb*{\bar{r}'}})^2
  \frac{1}{\abs{\vb*r-\vb*{\bar r'}}}\bigg|_{\vb*{\bar r'}=0}
  &=
  \frac{1}{2}\qty[r_i'\pdv {\bar r_i'}]\qty[r_j'\pdv{\bar r_j'}]
  \frac{1}{\abs{\vb*r-\vb*{\bar r'}}}
  \bigg|_{\vb*{\bar r'}=0}\\
  &=
  \frac{r'_ir'_j}{2} 
  \frac{\partial^2}{\partial_{\bar r_i'}\partial_{\bar r_j'}} 
  \frac{1}{\abs{\vb*r-\vb*{\bar r'}}}\bigg|_{\vb*{\bar r'}=0}\\
  &=
  \frac{r'_ir'_j}{2} \frac{\partial}{\partial_{\bar r_i'}} 
  \frac{x_j}{\abs{\vb*r-\vb*{\bar r'}}^3}\bigg|_{\vb*{\bar r'}=0}
  \qquad x_j\equiv (r_j-\bar r_j')\\
  &=
  \frac{r'_ir'_j}{2} \frac{\partial}{\partial_{\bar r_i'}} 
  \frac{x_j}{\abs{\vb*r-\vb*{\bar r'}}^3}\bigg|_{\vb*{\bar r'}=0}\\
  &=
  \frac{r'_ir'_j}{2} 
  \qty(
    \text{-}\frac{3}{2}
    \frac{\text{-}2x_i x_j}{\abs{\vb*r-\vb*{\bar r'}}^5}
    -\frac{\delta_{ij}}{\abs{\vb*r-\vb*{\bar r'}}^3} 
  )\bigg|_{\vb*{\bar r'}=0}\quad\text{(Produktregel)}\\
  &=
  \frac{1}{2}\sum_{i,j=1}^3 \frac{3r_ir_j-\delta_{ij}\vb*r^2}{r^5}
  r_i'r_j'
  \end{split}
\end{equation*}
\textit{Eine wichtige Bemerkung:}
Aus Symmetrie Grunden gilt
\begin{equation*}
  \frac{1}{2}\sum_{i,j=1}^3 \frac{3r_ir_j-\delta_{ij}\vb*r^2}{r^5}
  r_i'r_j'
  =
  \frac{1}{2}\sum_{i,j=1}^3 \frac{3r'_ir'_j-\delta_{ij}\vb*r'^2}{r^5}
  r_ir_j
\end{equation*}
Wir benutzen im allgemeinen die letzte Definition wenn wir die
elektrische und magnetische Multipole berechnen. Man findet also bis zum
2. Ordnung
\begin{equation}
  \frac{1}{\abs{\vb*r-\vb*r'}}
  \approx 
  \frac{1}{r} 
  + \frac{\vb*r'\cdot\vb*r}{r^3} 
  + \frac{1}{2}\sum_{i,j=1}^3 
    \frac{3r'_ir'_j-\delta_{ij}\vb*r'^2}{r^5}r_ir_j
\end{equation}
Setzt man dies in die Definition für das Elektrische Potential ein, so 
findet man
\begin{equation*}
  \begin{split}
    \phi(\vb*r) 
    &= k \int d^3 r' \frac{\rho(\vb*r')}{\abs{\vb*r - \vb*r'}} \\
    & \approx k\int d^3 r' \rho(\vb*r')
    \qty(
    \frac{1}{r} 
    + \frac{\vb*r'\cdot\vb*r}{r^3} 
    + \frac{1}{2}\sum_{i,j=1}^3 
    \frac{3r'_ir'_j-\delta_{ij}\vb*r'^2}{r^5}r_ir_j)\\
    &\equiv 
    k \frac{Q}{r} + k \frac{\vb*p\cdot\vb*r}{r^3} 
    + \frac{k}{2} \sum_{i,j=1}^{3} \frac{r_ir_j}{r^5}Q_{ij}
  \end{split}
\end{equation*}

\begin{center}
\begin{tabular}{ll}
  Monopol:    & $\ds Q=\int d^3r\rho(\vb* r)$ \quad\text(Gesammtladung)\\
  Dipol:      & $\ds p_i=\int d^3r\rho(\vb* r)r_i
                \quad\vb* p=p_i\vu{e}_i $\\
  Quadrupol:  & $\ds Q_{ij}=\int d^3r\rho(\vb* r)
  \qty(3r_ir_j - \delta_{ij}\abs{\vb* r}^2)$
\end{tabular}
\end{center}
Dabei hat die Quadrupoltensor $Q_{ij}$ nur 5 Freiheitsgraden. 
Mit nur 5 Rechnungen alle (9) Quadrupol Elemente berechnen. 
Es gilt zwar $Q_{ij}=Q_{ji}$ (symmertrisch) und $\text{sp}(\bm Q)=\sum_i Q_{ii}=0$ (spurfrei).

Im allgemeinen berechnet 
man keine weitere Ordnungen analytisch im Bachelorstudium, 
und zum Verständniss bringt dies auch nicht mehr (außer ärger), 
sodass höhere Ordnungen berechnen 
ein Problem ist das man lieber an Computer überlässt.

\subsection{Sphärische Multipolentwicklung}%
\label{ssub:sphaerische-Multipolentwicklung}
Für Ladungsverteilungen die Radial- oder Rotationssymetrisch sind ist die 
Sphärische Multipolentwicklung besonders geeignet, vor allem falls man die 
Ladungsdichte als Linearkombination von Kugelfächen-Funktionen schreiben 
kann. Um die Sphärische Multipolentwicklung zu motivieren machen wir 
zunächst die Annahme (mit dem Seperationsansatz), dass das 
Winkelanteil des Potentials unabhähngig vom Radialanteil ist, also
\begin{equation*}
  \phi(\vb*r) = \phi_r(r)\phi_\Omega(\theta,\varphi)
\end{equation*}
Weil man (wie wir später besprechen werden) die Poisson-Gleichung mit dem
gleichen Ansatz lösen wird, und die Kugelflächen-Funktionen 
$Y_{lm}(\theta,\varphi)$ eine Eigenfunktion der Laplace Operator ist, 
ist es eine gute Idee um dies als Basis für eine Sphärische Entwicklung 
zu benutzen. Dazu formen die Kugelflächen-Funktionen unter Integration über
$d\Omega=\sin\theta d\theta d\varphi$ eine orthonormale Basis. Es gilt die untere Zusammenhang.
\begin{equation*}
  \int d\Omega Y_{lm}(\Omega)Y^*_{l'm'}(\Omega)=\delta_{ll'}\delta_{mm'}
\end{equation*}

Die Entwicklung in Kugelkoordinaten ist rechnerisch etwas aufwändig, und
wird hier erstmal übersprungen, aber sie wird in gute Literatur über
die Elektrodynamik oft gegeben, wie z.B. in Kapitel 2.3.8 von Noltings
``Grundkurs Theoretische Physik 3, Elektrodynamik''. 
Man findet 
\begin{equation}
  \frac{1}{\abs{\vb*r-\vb*r'}}
  =
  \sum_{lm} \frac{4\pi}{2l+1} \frac{r_<^l}{r_>^{l+1}}
  Y^*_{lm}(\Omega')Y_{lm}(\Omega)
\end{equation}
\begin{center}
mit $l=0,1,2,\ldots$ und $m=-l,\ldots,l$\\
$r_<=\min(r,r')$, $r_>=\max(r,r')$.
\end{center}
Setzt man dies in der Definition für das Elektrische Potential ein, 
so findet man
\begin{equation}
  \begin{split}
    \phi(\vb*r) 
    &= k\int d^3r' \rho(\vb*r') \frac{1}{\abs{\vb*r-\vb*r'}} \\
    &= k\sum_{lm} \frac{4\pi}{2l+1} 
    \int_0^\infty dr'{r'}^2 \frac{r_<^l}{r_>^{l+1}} 
    \int_\Omega d\Omega' Y^*_{lm}(\Omega')Y_{lm}(\Omega) \rho(\vb*r')\\
    &= k\sum_{lm} \frac{4\pi}{2l+1} Y_{lm}(\Omega) 
    \qty(
    \int_0^{r} dr'{r'}^2 \frac{{r'}^l}{r^{l+1}} +
    \int_{r}^{\infty} dr'{r'}^2 \frac{r^l}{{r'}^{l+1}} 
    )
    \int_\Omega d\Omega' Y^*_{lm}(\Omega')\rho(\vb*r')\\
  \end{split}
\end{equation}
Sei nun die Ladungsdichte 
$\rho(\vb*r')=\sum_{l'm'}f_{f'm'}(r')Y_{l'm'}(\Omega')$
für beliebige $l'$ und $m'$ (also einfach irgendeine Linearkombination von
beliebige Kugelflächenfunktionen) so vereinfacht sich das Problem weiter,
und kann man für eine Endliche Summe sogar exakte Lösungen finden.
(Es ist nicht notwendig dass man die Ladungsdichte als Linearkombination
von Kugelflächenfunktionen schreibt, nur einfacher).
\begin{equation}
  \begin{split}
    \phi(\vb*r) 
    &= k\sum_{lm} \frac{4\pi}{2l+1} Y_{lm}(\Omega) \sum_{l'm'}
    \qty(
    \int_0^{r} dr'\frac{{r'}^(l+2)}{r^{l+1}}f_{l'm'}(r) +
    \int_{r}^{\infty} dr'\frac{r^l}{{r'}^{l-1}}f_{l'm'}(r) 
    )
    \int_\Omega d\Omega' Y^*_{lm}(\Omega')Y_{l'm'}(\Omega')\\
    &= k\sum_{lm} \frac{4\pi}{2l+1} Y_{lm}(\Omega) \sum_{l'm'}
    \qty(
    \int_0^{r} dr'\frac{{r'}^{l+2}}{r^{l+1}}f_{l'm'}(r) +
    \int_{r}^{\infty} dr'\frac{r^l}{{r'}^{l-1}}f_{l'm'}(r) 
    )
    \delta_{ll'}\delta_{mm'}\\
    &= k\sum_{l'm'} \frac{4\pi}{2l+1} Y_{l'm'}(\Omega)
    \qty(
    \int_0^{r} dr'\frac{{r'}^{l+2}}{r^{l+1}}f_{l'm'}(r) +
    \int_{r}^{\infty} dr'\frac{r^l}{{r'}^{l-1}}f_{l'm'}(r) 
    )
  \end{split}
\end{equation}
d.h.\ alle Terme wo $l\neq l'$ oder $m\neq m'$ fallen wegen dem Kronecker-delta weg. Wir werden aber weiter wieder von eine allgemeine Ladungsdichte 
ausgehen.

Ist die Ladungsdichte nun nach Innen oder Nach außen beschränkt, dann kann
man $r_>$ und $r_<$ eindeutig als $r$ oder $r'$ festlegen in bestimmte 
Raumbereiche. Hat man z.B. eine geladene Kugelschale wobei die 
Ladungsdichte, noch von $\theta$ und $\varphi$ abhängen darf, so kann man
das Potential im Inneren mit nur dem $\int_r^\infty dr$ Integral beschreiben, und
dem Außenraum mit durch das $\int_0^r dr$ Integral. Es macht manchmal 
sogar kein Sinn die Lösungen der beide Raumbereiche gleichzeitig zu 
berechnen weil die Lösung für das Innere Raumbereich im äußeren Raumbereich
divergiren kann oder umgekehrt. Wir Teilen hier also zunächst die 
Raumgebiete auf und finden Multipolmomente für dem innen und außen Räume 
$q_<^{lm}$ und $q_>^{lm}$\\

\noindent
Im Innenraum ($r\approx 0$):
\begin{equation}
  \begin{split}
    \phi(\vb*r) 
    &= k\sum_{lm} \frac{4\pi}{2l+1} Y_{lm}(\Omega)
    \int_0^{r} dr'\frac{{r}^{l}}{{r'}^{l-1}}
    \int_\Omega d\Omega'Y^*_{lm}(\Omega')\rho(\vb*r')\\
    &= k\sum_{lm} \frac{4\pi}{2l+1}r^l q_<^{lm} Y_{lm}(\Omega) 
  \end{split}
\end{equation}
\begin{center}
  mit $\ds q_<^{lm}=\int_0^\infty dr' {r'}^{1-l}
  \int_\Omega d\Omega' Y^*_{lm}(\Omega')\rho(\vb*r)$
\end{center}
\noindent
Im Außenraum ($r\gg0$):
\begin{equation}
  \begin{split}
    \phi(\vb*r) 
    &= k\sum_{lm} \frac{4\pi}{2l+1} Y_{lm}(\Omega)
    \int_0^{r} dr'\frac{{r'}^{l+2}}{r^{l+1}}
    \int_\Omega d\Omega'Y^*_{lm}(\Omega')\rho(\vb*r')\\
    &= k\sum_{lm} \frac{4\pi}{2l+1} 
    \frac{q_>^{lm}}{r^{l+1}} Y_{lm}(\Omega) 
  \end{split}
\end{equation}
\begin{center}
  mit $\ds q_>^{lm}=\int_0^\infty dr' {r'}^{l+2}
  \int_\Omega d\Omega' Y^*_{lm}(\Omega')\rho(\vb*r)$
\end{center}


\newpage
\section{Magnetostatik in Materie}%

\subsection{Magnetische Felstärke und Magnetische Polarisation}%
\label{sub:H-und-M}

Analog wie bei der Elektrostatik teilen wir nun in Medien den Gesammtstrom auf makroskopische ebene auf in eine freie und eine gebundene Stromdichte
\begin{equation}
  \vb*j(\vr) = \vb*j_f(\vr) + \vb*j_b(\vr)
\end{equation}
Dabei sind die Gebundene Ströme im Medium zum Beispiel die Elektronen die um
das Atom kreisen, sowie induzierte Ströme durch die Anwesendheit eines äußeren magnet Feldes, die jedoch auf kleine Raumgebiete beschränkt bleiben.

Nun führen die freie Ströme wieder zu einem Feld, die magnetische Feldstärke $\vb H(\vr)$, dass das magnetische Feld sehr ähnelt. Die gebundene Ströme führen zu eine Dipoldichte oder Magnetisierung $\vb M(\vr)$. Dabei läßt die Dipoldichte sich wieder verstehen als Volumenmittelung von Punktdipolen $\vb*m_i$ und gilt
\begin{equation}
  \vb M(\vr) = \frac{1}{\Delta V(\vr)} \sum_{\vb*m_i\in\Delta V(\vr)}\vb*m_i
\end{equation}
Nun wird die Magnetisierung wieder aufgeteilt in eine Spontane Magnetisierung $\vb M_{\text{sp}}$ und eine induzierte
Magnetisierung $\vb M_{\text{ind}}$ mit
\begin{equation}
  \vb M(\vr) = \vb M_\text{sp}(\vr) + \vb M_\text{ind}(\vr)
\end{equation}
Die Spontane Magnetisierung ist verantwortlich für Permanentmagneten, und ist viel bekannter als die spontane elektrische Polarisation. Obwohl in der klassiche Magnetostatik angenommen wird, dass die Dipoldichte durch Ströme verursacht wird, ist es in der Realität so, daß die Kreisströme von Elektronen
nicht ausreichend sind um Magnetismus zu beschreiben. Dafür ist tatsächlich die Spin des Elektrons zum größten Teil verantwortlich, ist also eine Quantenmechenische Beschreibung erforderlich, aber dies ist nicht Thema dieser Einführungskurs.

Ziemlich analog wie bei der Elektrostatik wird $\vb M_\text{ind}$ von einer äußeren magnetische Feldstärke induziert und man definiere
\begin{equation}
  \vb M_\text{ind}(\vr) = \frac{1}{4\pi k'}\vu*\chi_{M}(\vr, \vb H(\vr)) \cdot \vb H(\vr)
\end{equation}
Für LHI-Medien folgt nun wieder
\begin{equation}
  \label{eq:Mind}
  \vb M_\text{ind}(\vr) = \frac{\chi_M}{4\pi k'}\vb H(\vr) 
\end{equation}
Weiter gilt
\begin{equation}
  \curl \vb M(\vr) = \vb*j_b(\vr)
  \quad\text{und}\quad
  \curl \vb H(\vr) = \vb*j_f(\vr)
\end{equation}
sodass folgt
\begin{equation}
  \begin{split}
    \curl\vb B(\vr) 
    &= 4\pi k'\vb*j(\vr)\\ 
    &= 4\pi k'(\vb*j_f(\vr) +  \vb*j_b(\vr))\\
    &= 4\pi k'(\curl \vb H(\vr) + \curl \vb M(\vr))\\
    \Leftrightarrow \vb B(\vr) &= 4\pi k' (\vb H(\vr) + \vb M(\vr)) 
  \end{split}
\end{equation}
Setzt man $\vb M_\text{sp}=0$, und mit Gleichung~\ref{eq:Mind}
folgt weiter
\begin{equation}
  \begin{split}
    \vb B(\vr) &=  \frac{(1+\chi_M)}{4\pi k'} \vb H(\vr)\\
               &\stackrel{\text{SI}}{=} \mu_r\mu_0 \vb H(\vr) \qquad \mu_r=(1+\chi_M)
  \end{split}
\end{equation}



\newpage
\section{Randwertprobleme in der Magnetostatik}%
Im Vakuum gelten es die Feldgleichungen
\begin{equation}
  \begin{split}
    \div \vb B(\vr) &= 0\\
    \curl \vb B(\vr) &= 4\pi k' \vb*j(\vr)
  \end{split}
\end{equation}
Daraus folgen die Randbedingungen, analog wie beim elektrischen Feld
\begin{equation}
  \begin{split}
    \vb*n\cdot \vb B(\vr) &= 0\\
    \vb*n\times \vb B(\vr) &= 4\pi k'\vb*j^F(\vr)\\
  \end{split}
\end{equation}
Und ist die orthogonalkomponente des magnetischen Feldes immer stetig an einer Randfläche, während die Tangentialkomponente einen Sprung um $4\pi k\vb*j^F(\vr)$ machen kann in der Anwesendheit von einer nichtverschwindenden Flächenstromdichte.

In medien Gelten die Feldgleichungen
\begin{equation}
  \begin{split}
    \div \vb B(\vr) &= 0\\
    \curl \vb H(\vr) &= \vb*j_f(\vr)
  \end{split}
\end{equation}
mit $\vb*j_f(\vr)$ die freie Stromdichte und gelten die analoge Randbedingungen
\begin{equation*}
  \begin{split}
    \vb*n\cdot \vb B(\vr) &= 0\\
    \vb*n\times \vb H(\vr) &= \vb*j_f^F(\vr)\\
  \end{split}
\end{equation*}


\newpage
\section{Ohmsches Gesetz}%
\label{sec:ohmsches-gesetz}

\newpage
\section{Formelzettel}%
