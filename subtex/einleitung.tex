Diese Zusammenfassung wurde geschrieben um meine persönliches Verständniss 
zum Thema Elektrodynamik zu verbessern, denn man lernt am Meisten indem man
versucht, sich in eine lehrende weise mit die Themen auseinanderzusetzen.
Daneben finde ich persönlich das viele Skripte die ich gesehen habe einen 
unübersichtlichen überblick von der Elektrodynamik gaben, teilweise durch
eine Informationsoverload und teilweise wegen oft fehlende motivierung der
verschiedene Themenbereiche. Diese Zusammenfassung versucht ein bottom-up
konstruierte Übersicht zu geben des Themas. 

Natürlich ist eine Zusammenfassung nie so vollständig wie Skripte zur Vorlesungen oder Literatur. Auch Übungsaufgaben und Beispiele werden hier großenteils für kompaktheit aus der Zusammenfassung gelassen. Theorie verstehen und 
rechnen können sind immer zwei unterschiedliche Sachen, also zum 
vollständigen Verständniss des Themas muss man auch viele Zusammenhänge
selbst Herleiten, sodass man eine intuition aufbaut wie man die Theorie in
der Praxis anwendet. Dazu kann man am Besten Übungsaufgaben von der Üniversität widerholen oder aus Literatur Rechnenaufgaben machen.

Weil Deutsch nicht meine Muttersprache ist gibt es leider bestimmt welche Schreibfehler. Ich versuche sie so viel wie Möglich rauszunehmen wo ich sie sehe.
